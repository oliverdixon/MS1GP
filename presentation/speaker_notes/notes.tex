% OWD 2023

% Keep the larger font size on the document (12pt or above), since these notes
% will serve as prompts to a presenter during a speech. They need to be clear,
% short, legible, and visible. I also commit the unthinkable and manually modify
% margins; this is to minimise the risk of having to flip pages over during a
% slideshow delivery.

\documentclass[12pt]{article}
\usepackage[a4paper, total={6in, 9in}]{geometry}
\usepackage{amsmath}

\DeclareMathOperator{\catendo}{\text{\textbf{Endo}}}
\DeclareMathOperator{\cathask}{\text{\textbf{Hask}}}

\newcommand{\printheading}[1]{% (Takes the name of the speaker.)
        \clearpage
        \large \textbf{MS1GP Presentation Speaker Notes}
        \hfill \textit{\today}

        \vspace{.5em}
        \textsc{#1} \normalsize

        \vspace{1em}
        \hrule
        \vspace{1em}
}

\begin{document}
\printheading{Matthew Drury}
\begin{itemize}
        \item TODO
\end{itemize}

\printheading{Ben Brook}
\begin{itemize}
        \item TODO
\end{itemize}

\printheading{Oliver Dixon}
\begin{itemize}
        \item To start, we need to define some more theory (not much!). Follow
        the slide prompts for construction of the general monoidal category with
        a base category of $\mathcal{C}_0$.
        \item We need these natural transformations to induce certain properties
        on the bifunctor $\otimes$, and it may not possess these natively. In
        other words, these diagrams must commute [show the CDs very briefly].
        \item We sometimes omit the natural transformations from the tuple, as
        their particulars are seldom of especial importance.
        \item Monoids are objects in a monoidal categories together with two
        objects. [Emphasise the importance of $\mu$ having a domain of $M \times
        M$, as it is integral for the monad application is Haskell.]
        \item Again, these diagrams must commute [show very briefly].
        \item There's not much to say about functional programming in its most
        general sense besides the stipulation that functions \emph{cannot induce
        side-effects}; they must only work with the data they are given, and
        return an explicit transformation of that data. We're beginning to
        understand why strong type systems, as those in functional languages,
        interface so nicely with Category Theory!
        \item We have already seen endofunctors in MD's section, so considering
        the category of such objects should not be difficult. We've also seen
        natural transformations, although bifunctors have not yet been
        (explicitly) mentioned.
        \item Introduce Haskell as the ``canonical'' functional language, whose
        purity facilitates these direct connexions with Category Theory.
        Consider $\catendo{(\cathask)}$, which is semantically consistent, since
        monoidal categories (such as $\cathask$) require a base category, the
        product of which its bifunctor can use as its domain.
        \item Reiterate the notion of a shared mutable state, and explain how
        weaving through imperative-like sequential call structures is possible
        with monads, due to the category product being used to encode the
        function parameter and current context.
        \begin{equation*}
                \otimes \colon \cathask \times \cathask \to \cathask
        \end{equation*}

        This is the core idea of enabling stateful computation. We will not have
        time to discuss \emph{the} $\lambda$-calculus in detail, unfortunately,
        but its theory (developed by Alonzo Church, Alan Turing's supervisor at
        Cambridge!) does form the basis of much of functional paradigms, even in
        their modern forms.
        \item Take questions.
\end{itemize}
\end{document}

