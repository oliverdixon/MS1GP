% OWD, 2023.

\documentclass{beamer}
\usepackage{tikz-cd,adjustbox}

\tikzcdset{outer sep=2pt}

\usetheme{CambridgeUS}
\setbeamertemplate{caption}[numbered]
\numberwithin{figure}{section}

\DeclareMathOperator{\catobj}{ob}
\DeclareMathOperator{\catid}{id}
\DeclareMathOperator{\cathom}{hom}
\DeclareMathOperator{\catendo}{\mathbf{Endo}}
\DeclareMathOperator{\cathask}{\mathbf{Hask}}

\title[Elementary Category Theory]%
{An Investigation of Elementary Category Theory}

\subtitle{\ldots\ \fontfamily{lmr}%
        \textsc{with Selected Applications in PM and TCS.}}

\author[Dixon, Drury \& Brook]%
{Oliver~Dixon \and Matthew~Drury \and Ben~Brook}

\institute[]{Department of Mathematics, University of York}
\date{Spring Term, 2023}

\AtBeginSection
{
        \begin{frame}
                \frametitle{Presentation Overview}
                \tableofcontents[currentsection]
        \end{frame}
}

\begin{document}

\titlegraphic{\adjustbox{scale=0.7}{%
        % This could be in the preamble, but most editors that pseudo-parse the
        % TeX to provide syntax highlighting won't consider maths environments
        % unless they're inside the document proper.
        \begin{tikzcd}[sep=large, color=gray, ampersand replacement=\&]
                A
                        \arrow[r, "\alpha"]
                        \arrow[d, "\mu" swap] \&
                A^\prime
                        \arrow[r, "\alpha^\prime"]
                        \arrow[d, "\mu^\prime" swap] \&
                A^{\prime\prime}
                        \arrow[r, "\alpha^{\prime\prime}"]
                        \arrow[d, "\mu^{\prime\prime}" swap] \& {} \\
                B \arrow[r, "\beta" swap] \&
                B^\prime \arrow[r, "\beta^\prime" swap] \&
                B^{\prime\prime} \arrow[r, "\beta^{\prime\prime}" swap] \& {}
        \end{tikzcd}
        \textcolor{gray}{\Large\ldots}%
        \begin{tikzcd}[sep=large, color=gray, ampersand replacement=\&]
                {} \arrow[r, "\alpha^{(n-1)}"] \&
                A^n \arrow[d, "\mu^n"] \\
                {} \arrow[r, "\beta^{(n-1)}" swap] \& B^n
        \end{tikzcd}}%
}

\frame{\titlepage}

\section{Theoretical Underpinnings: Axiomatic Constructions [MD]}

\section{Category-Theoretic Interpretations of Familiar Structures [BB]}

\section[Further Applications: Func. Prog. \protect\& Lambda-Calculus {[OD]}]%
        {Further Applications: Functional Programming and
        \texorpdfstring{$\lambda$}{Lambda}-Calculus [OD]}
\begin{frame}
        \frametitle{Monoidal Categories (1)}
        To construct a monoidal category $\mathcal{C}$, we need to consider six
        elements:
        \begin{itemize}[<+->]
                \item A base category $\mathcal{C}_0$
                \item A bifunctor $\otimes:\mathcal{C}_0 \times \mathcal{C}_0
                        \to \mathcal{C}_0$
                \item An identity from our base category, $I \in \catobj
                \mathcal{C}_0$
                \item An associativity natural transformation:
                        $\alpha_{A,B,C} : (A \otimes B) \otimes C \to A \otimes
                        (B \otimes C)$
                \item A left-identity natural transformation:
                        $\lambda_A: I \otimes A \to A$
                \item A right-identity natural transformation:
                        $\rho_A: A \otimes I \to A$.
        \end{itemize}
        \onslide<+->{Then, $\mathcal{C}=(\mathcal{C}_0, \otimes, I, \lambda,
        \rho)$, or just $\mathcal{C}=(\mathcal{C}_0, \otimes, I)$.}
\end{frame}

\begin{frame}
        \frametitle{Monoidal Categories (2)}
        \begin{figure}
                \adjustbox{scale=0.8, center}{%
                \begin{tikzcd}[sep=large, ampersand replacement=\&]
                        \left[(A \otimes B) \otimes C \right] \otimes D
                                \arrow[r, "\alpha"]
                                \arrow[d, "\alpha\,\otimes\,\catid D" swap] \&
                        (A \otimes B) \otimes (C \otimes D)
                                \arrow[r, "\alpha"] \&
                        A \otimes \left[ B \otimes (C \otimes D) \right] \\
                        \left[ A \otimes (B \otimes C) \right] \otimes D
                                \arrow[rr, "\alpha" swap] \& \&
                        A \otimes \left[ (B \otimes C) \otimes D \right]
                                \arrow[u, "\catid A\,\otimes\,\alpha" swap]
                \end{tikzcd}}
                \caption{Associativity induced by $\alpha$}
        \end{figure}

        \vfill
        \pause
        \begin{figure}
                \adjustbox{scale=0.8, center}{%
                \begin{tikzcd}[sep=large, ampersand replacement=\&]
                        (B \otimes I) \otimes C
                                \arrow[rr, "\alpha"]
                                \arrow[rd, "\rho\,\otimes\,\catid C" swap] \& \&
                        B \otimes (I \otimes C)
                                \arrow[ld, "\catid B\,\otimes\,\lambda"] \\
                        \& B \otimes C \&
                \end{tikzcd}}
                \caption{Left- and right-identities induced by $\lambda$ and
                $\rho$}
        \end{figure}
\end{frame}

\begin{frame}
        \frametitle{Monoids (1)}
        A monoid $(M,\mu,\eta)$ is composed of:
        \begin{itemize}[<+->]
                \item Some object $M \in \catobj \mathcal{C}_0$
                \item An associated ``multiplication'' bifunctor $\mu \in
                        \cathom (M \otimes M, M)$
                \item An associated ``unit'' identity $\eta \in \cathom (I, M)$.
        \end{itemize}
\end{frame}

\begin{frame}
        \frametitle{Monoids (2)}

        ($\alpha$, $\lambda$, and $\rho$ refer to the natural isomorphic
        transformations from the parent monoidal category.)

        \begin{figure}
                \adjustbox{scale=0.8, center}{%
                \begin{tikzcd}[sep=large, ampersand replacement=\&]
                        (M \otimes M) \otimes M
                                \arrow[r, "\alpha"]
                                \arrow[d, "\mu\,\otimes\,\catid M" swap] \&
                        M \otimes (M \otimes M)
                                \arrow[r, "\catid M\,\otimes\,\mu"] \&
                        M \otimes M
                                \arrow[d, "\mu"] \\
                        M \otimes M
                                \arrow[rr, "\mu" swap] \& \& M
                \end{tikzcd}}
                \caption{Associativity of the monoid}
        \end{figure}

        \vfill
        \pause
        \begin{figure}
                \adjustbox{scale=0.8, center}{%
                \begin{tikzcd}[sep=large, ampersand replacement=\&]
                        I \otimes M
                                \arrow[r, "\eta\,\otimes\,\catid M"]
                                \arrow[rd, "\lambda" swap] \&
                        M \otimes M
                                \arrow[d, "\mu" swap] \&
                        M \otimes I
                                \arrow[l, "\catid M\,\otimes\,\eta" swap]
                                \arrow[ld, "\rho"] \\
                        \& M \&
                \end{tikzcd}}
                \caption{Left- and right-identities of the monoid}
        \end{figure}
\end{frame}

\begin{frame}
        \frametitle{Functional Programming Languages}
        \begin{itemize}[<+->]
                \item In \textit{purely} functional programming languages,
                        functions must not cause side-effects: for some input,
                        they must always return the same output, independent of
                        the state of the wider environment.
                \item There is no allowance for a shared mutable state. At first
                        glance, this causes problems, since many common
                        operations that are made easy in the imperative world,
                        are in direct contravention with this safety principle.
                \item How can monoids and monodial categories help us with
                        printing ``Hello World''?
        \end{itemize}
\end{frame}

\begin{frame}
        \frametitle{An Especially Useful Monoidal Category}
        Let's consider an example: the category of \textit{endofunctors}! We
        have already encountered endofunctors in the first section.
        \begin{itemize}[<+->]
                \item The endofunctors form a (strict) monoidal category.
                \item The normal functor composition operation becomes the
                        associated bifunctor $\otimes$.
                \item The identity functor becomes the identity element $I$.
                \item By our previous definitions, an endofunctor can be
                        interpreted as a monoid.
                \item A \textit{monad} is a monoid in this category.
                \item We can specify the base category $\mathcal{C}_0$ with
                        $\catendo{(\mathcal{C}_0)}$.
        \end{itemize}
\end{frame}

\begin{frame}
        % I can't use a math operator (\cathask or \catendo) in a title that may
        % be written to a PDF string or index; some viewers will not like it!
        \frametitle{Haskell, \textbf{Hask}, and \textbf{Endo}(\textbf{Hask})}
        \begin{center}
                \begin{minipage}{.2\textwidth}
                        \includegraphics{hasklogo.pdf}
                \end{minipage}
                \begin{minipage}{.7\textwidth}
                        The \textit{Haskell} language is a useful demonstration
                        tool, however the type systems of most languages will be
                        equally valid for these purposes. (Early versions of
                        Haskell weren't capable of I/O!)
                \end{minipage}
        \end{center}
        \pause
        \begin{itemize}[<+->]
                \item Consider the category $\cathask$: its objects are Haskell
                        types, and its morphisms are functions between types.
                        Haskell is a \textit{purely} functional language, so
                        $\cathask$ is well-defined.
                \item What about $\catendo(\cathask)$?
                \item Its objects are endofunctors on $\cathask$, and its
                        morphisms are natural transformations between these
                        functors.
                \item Thus, endofunctors are monads.
        \end{itemize}
\end{frame}

\begin{frame}
        \frametitle{In Closing: An Example}
        TODO: Why do monads allow stateful computations?
\end{frame}
\end{document}

