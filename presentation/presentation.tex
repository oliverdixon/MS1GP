% OWD, 2023.

\documentclass{beamer}
\usepackage{tikz-cd,adjustbox,listings,caption,mathtools}

\tikzcdset{outer sep=2pt}

\usetheme{CambridgeUS}
\setbeamertemplate{caption}[numbered]
\numberwithin{figure}{section}
\beamerdefaultoverlayspecification{<+->}

% Style note: please typeset named categories (e.g. Hask) in boldface, as shown
% below. For arbitrary categories (e.g. C or C_0), use \arbcat{...}.
\DeclareMathOperator{\catobj}{ob}
\DeclareMathOperator{\catid}{id}
\DeclareMathOperator{\cathom}{hom}
\newcommand{\arbcat}[1]{\ensuremath\mathcal{#1}}

\tikzset{FUNCTOR/.style = {line width=#1},%
         FUNCTOR/.default=1pt}

% Portability note: \text{\textbf{...}} is more universally accepted than
% \mathbf{...} when working with Latin fonts, especially in a Sans typeface as
% with Beamer. I'm still using DeclareMathOperator to ensure correct spacing,
% particularly around parentheses.
\DeclareMathOperator{\catendo}{\text{\textbf{Endo}}}
\DeclareMathOperator{\cathask}{\text{\textbf{Hask}}}
\DeclareMathOperator{\catset}{\text{\textbf{Set}}}
\DeclareMathOperator{\catpos}{\text{\textbf{Pos}}}
\DeclareMathOperator{\catvect}{\mathit{K}-\text{\textbf{Vect}}}
\DeclareMathOperator{\catpre}{\text{\textbf{Ord}}}
\DeclareMathOperator{\catpoint}{\text{\textbf{Based}}}
\DeclareMathOperator{\catdeduce}{\text{\textbf{Deduce}}}
\DeclareMathOperator{\catfun}{\text{\textbf{Fun}}}

\lstset{language = Haskell,
        showstringspaces = false,%
        columns = flexible,%
        basicstyle = {\small\ttfamily},%
        keywordstyle = \color{blue},%
        stringstyle = \color{darkgray}
}

\title[Elementary Category Theory]%
{An Investigation of Elementary Category Theory}

\subtitle{\ldots\ \fontfamily{lmr}%
        \textsc{with Selected Applications in PM and TCS.}}

\author[Dixon, Drury \& Brook]%
{Oliver~Dixon \and Matthew~Drury \and Ben~Brook}

\institute[]{Department of Mathematics, University of York}
\date{Spring Term, 2023}

\AtBeginSection{%
        \begin{frame}
                \frametitle{Presentation Overview}
                \tableofcontents[currentsection]
        \end{frame}
}

\newcommand{\infinitesnake}[1]{%
        \adjustbox{scale=0.7}{%
        \begin{tikzcd}[sep=large, color=#1, ampersand replacement=\&]
                A
                        \arrow[r, "\alpha"]
                        \arrow[d, "\mu" swap] \&
                A^\prime
                        \arrow[r, "\alpha^\prime"] \&
                A^{\prime\prime}
                        \arrow[r, "\alpha^{\prime\prime}"]
                        \arrow[d, "\mu^{\prime\prime}" swap] \& {} \\
                B
                        \arrow[r, "\beta" swap] \&
                B^\prime
                        \arrow[r, "\beta^\prime" swap]
                        \arrow[u, "\mu^\prime" swap] \&
                B^{\prime\prime}
                        \arrow[r, "\beta^{\prime\prime}" swap] \& {}
        \end{tikzcd}
        \textcolor{#1}{\Large\ldots}%
        \begin{tikzcd}[sep=large, color=#1, ampersand replacement=\&]
                {} \arrow[r, "\alpha^{(n-1)}"] \&
                A^n \\
                {} \arrow[r, "\beta^{(n-1)}" swap] \&
                B^n \arrow[u, "\mu^n" swap]
        \end{tikzcd}}%
}

\titlegraphic{\infinitesnake{gray}}

\begin{document}

\frame{\titlepage}

\section{Theoretical Underpinnings: Axiomatic Constructions [MD]}

\begin{frame}
        \frametitle{The Intuition: What are Categories? (1)}
        \begin{itemize}
                \item Categories are a remarkably mathematical way of defining
                        abstract objects, and ways to morph between those
                        objects.
                \item The power of categories, and the associated theories, lie
                        in their generality: objects can be any mathematically
                        definable structure, and morphisms may be any process or
                        algorithm by which you move from one object to another.
                \item At first glance, morphisms appear indistinguishable from
                        functions. Whilst traditional sets and their functions
                        certainly do form a category, as we will see later, the
                        expanse of Category Theory extends far beyond naive Set
                        Theory!
        \end{itemize}
\end{frame}

\begin{frame}
        \frametitle{The Intuition: What are Categories? (2)}
        A useful feature in Category Theory arrives in the form of
        \emph{commutative diagrams}. Objects are shown as labels, and morphisms
        are shown, intuitively, as directed arrows between their domain (source
        object) and codomain (destination object).
        \pause
        \begin{figure}
                \begin{tikzcd}[sep=large, ampersand replacement=\&]
                        A
                                \arrow[loop left, distance=3em, "1_A"]
                                \arrow[r, "a"] \&
                        B
                                \arrow[loop right, distance=3em, "1_B"]
                                \arrow[d, "b"] \\
                        D
                                \arrow[loop left, distance=3em, "1_D"]
                                \arrow[u, "d"] \&
                        C
                                \arrow[loop right, distance=3em, "1_C"]
                                \arrow[l, "c"]
                \end{tikzcd}%
                \caption{A commutative diagram for a simple, four-object
                        category.}
        \end{figure}
        Identity morphisms are shown explicitly here, however they are usually
        omitted to maintain brevity.
\end{frame}

\begin{frame}
        \frametitle{The Definition: The Axioms of a Category}
        For any category $\arbcat{C}$, we need to consider two important sets:
        \begin{itemize}
                \item \emph{The obset}: The set of objects in $\arbcat{C}$,
                        denoted $\catobj{C}$.
                \item \emph{The homset}: The set of (homeo)morphisms in
                        $\arbcat{C}$, denoted $\cathom{\arbcat{C}}$. Morphisms
                        are sometimes called ``arrows''.
        \end{itemize}
        \onslide<+->{In practice, the morphisms are the defining feature of any
        given category. As with existing algebraic structures, we do need to
        satisfy some laws to have a well-defined category:}
        \begin{itemize}
                \item Each object $A$ must have an identity morphism $1_A \colon
                        A \to A$.
                \item The composition of morphisms, where applicable, must be an
                        associative operation.
                \item The usual unit law must hold: $f \circ 1_A = f = 1_B \circ
                        f$ for all $f \colon A \to B$.
        \end{itemize}
\end{frame}

\begin{frame}
        \frametitle{Functors (1)}
        Being able to define multiple categories, it would be useful to have a
        method of going between them. Just as \emph{functions} serve as mappings
        between sets, and \emph{linear maps} serve as mappings between vector
        spaces, \emph{functors} can define relationships between categories.
        \pause
        \begin{itemize}
                \item A functor $F \colon \arbcat{C} \to \arbcat{D}$ maps the
                        objects and morphisms of $\arbcat{C}$ to corresponding
                        entities in $\arbcat{D}$.
                \item These mappings may be chosen arbitrarily, subject to three
                        coherence conditions:
                        \begin{enumerate}
                                \item Preservation of domains and codomains:
                                        $F(f \colon a \to b) = F(f) \colon F(A)
                                        \to F(B)$
                                \item Preservation of identity:
                                        $F(1_A) = 1_F(A)$
                                \item Preservation of composition:
                                        $F(f \circ g) = F(f) \circ F(g)$
                        \end{enumerate}
        \end{itemize}
\end{frame}

\begin{frame}
        \frametitle{Functors (2)}
        A pair of connected commutative diagrams can pictorially demonstrate the
        role of a functor $F \colon \arbcat{C} \to \arbcat{D}$:

        \begin{figure}
                $\underbrace{\begin{tikzcd}[sep=large, ampersand replacement=\&]
                        A \arrow[r, "f"] \arrow[rd, "g \circ f" swap]
                        \& B \arrow[d, "g"] \\
                        \& C
                \end{tikzcd}}_{\substack{\\[.2em]\text{Domain Category }%
                        \arbcat{C}}}$
                \begin{tikzcd}[sep=large, ampersand replacement=\&]
                        {} \arrow[rr, FUNCTOR, "F"] \& \& {}
                \end{tikzcd}
                $\underbrace{\begin{tikzcd}[sep=large, ampersand replacement=\&]
                        F(B) \arrow[d, "F(g)" swap] \&
                        F(A) \arrow[l, "F(f)" swap] \arrow[dl, "F(g \circ f)"]
                        \\ F(C) \&
                \end{tikzcd}}_{\substack{\\[.2em]\text{Codomain Category }%
                        \arbcat{D}}}$%
                \caption{A functor $F$ between simple categories $\arbcat{C}$
                        and $\arbcat{D}$}
        \end{figure}
        \pause
        \begin{itemize}
                \item An \emph{endofunctor} is a functor whose domain and
                codomain is the same. This concept is simple, but it will become
                very important shortly!
        \end{itemize}
\end{frame}

\begin{frame}
        \frametitle{The Category Product}
        It would be similarly useful to combine two categories and in some
        meaningful way to produce a new category.
        \pause
        \begin{itemize}
                \item Analogous to the Cartesian product, we have the
                        \emph{category product}, denoted $\arbcat{C} \times
                        \arbcat{D}$.
                \item The objects of $\arbcat{C} \times \arbcat{D}$ have the
                        form $(c,d)$ for $c \in \catobj \arbcat{C}$ and $d \in
                        \catobj \arbcat{D}$. Morphisms have the form $(f,g)
                        \colon (c,d) \to (c^\prime,d^\prime)$, where
                        $f \colon c \to c^\prime \in \catobj \arbcat{C}$ and
                        $g \colon d \to d^\prime \in \catobj \arbcat{D}$.
                \item $\arbcat{C} \times \arbcat{D}$ can be easily shown to
                        satisfy the category axioms; hence, $\times$ is capable
                        of producing a new category.
                \item We also have two obvious \emph{projection functors}
                        $\pi_\arbcat{C}$ and $\pi_\arbcat{D}$ from the product
                        to the original constituent categories:

                        \vspace{1em}
                        \begin{figure}
                                \begin{tikzcd}[sep=large,
                                                ampersand replacement=\&]
                                        \arbcat{C} \&
                                        \arbcat{C} \times \arbcat{D}
                                                \arrow[l, "\pi_\arbcat{C}"]
                                                \arrow[r, "\pi_\arbcat{D}" swap]
                                        \& \arbcat{D}
                                \end{tikzcd}
                        \end{figure}
        \end{itemize}
\end{frame}

\begin{frame}
        \frametitle{Natural Transformations}
        In the standard tune of categorical thinking, we can push the
        abstraction further, and consider the \emph{category of functors}
        between two fixed categories $\arbcat{C}$ and $\arbcat{D}$.
        \pause
        \begin{itemize}
                \item Consider categories $\arbcat{C}$, $\arbcat{D}$, and
                        functors $F, G \colon \arbcat{C} \to \arbcat{D}$. Then,
                        consider the category formed with these functors as
                        objects. Denote this with
                        $\catfun{(\arbcat{C},\arbcat{D})}$.
                \item The formal definition of natural transformations is
                        slightly technical, however, for intuition, it suffices
                        to think of natural transformations as morphisms between
                        functors in $\catfun{(\arbcat{C},\arbcat{D})}$.
                \item As usual, these morphisms meet certain coherence
                        conditions; most of the time these are trivially
                        satisfiable.
        \end{itemize}
\end{frame}

\section{Category-Theoretic Interpretations of Familiar Structures [BB]}
\begin{frame}
        \frametitle{Category of Sets}
        \begin{itemize}
                \item Let $A,B,C,D$ be sets, and let $f,g,h$ be (set-theoretic)
                        functions such that $f \colon A \to B$, $g \colon B \to
                        C$, and $h \colon C \to D$.
                \item Immediately, we have that
                        \begin{equation*}
                                \left( f \circ g \right) \circ h = f \circ
                                \left( g \circ h \right),
                        \end{equation*}
                        due to the normal associativity property of function
                        composition. We also have the obvious identity morphism
                        $i \in \cathom(A,A)$ for some object A, henceforth
                        denoted as $1_A$.
                \item Hence, the category axioms are satisfied, and there is a
                        category whose objects are sets and morphisms are the
                        functions between those sets. We'll denote this category
                        $\catset$ moving forwards.
        \end{itemize}
\end{frame}

\begin{frame}
        \frametitle{Category of Partially Ordered Sets (``Posets'')}
        \begin{itemize}
                \item A \emph{poset} is a set $A$ with a relation, $\sim_A$,
                        that is reflexive, transitive, and antisymmetric.
                \item A monotone map $m \colon A \to B $ is an
                        order-preserving function such that
                        \begin{equation*}
                                (\forall a, b\in A)\,(a \sim_A b)
                                \implies m(a) \sim_B m(b).
                        \end{equation*}
                \item We have a category $\catpos$ whose objects are posets and
                        morphisms are monotone maps between those posets. The
                        identity morphism exists for all objects due to the
                        reflexivity of $\sim_A$, and composition follows
                        similarly due to transitivity.
                \item Analogous connexions can be drawn for sets equipped with
                        preording relations ($\catpre$), pointed sets
                        ($\catpoint$), etc.
        \end{itemize}
\end{frame}

\begin{frame}
        \frametitle{Category of Vector Spaces}
        \begin{itemize}
                \item We have a category, $\catvect$, whose objects are
                        \emph{vector spaces} over some field $K$ and morphisms
                        are \emph{linear mappings} between those vector spaces.
                \item We take a linear map to be a function, $f$, between two
                        vector spaces, such that vector addition and scalar
                        multiplication in $K$ is preserved. That is,
                        \begin{alignat*}{2}
                                f(\vec{u}+\vec{v}) &= f(\vec{u})+f(\vec{v});
                                \qquad && \text{[\textbf{Additivity}]} \\
                                f(\lambda\vec{u}) &= \lambda f(\vec{u}),
                                \qquad && \text{[\textbf{Homogeinity}]}
                        \end{alignat*}
                        where $\vec{u},\vec{v}\in K^n \text{ and }\lambda\in K$.
                \item This only applied to \emph{finite-dimensional} vector
                        spaces. Banach spaces and Functional Analysis are out of
                        the scope of this talk! ($n\in\mathbb{N}$)
        \end{itemize}
\end{frame}

\begin{frame}
        \frametitle{The Category of Deduction and Deducability}
        All of these examples are taken, more or less directly, from constructs
        expressable purely in non-categorical/algebraic terms. The power of
        Category Theory lies in its ability to abstract arbitrarily, so what
        type of non-algebraic structures might we want to represent?
        \pause
        \begin{itemize}
                \item How about the Category of Deducability? Objects are
                        mathematical statements embedded in some proof system,
                        and morphisms represent \emph{the existence of a
                        deduction}, in a fixed theory, between antecedents and
                        consequents.
                \item A richer category is the Category of Deduction,
                        $\catdeduce$. For every pair of objects, a corresponding
                        morphism in the homset represents a \emph{specific
                        manner of deduction} between its domain and codomain.
                        Deductive cancelling is equivalent to forging morphisms
                        together!
        \end{itemize}
\end{frame}

\begin{frame}
        \frametitle{The True Isomorphism! (1)}
        Although pervasive throughout Abstract Algebra and Set Theory, the
        ``correct'' definition of a general isomorphism is expressed in purely
        category-theoretic terms, demonstrating the existence of an \emph{undo}
        morphism, such that we have elements in $\cathom(A,C)$ and
        $\cathom(C,A)$.
        \pause
        \begin{figure}
                \begin{tikzcd}[sep=large, ampersand replacement=\&]
                        A \arrow[bend right=40, rr, "f" swap] \&
                        B \arrow[l, "g" swap] \arrow[r, "h"] \&
                        C \arrow[bend right=60, ll, "i" swap]
                \end{tikzcd}
                \caption{$f$ and $i$ are \emph{isomorphisms}; $A$ and $C$ are
                        \emph{isomorphic}.}
        \end{figure}
        \pause
        Isomorphisms are easy to spot with well-drawn commutative diagrams!
\end{frame}

\begin{frame}
        \frametitle{The True Isomorphism! (2)}
        In certain categories, this isomorphic equivalence can be used to deduce
        particular properties of objects in the corresponding category. Here we
        consider the Category of Deduction, $\catdeduce$:
        \pause
        \begin{figure}
                \begin{tikzcd}[sep=large, ampersand replacement=\&]
                        \phi \arrow[r, "\alpha" swap] \&
                        \psi \arrow[r, "\beta" swap] \&
                        \theta \arrow[bend right=40, ll, "\gamma" swap]
                \end{tikzcd}
                \caption{$\phi$ and $\theta$ are logically equivalent through
                deductions $[\alpha,\beta]$ and $\gamma$.}
        \end{figure}
        \pause
        Using deductive cancelling, $\alpha$ and $\beta$ can be forged into a
        single morphism, with $\psi$ being embedded into the environmental
        axioms of $\phi$.  Given our knowledge of the nature of $\catobj%
        {(\catdeduce)}$, we know that
        \begin{equation*}
                (\phi \vdash \theta) \land (\theta \vdash \phi) \iff
                        \phi \equiv \theta \iff \phi \cong \theta.
        \end{equation*}
\end{frame}

\section[Further Applications: Functional Programming {[OD]}]%
        {Further Applications: Functional Programming and
        \texorpdfstring{$\lambda$}{Lambda}-Calculus [OD]}
\begin{frame}
        \frametitle{Monoidal Categories (1)}
        To construct a monoidal category $\arbcat{C}$, we need to consider six
        elements:
        \begin{itemize}
                \item A base category $\arbcat{C}_0$
                \item A bifunctor $\otimes:\arbcat{C}_0 \times \arbcat{C}_0
                        \to \arbcat{C}_0$
                \item An identity from our base category, $I \in \catobj
                \arbcat{C}_0$
                \item An associativity natural transformation:
                        $\alpha_{A,B,C} : (A \otimes B) \otimes C \to A \otimes
                        (B \otimes C)$
                \item A left-identity natural transformation:
                        $\lambda_A: I \otimes A \to A$
                \item A right-identity natural transformation:
                        $\rho_A: A \otimes I \to A$.
        \end{itemize}
        \onslide<+->{Then, $\arbcat{C}=(\arbcat{C}_0, \otimes, I, \alpha,
        \lambda, \rho)$, or just $\arbcat{C}=(\arbcat{C}_0, \otimes, I)$.}
\end{frame}

\begin{frame}
        \frametitle{Monoidal Categories (2)}
        \begin{figure}
                \adjustbox{scale=0.95}{\begin{tikzcd}[ampersand replacement=\&]
                        \left[(A \otimes B) \otimes C \right] \otimes D
                                \arrow[r, "\alpha"]
                                \arrow[d, "\alpha\,\otimes\,\catid D" swap] \&
                        (A \otimes B) \otimes (C \otimes D)
                                \arrow[r, "\alpha"] \&
                        A \otimes \left[ B \otimes (C \otimes D) \right] \\
                        \left[ A \otimes (B \otimes C) \right] \otimes D
                                \arrow[rr, "\alpha" swap] \& \&
                        A \otimes \left[ (B \otimes C) \otimes D \right]
                                \arrow[u, "\catid A\,\otimes\,\alpha" swap]
                \end{tikzcd}}%
                \caption{Associativity induced by $\alpha$}
        \end{figure}

        \vfill
        \pause
        \begin{figure}
                % An adjustbox isn't necessary here, but I'm using it to scale
                % the diagram to the same level as the other one on the same
                % slide. It's immediately noticeable if they're different sizes.
                \adjustbox{scale=0.95, center}{%
                \begin{tikzcd}[sep=large, ampersand replacement=\&]
                        (B \otimes I) \otimes C
                                \arrow[rr, "\alpha"]
                                \arrow[rd, "\rho\,\otimes\,\catid C" swap] \& \&
                        B \otimes (I \otimes C)
                                \arrow[ld, "\catid B\,\otimes\,\lambda"] \\
                        \& B \otimes C \&
                \end{tikzcd}}
                \caption{Left- and right-identities induced by $\lambda$ and
                $\rho$}
        \end{figure}
\end{frame}

\begin{frame}
        \frametitle{Monoids (1)}
        A monoid $(M,\mu,\eta)$ is composed of:
        \begin{itemize}
                \item Some object $M \in \catobj \arbcat{C}$
                \item An associated ``multiplication'' bifunctor $\mu \in
                        \cathom (M \otimes M, M)$
                \item An associated ``unit'' identity $\eta \in \cathom (I, M)$.
        \end{itemize}
\end{frame}

\begin{frame}
        \frametitle{Monoids (2)}
        \begin{figure}
                \begin{tikzcd}[sep=large, ampersand replacement=\&]
                        (M \otimes M) \otimes M
                                \arrow[r, "\alpha"]
                                \arrow[d, "\mu\,\otimes\,\catid M" swap] \&
                        M \otimes (M \otimes M)
                                \arrow[r, "\catid M\,\otimes\,\mu"] \&
                        M \otimes M
                                \arrow[d, "\mu"] \\
                        M \otimes M
                                \arrow[rr, "\mu" swap] \& \& M
                \end{tikzcd}%
                \caption{Associativity of the monoid}
        \end{figure}

        \vfill
        \pause
        \begin{figure}
                \begin{tikzcd}[sep=large, ampersand replacement=\&]
                        I \otimes M
                                \arrow[r, "\eta\,\otimes\,\catid M"]
                                \arrow[rd, "\lambda" swap] \&
                        M \otimes M
                                \arrow[d, "\mu" swap] \&
                        M \otimes I
                                \arrow[l, "\catid M\,\otimes\,\eta" swap]
                                \arrow[ld, "\rho"] \\
                        \& M \&
                \end{tikzcd}%
                \caption{Left- and right-identities of the monoid}
        \end{figure}
\end{frame}

\begin{frame}
        \frametitle{Functional Programming Languages}
        \begin{itemize}
                \item In \textit{purely} functional programming languages,
                        functions must not cause side-effects: for some input,
                        they must always return the same output, independent of
                        the state of the wider environment.
                \item There is no allowance for a shared mutable state. At first
                        glance, this causes problems, since many common
                        operations that are made easy in the imperative world,
                        are in direct contravention with this safety principle.
                \item How can monoids and monodial categories help us with
                        printing ``Hello World''?
        \end{itemize}
\end{frame}

\begin{frame}
        \frametitle{An Especially Useful Monoidal Category}
        Let's consider an example: the category of \textit{endofunctors}! We
        have already encountered endofunctors in the first section.
        \begin{itemize}
                \item The endofunctors form a (strict) monoidal category.
                \item The normal functor composition operation becomes the
                        associated bifunctor $\otimes$.
                \item The identity functor becomes the identity element $I$.
                \item By our previous definitions, an endofunctor can be
                        interpreted as a monoid.
                \item A \textit{monad} is a monoid in this category.
                \item We can specify the base category $\arbcat{C}_0$ with
                        $\catendo{(\arbcat{C}_0)}$.
        \end{itemize}
\end{frame}

\begin{frame}
        % I can't use a math operator (\cathask or \catendo) in a title that may
        % be written to a PDF string or index; some viewers will not like it!
        \frametitle{Haskell, \textbf{Hask}, and \textbf{Endo}(\textbf{Hask})}
        \begin{center}
                \begin{minipage}{.2\textwidth}
                        \includegraphics{hasklogo.pdf}
                \end{minipage}
                \begin{minipage}{.7\textwidth}
                        The \textit{Haskell} language is a useful demonstration
                        tool, however the type systems of most languages will be
                        equally valid for these purposes. (Early versions of
                        Haskell weren't capable of I/O!)
                \end{minipage}
        \end{center}
        \pause
        \begin{itemize}
                \item Consider the category $\cathask$: its objects are Haskell
                        types, and its morphisms are functions between types.
                        Haskell is a \textit{purely} functional language, so
                        $\cathask$ is well-defined.
                \item What about $\catendo(\cathask)$?
                \item Its objects are endofunctors on $\cathask$, and its
                        morphisms are natural transformations between these
                        functors.
                \item Thus, endofunctors on $\cathask$ are monads.
        \end{itemize}
\end{frame}

\begin{frame}
        \frametitle{In Closing: Practical Notes}
        We have established a tangential, perhaps contrived, link between
        Category Theory and functional programming, the latter of which is 
        based on the $\lambda$-calculus. Why is this interdisciplinary
        observation useful?
        \pause
        \begin{itemize}
                \item In practice, monads allow stateful calculation. Unlike
                the mindless emulation of imperative languages, this is achieved
                without breaking function purity. This can be seen intuitively
                as \textit{threading state though a chain of functions}.
                \item This is akin to ``following the arrows'' on a commutative
                diagram of $\catendo{(\cathask)}$ using the category product
                $\cathask \times \cathask$ to pass the desired parameter with
                the current state, and returning the transformed state,
                recalling that the endofunctors are morphisms in $\cathask$.
        \end{itemize}
\end{frame}

\begin{frame}[fragile]
        \frametitle{Any Questions?}
        \begin{figure}
                \infinitesnake{black}
                \vspace{.5em}
                \caption*{``\textit{The Categorical Snake of Morphistic
                        Infinitude}''}
        \end{figure}
\end{frame}
\end{document}

