% OWD, 2023.

\documentclass{beamer}
\usepackage{tikz-cd,adjustbox,listings,caption,mathtools}

\tikzcdset{outer sep=2pt}

\usetheme{CambridgeUS}
\setbeamertemplate{caption}[numbered]
\numberwithin{figure}{section}
\beamerdefaultoverlayspecification{<+->}

\DeclareMathOperator{\catobj}{ob}
\DeclareMathOperator{\catid}{id}
\DeclareMathOperator{\cathom}{hom}

% Portability note: \text{\textbf{...}} is more universally accepted than
% \mathbf{...} when working with Latin fonts, especially in a Sans typeface as
% with Beamer. I'm still using DeclareMathOperator to ensure correct spacing,
% particularly around parentheses.
\DeclareMathOperator{\catendo}{\text{\textbf{Endo}}}
\DeclareMathOperator{\cathask}{\text{\textbf{Hask}}}
\DeclareMathOperator{\catset}{\text{\textbf{Set}}}
\DeclareMathOperator{\catpos}{\text{\textbf{Pos}}}
\DeclareMathOperator{\catvect}{\mathit{K}-\text{\textbf{Vect}}}
\DeclareMathOperator{\catpre}{\text{\textbf{Ord}}}
\DeclareMathOperator{\catpoint}{\text{\textbf{Based}}}
\DeclareMathOperator{\catdeduce}{\text{\textbf{Deduce}}}

\lstset{language = Haskell,
        showstringspaces = false,%
        columns = flexible,%
        basicstyle = {\small\ttfamily},%
        keywordstyle = \color{blue},%
        stringstyle = \color{darkgray}
}

\title[Elementary Category Theory]%
{An Investigation of Elementary Category Theory}

\subtitle{\ldots\ \fontfamily{lmr}%
        \textsc{with Selected Applications in PM and TCS.}}

\author[Dixon, Drury \& Brook]%
{Oliver~Dixon \and Matthew~Drury \and Ben~Brook}

\institute[]{Department of Mathematics, University of York}
\date{Spring Term, 2023}

\AtBeginSection{%
        \begin{frame}
                \frametitle{Presentation Overview}
                \tableofcontents[currentsection]
        \end{frame}
}

\newcommand{\infinitesnake}[1]{%
        \adjustbox{scale=0.7}{%
        \begin{tikzcd}[sep=large, color=#1, ampersand replacement=\&]
                A
                        \arrow[r, "\alpha"]
                        \arrow[d, "\mu" swap] \&
                A^\prime
                        \arrow[r, "\alpha^\prime"] \&
                A^{\prime\prime}
                        \arrow[r, "\alpha^{\prime\prime}"]
                        \arrow[d, "\mu^{\prime\prime}" swap] \& {} \\
                B
                        \arrow[r, "\beta" swap] \&
                B^\prime
                        \arrow[r, "\beta^\prime" swap]
                        \arrow[u, "\mu^\prime" swap] \&
                B^{\prime\prime}
                        \arrow[r, "\beta^{\prime\prime}" swap] \& {}
        \end{tikzcd}
        \textcolor{#1}{\Large\ldots}%
        \begin{tikzcd}[sep=large, color=#1, ampersand replacement=\&]
                {} \arrow[r, "\alpha^{(n-1)}"] \&
                A^n \\
                {} \arrow[r, "\beta^{(n-1)}" swap] \&
                B^n \arrow[u, "\mu^n" swap]
        \end{tikzcd}}%
}

\titlegraphic{\infinitesnake{gray}}

\begin{document}

\frame{\titlepage}

\section{Theoretical Underpinnings: Axiomatic Constructions [MD]}

\begin{frame}
        \frametitle{What are Categories?}
        \emph{MD: TODO!}

        You need to cover: categories (objects and morphisms), the axioms of a
        category, functors between categories (and a word on endofunctors), and
        any other theory you feel like including. The more, the better, assuming
        you can speak quickly to fit it in three minutes :)
\end{frame}

\begin{frame}
        
\end{frame}
\section{Category-Theoretic Interpretations of Familiar Structures [BB]}
\begin{frame}
        \frametitle{Category of Sets}
        \begin{itemize}
                \item Let $A,B,C,D$ be sets, and let $f,g,h$ be (set-theoretic)
                        functions such that $f \colon A \to B$, $g \colon B \to
                        C$, and $h \colon C \to D$.
                \item Immediately, we have that
                        \begin{equation*}
                                \left( f \circ g \right) \circ h = f \circ
                                \left( g \circ h \right),
                        \end{equation*}
                        due to the normal associativity property of function
                        composition. We also have the obvious identity morphism
                        $i \in \cathom(A,A)$ for some object A, henceforth
                        denoted as $1_A$.
                \item Hence, the category axioms are satisfied, and there is a
                        category whose objects are sets and morphisms are the
                        functions between those sets. We'll denote this category
                        $\catset$ moving forwards.
        \end{itemize}
\end{frame}

\begin{frame}
        \frametitle{Category of Partially Ordered Sets (``Posets'')}
        \begin{itemize}
                \item A \emph{poset} is a set $A$ with a relation, $\sim_A$,
                        that is reflexive, transitive, and antisymmetric.
                \item A monotone map $m \colon A \to B $ is an
                        order-preserving function such that
                        \begin{equation*}
                                (\forall a, b\in A)\,(a \sim_A b)
                                \implies m(a) \sim_B m(b).
                        \end{equation*}
                \item We have a category $\catpos$ whose objects are posets and
                        morphisms are monotone maps between those posets. The
                        identity morphism exists for all objects due to the
                        reflexivity of $\sim_A$, and composition follows
                        similarly due to transitivity.
                \item Analogous connexions can be drawn for sets equipped with
                        preording relations ($\catpre$), pointed sets
                        ($\catpoint$), etc.
        \end{itemize}
\end{frame}

\begin{frame}
        \frametitle{Category of Vector Spaces}
        \begin{itemize}
                \item We have a category, $\catvect$, whose objects are
                        \emph{vector spaces} over some field $K$ and morphisms
                        are \emph{linear mappings} between those vector spaces.
                \item We take a linear map to be a function, $f$, between two
                        vector spaces, such that vector addition and scalar
                        multiplication in $K$ is preserved. That is,
                        \begin{alignat*}{2}
                                f(\vec{u}+\vec{v}) &= f(\vec{u})+f(\vec{v});
                                \qquad && \text{[\textbf{Additivity}]} \\
                                f(\lambda\vec{u}) &= \lambda f(\vec{u}),
                                \qquad && \text{[\textbf{Homogeinity}]}
                        \end{alignat*}
                        where $\vec{u},\vec{v}\in K^n \text{ and }\lambda\in K$.
                \item This only applied to \emph{finite-dimensional} vector
                        spaces. Banach spaces and Functional Analysis are out of
                        the scope of this talk! ($n\in\mathbb{N}$)
        \end{itemize}
\end{frame}

\begin{frame}
        \frametitle{The Category of Deduction and Deducability}
        \emph{BB: TODO!}

        This is my favourite example, so do it well :)
\end{frame}

\begin{frame}
        \frametitle{The True Isomorphism! (1)}
        Although pervasive throughout Abstract Algebra and Set Theory, the
        ``correct'' definition of a general isomorphism is expressed in purely
        category-theoretic terms, demonstrating the existence of an \emph{undo}
        morphism, such that we have elements in $\cathom(A,C)$ and
        $\cathom(C,A)$.
        \pause
        \begin{figure}
                \adjustbox{scale=0.8, center}{%
                \begin{tikzcd}[sep=large, ampersand replacement=\&]
                        A \arrow[bend right=40, rr, "f" swap] \&
                        B \arrow[l, "g" swap] \arrow[r, "h"] \&
                        C \arrow[bend right=60, ll, "i" swap]
                \end{tikzcd}}
                \caption{$f$ and $i$ are \emph{isomorphisms}; $A$ and $C$ are
                        \emph{isomorphic}.}
        \end{figure}
        \pause
        Isomorphisms are easy to spot with well-drawn commutative diagrams!
\end{frame}

\begin{frame}
        \frametitle{The True Isomorphism! (2)}
        In certain categories, this isomorphic equivalence can be used to deduce
        particular properties of objects in the corresponding category. Here we
        consider the Category of Deducability, $\catdeduce$:
        \pause
        \begin{figure}
                \adjustbox{scale=0.8, center}{%
                \begin{tikzcd}[sep=large, ampersand replacement=\&]
                        \phi \arrow[r, "\alpha"] \&
                        \psi \arrow[r, "\beta"] \&
                        \theta \arrow[bend right=40, ll, "\gamma" swap]
                \end{tikzcd}}
                \caption{$\phi$ and $\theta$ are logically equivalent through
                deductions $[\alpha,\beta]$ and $\gamma$.}
        \end{figure}
        \pause
        Using deductive cancelling, $\alpha$ and $\beta$ can be forged into a
        single morphism, with $\psi$ being embedded into the environmental
        axioms of $\phi$.  Given our knowledge of the nature of $\catobj%
        {(\catdeduce)}$, we know that
        \begin{equation*}
                (\phi \vdash \theta) \land (\theta \vdash \phi) \iff
                        \phi \equiv \theta \iff \phi \cong \theta.
        \end{equation*}
\end{frame}

\section[Further Applications: Functional Programming {[OD]}]%
        {Further Applications: Functional Programming and
        \texorpdfstring{$\lambda$}{Lambda}-Calculus [OD]}
\begin{frame}
        \frametitle{Monoidal Categories (1)}
        To construct a monoidal category $\mathcal{C}$, we need to consider six
        elements:
        \begin{itemize}
                \item A base category $\mathcal{C}_0$
                \item A bifunctor $\otimes:\mathcal{C}_0 \times \mathcal{C}_0
                        \to \mathcal{C}_0$
                \item An identity from our base category, $I \in \catobj
                \mathcal{C}_0$
                \item An associativity natural transformation:
                        $\alpha_{A,B,C} : (A \otimes B) \otimes C \to A \otimes
                        (B \otimes C)$
                \item A left-identity natural transformation:
                        $\lambda_A: I \otimes A \to A$
                \item A right-identity natural transformation:
                        $\rho_A: A \otimes I \to A$.
        \end{itemize}
        \onslide<+->{Then, $\mathcal{C}=(\mathcal{C}_0, \otimes, I, \alpha,
        \lambda, \rho)$, or just $\mathcal{C}=(\mathcal{C}_0, \otimes, I)$.}
\end{frame}

\begin{frame}
        \frametitle{Monoidal Categories (2)}
        \begin{figure}
                \adjustbox{scale=0.8, center}{%
                \begin{tikzcd}[sep=large, ampersand replacement=\&]
                        \left[(A \otimes B) \otimes C \right] \otimes D
                                \arrow[r, "\alpha"]
                                \arrow[d, "\alpha\,\otimes\,\catid D" swap] \&
                        (A \otimes B) \otimes (C \otimes D)
                                \arrow[r, "\alpha"] \&
                        A \otimes \left[ B \otimes (C \otimes D) \right] \\
                        \left[ A \otimes (B \otimes C) \right] \otimes D
                                \arrow[rr, "\alpha" swap] \& \&
                        A \otimes \left[ (B \otimes C) \otimes D \right]
                                \arrow[u, "\catid A\,\otimes\,\alpha" swap]
                \end{tikzcd}}
                \caption{Associativity induced by $\alpha$}
        \end{figure}

        \vfill
        \pause
        \begin{figure}
                \adjustbox{scale=0.8, center}{%
                \begin{tikzcd}[sep=large, ampersand replacement=\&]
                        (B \otimes I) \otimes C
                                \arrow[rr, "\alpha"]
                                \arrow[rd, "\rho\,\otimes\,\catid C" swap] \& \&
                        B \otimes (I \otimes C)
                                \arrow[ld, "\catid B\,\otimes\,\lambda"] \\
                        \& B \otimes C \&
                \end{tikzcd}}
                \caption{Left- and right-identities induced by $\lambda$ and
                $\rho$}
        \end{figure}
\end{frame}

\begin{frame}
        \frametitle{Monoids (1)}
        A monoid $(M,\mu,\eta)$ is composed of:
        \begin{itemize}
                \item Some object $M \in \catobj \mathcal{C}_0$
                \item An associated ``multiplication'' bifunctor $\mu \in
                        \cathom (M \otimes M, M)$
                \item An associated ``unit'' identity $\eta \in \cathom (I, M)$.
        \end{itemize}
\end{frame}

\begin{frame}
        \frametitle{Monoids (2)}

        ($\alpha$, $\lambda$, and $\rho$ refer to the natural isomorphic
        transformations from the parent monoidal category.)

        \begin{figure}
                \adjustbox{scale=0.8, center}{%
                \begin{tikzcd}[sep=large, ampersand replacement=\&]
                        (M \otimes M) \otimes M
                                \arrow[r, "\alpha"]
                                \arrow[d, "\mu\,\otimes\,\catid M" swap] \&
                        M \otimes (M \otimes M)
                                \arrow[r, "\catid M\,\otimes\,\mu"] \&
                        M \otimes M
                                \arrow[d, "\mu"] \\
                        M \otimes M
                                \arrow[rr, "\mu" swap] \& \& M
                \end{tikzcd}}
                \caption{Associativity of the monoid}
        \end{figure}

        \vfill
        \pause
        \begin{figure}
                \adjustbox{scale=0.8, center}{%
                \begin{tikzcd}[sep=large, ampersand replacement=\&]
                        I \otimes M
                                \arrow[r, "\eta\,\otimes\,\catid M"]
                                \arrow[rd, "\lambda" swap] \&
                        M \otimes M
                                \arrow[d, "\mu" swap] \&
                        M \otimes I
                                \arrow[l, "\catid M\,\otimes\,\eta" swap]
                                \arrow[ld, "\rho"] \\
                        \& M \&
                \end{tikzcd}}
                \caption{Left- and right-identities of the monoid}
        \end{figure}
\end{frame}

\begin{frame}
        \frametitle{Functional Programming Languages}
        \begin{itemize}
                \item In \textit{purely} functional programming languages,
                        functions must not cause side-effects: for some input,
                        they must always return the same output, independent of
                        the state of the wider environment.
                \item There is no allowance for a shared mutable state. At first
                        glance, this causes problems, since many common
                        operations that are made easy in the imperative world,
                        are in direct contravention with this safety principle.
                \item How can monoids and monodial categories help us with
                        printing ``Hello World''?
        \end{itemize}
\end{frame}

\begin{frame}
        \frametitle{An Especially Useful Monoidal Category}
        Let's consider an example: the category of \textit{endofunctors}! We
        have already encountered endofunctors in the first section.
        \begin{itemize}
                \item The endofunctors form a (strict) monoidal category.
                \item The normal functor composition operation becomes the
                        associated bifunctor $\otimes$.
                \item The identity functor becomes the identity element $I$.
                \item By our previous definitions, an endofunctor can be
                        interpreted as a monoid.
                \item A \textit{monad} is a monoid in this category.
                \item We can specify the base category $\mathcal{C}_0$ with
                        $\catendo{(\mathcal{C}_0)}$.
        \end{itemize}
\end{frame}

\begin{frame}
        % I can't use a math operator (\cathask or \catendo) in a title that may
        % be written to a PDF string or index; some viewers will not like it!
        \frametitle{Haskell, \textbf{Hask}, and \textbf{Endo}(\textbf{Hask})}
        \begin{center}
                \begin{minipage}{.2\textwidth}
                        \includegraphics{hasklogo.pdf}
                \end{minipage}
                \begin{minipage}{.7\textwidth}
                        The \textit{Haskell} language is a useful demonstration
                        tool, however the type systems of most languages will be
                        equally valid for these purposes. (Early versions of
                        Haskell weren't capable of I/O!)
                \end{minipage}
        \end{center}
        \pause
        \begin{itemize}
                \item Consider the category $\cathask$: its objects are Haskell
                        types, and its morphisms are functions between types.
                        Haskell is a \textit{purely} functional language, so
                        $\cathask$ is well-defined.
                \item What about $\catendo(\cathask)$?
                \item Its objects are endofunctors on $\cathask$, and its
                        morphisms are natural transformations between these
                        functors.
                \item Thus, endofunctors on $\cathask$ are monads.
        \end{itemize}
\end{frame}

\begin{frame}
        \frametitle{In Closing: Practical Notes}
        We have established a tangential, perhaps contrived, link between
        Category Theory and functional programming, the latter of which is 
        based on the $\lambda$-calculus. Why is this interdisciplinary
        observation useful?
        \pause
        \begin{itemize}
                \item In practice, monads allow stateful calculation. Unlike
                the mindless emulation of imperative languages, this is achieved
                without breaking function purity. This can be seen intuitively
                as \textit{threading state though a chain of functions}.
                \item This is akin to ``following the arrows'' on a commutative
                diagram of $\catendo{(\cathask)}$ using the category product
                $\cathask \times \cathask$ to pass the desired parameter with
                the current state, and returning the transformed state,
                recalling that the endofunctors are morphisms in $\cathask$.
        \end{itemize}
\end{frame}

\begin{frame}[fragile]
        \frametitle{Any Questions?}
        \begin{figure}
                \infinitesnake{black}
                \vspace{.5em}
                \caption*{``\textit{The Categorical Snake of Morphistic
                        Infinitude}''}
        \end{figure}
\end{frame}
\end{document}

