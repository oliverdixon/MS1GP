% OWD 2023

% DO NOT COMPILE THIS FILE IN ISOLATION! It is designed to be included by other
% TeX files, and will not produce anything on its own.
%
% Whenever you need a macro involving categories, or load a package for the
% typesetting thereof, add it here, since it's likely that most macros will be
% mutually required between the report and presentation.

\usepackage{amsmath}
\usepackage{tikz-cd} % Drawing commutative diagrams with TikZ.
\usepackage{adjustbox} % Scaling tikz-cd commutative diagrams.

\tikzcdset{outer sep=2pt}

% Style note: please typeset named categories (e.g. Hask) in boldface, as shown
% below. For arbitrary categories (e.g. C or C_0), use \arbcat{...}.
\DeclareMathOperator{\catobj}{ob}
\DeclareMathOperator{\catid}{id}
\DeclareMathOperator{\cathom}{hom}

\newcommand{\arbcat}[1]{\ensuremath\mathcal{#1}}
% Portability note: \text{\textbf{...}} is more universally accepted than
% \mathbf{...} when working with Latin fonts, especially in a Sans typeface as
% with Beamer. I'm still using DeclareMathOperator to ensure correct spacing,
% particularly around parentheses.
\DeclareMathOperator{\catendo}{\text{\textbf{Endo}}}
\DeclareMathOperator{\cathask}{\text{\textbf{Hask}}}
\DeclareMathOperator{\catset}{\text{\textbf{Set}}}
\DeclareMathOperator{\catpos}{\text{\textbf{Pos}}}
\DeclareMathOperator{\catvect}{\mathit{K}-\text{\textbf{Vect}}}
\DeclareMathOperator{\catpre}{\text{\textbf{Ord}}}
\DeclareMathOperator{\catpoint}{\text{\textbf{Based}}}
\DeclareMathOperator{\catdeduce}{\text{\textbf{Deduce}}}
\DeclareMathOperator{\catfun}{\text{\textbf{Fun}}}

