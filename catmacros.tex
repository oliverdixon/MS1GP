% OWD 2023

% DO NOT COMPILE THIS FILE IN ISOLATION! It is designed to be included by other
% TeX files, and will not produce anything on its own (other than a swathe of
% meaningless temporary files and an incomprehensible error message).
%
% Whenever you need a macro involving categories, or load a package for the
% typesetting thereof, add it here, since it's likely that most macros will be
% mutually required between the report and presentation.

\usepackage{amsmath}
\usepackage{tikz-cd} % Drawing commutative diagrams with TikZ.
\usepackage{adjustbox} % Scaling tikz-cd commutative diagrams.

\tikzcdset{outer sep=2pt} % Apply some padding to arrows in comm. diagrams.

% Provide a thicker arrow style (1pt) for functor morphisms in pseudo-CDs.
\tikzset{FUNCTOR/.style = {line width=#1},%
         FUNCTOR/.default=1pt}

% Style note: please typeset named categories (e.g. Hask) in boldface, as shown
% below. For arbitrary categories (e.g. C or C_0), use \arbcat. For the homset,
% use the built-in \hom macro (\cathom is superfluous).

\newcommand{\arbcat}[1]{\ensuremath\mathcal{#1}}
\newcommand{\arbfunc}[1]{\ensuremath #1} % I might change this to \mathbf{...}

\DeclareMathOperator{\catobj}{ob}
\DeclareMathOperator{\catid}{id}
\DeclareMathOperator{\catcat}{\mathbf{Cat}}
\DeclareMathOperator{\catendo}{\mathbf{Endo}}
\DeclareMathOperator{\cathask}{\mathbf{Hask}}
\DeclareMathOperator{\catset}{\mathbf{Set}}
\DeclareMathOperator{\catpos}{\mathbf{Pos}}
\DeclareMathOperator{\catkvect}{\mathit{K}-\mathbf{Vect}}
\DeclareMathOperator{\catfinvect}{\mathbf{FinVect}_\mathbb{R}}
\DeclareMathOperator{\catpre}{\mathbf{Ord}}
\DeclareMathOperator{\catpoint}{\mathbf{Based}}
\DeclareMathOperator{\catdeduce}{\mathbf{Deduce}}
\DeclareMathOperator{\catfun}{\mathbf{Fun}}

