% OWD 2023
%
% BEN'S REPORT

\documentclass[10pt,a4paper,reqno]{amsart}
\usepackage{xcolor}
\usepackage[foot]{amsaddr} % Move authorship information to the front page
\usepackage{caption, subcaption, float, etoolbox} % Enhanced float environments
\usepackage{minted, realboxes, listings} % Source code (Haskell) formatting

\usepackage[%
	backend = biber,
	sorting = none,
	bibencoding = utf8,
	style = alphabetic
]{biblatex}

\usepackage[%
	colorlinks = true,
	allcolors = blue,
        linktoc = page
]{hyperref}

% OWD 2023

% DO NOT COMPILE THIS FILE IN ISOLATION! It is designed to be included by other
% TeX files, and will not produce anything on its own.
%
% Whenever you need a macro involving categories, or load a package for the
% typesetting thereof, add it here, since it's likely that most macros will be
% mutually required between the report and presentation.

\usepackage{amsmath}
\usepackage{tikz-cd} % Drawing commutative diagrams with TikZ.
\usepackage{adjustbox} % Scaling tikz-cd commutative diagrams.

\tikzcdset{outer sep=2pt}

% Style note: please typeset named categories (e.g. Hask) in boldface, as shown
% below. For arbitrary categories (e.g. C or C_0), use \arbcat{...}.
\DeclareMathOperator{\catobj}{ob}
\DeclareMathOperator{\catid}{id}
\DeclareMathOperator{\cathom}{hom}

\newcommand{\arbcat}[1]{\ensuremath\mathcal{#1}}
\DeclareMathOperator{\catendo}{\mathbf{Endo}}
\DeclareMathOperator{\cathask}{\mathbf{Hask}}
\DeclareMathOperator{\catset}{\mathbf{Set}}
\DeclareMathOperator{\catpos}{\mathbf{Pos}}
\DeclareMathOperator{\catvect}{\mathit{K}-\mathbf{Vect}}
\DeclareMathOperator{\catpre}{\mathbf{Ord}}
\DeclareMathOperator{\catpoint}{\mathbf{Based}}
\DeclareMathOperator{\catdeduce}{\mathbf{Deduce}}
\DeclareMathOperator{\catfun}{\mathbf{Fun}}

 % Common category-theoretic macros

% -- Begin listing options for Haskell

\newcommand{\inlinehask}[1]{\mintinline{haskell}{#1}}
\newcommand{\barehask}[1]{\Colorbox{codebg}{\lstinline[style = inlineHask]{#1}}}
\newcommand{\haskmath}[2]{%
        \ensuremath\text{\Colorbox{codebg}{%
                \vphantom{\lstinline[basicstyle = \haskMorphStyle]{fg}}%
                \lstinline[style = inlineHask, basicstyle = #2]{#1}%
        }}%
}
\AtBeginEnvironment{listing}{\setlength{\abovecaptionskip}{.8em}}
\newcommand{\listingautorefname}{Listing}

\renewcommand{\ttdefault}{pcr} % Use Courier instead of CM for fixed-width text.
\newcommand{\haskMorphStyle}{\normalsize\ttfamily\color{haskkeyword}}
\newcommand{\haskObjStyle}{\normalsize\ttfamily\color{black}}
\newcommand{\haskFuncStyle}{\normalsize\ttfamily\color{haskkeyword}\bfseries}

\definecolor{codebg}{RGB}{220,220,220}
\definecolor{haskkeyword}{RGB}{173,46,252}

% I am using the Listings package for printing inline Haskell, and Minted for
% printing verbatim blocks. In general, Minted is far superior, but Listings is
% more crudely customisable for pseudo-code inline blocks that require very
% literal pattern-matching for given tokens.

\setminted{%
        frame = lines,
        framesep = 2mm,
        baselinestretch = 1.2,
        bgcolor = codebg,
        fontsize = \footnotesize,
        linenos
}

\lstdefinestyle{inlineHask}{%
        language = Haskell,
        alsoletter = {->, ::},
        morekeywords = {->, ::, fmap},
        keywordstyle = \color{haskkeyword}\bfseries,
        basicstyle = \footnotesize\ttfamily
}

% -- End listing options for Haskell

% \yorkemail: takes the username and outputs a clickable mailto link.
\newcommand{\yorkemail}[1]{\href{mailto:#1@york.ac.uk}{#1@york.ac.uk}}

% Make bullet points slightly smaller (they look overwhelming in a 10pt font).
\renewcommand{\labelitemi}{$\vcenter{\hbox{\tiny$\bullet$}}$}

% Add appropriately padded frames around numbered figures and sub-figures.
\numberwithin{figure}{section}
\AtBeginEnvironment{subfigure}{\vspace{2em}}
\floatstyle{boxed}
\restylefloat*{figure}

\setcounter{tocdepth}{1} % Only show major sections in the Table of Contents
\addbibresource{../report.bib} % Use a universal sources file for BibLaTeX

% -- Begin document metadata setup

\title[An Investigation of Elementary Category Theory]{An Investigation of %
        Elementary Category Theory, with applications in Pure Mathematics and %
        Theoretical Computer Science}

\author{Matthew Drury} \email{\yorkemail{md1499}}
\author{Ben Brook}     \email{\yorkemail{bb1170}}
\author{Oliver Dixon}  \email{\yorkemail{od641}}

\address{Department of Mathematics \\
        University of York \\
        United Kingdom}
\date{Spring--Summer Term, 2023}

% -- End document metadata setup
\begin{document}
\begin{abstract}
        TODO
\end{abstract}
\maketitle
\tableofcontents
\section{\for{toc}{\texorpdfstring{\textbf{[MD]~}}{}}Theoretical %
        Underpinnings: Axiomatic Constructions}
\begin{flushright}
        \textbf{[Written by Matthew Drury]}
\end{flushright}

\noindent Temporarily moved to \texttt{md\_tmp/report.tex}.

\section{\for{toc}{\texorpdfstring{\textbf{[BB]~}}{}}Category-Theoretic %
        Interpretations of Familiar Structures}
\begin{flushright}
        \textbf{[Written by Ben Brook]} 
\end{flushright}

\subsection{Introductory examples}
Now that we have established an axiomatic definition of the category, the
natural next step is the exploration of some simple examples of category theory
in action. In this section, we will cover multiple categories with objects and
morphisms that are already familiar to us and not too daunting. This will make
us more comfortable with the fleshed out examples coming later in the report.

For the first three of these examples, our objects will be
nothing but sets and morphisms nothing but functions (both with some additional
restrictions). However, it's important to remember that categories need not
contain functions nor sets. In fact, we will see this in our final introductory
example, where we do away with both entirely!

All the exemplary categories that we are going to touch on will be infinite.
This means that we won't be using any diagrams for this section of the report.

\subsection{The category of sets}
An easy to digest example to start with is the category of sets, whose objects are sets and
morphisms are functions between sets. These are the set-theoretic functions we
are familiar with and have been using throughout the Autumn Term. We will follow
the common denotation [REF] for this category: $\catset$.

We can easily verify that this category indeed satisfies the category axioms
seen in [AXIOMSREF].
\begin{itemize}
        \item There are objects, which are sets.
        \item There are morphisms between those objects, which are functions.
                Remember, however: morphisms are not necessarily functions! In fact,
                in most applications of category theory, it is rare that we see
                functions as morphisms. (REF)
        \item The composition of two functions is also a function, with its
                domain and codomain coming from the functions involved in the
                composition operation. Hence, the axiom of composition is
                satisfied.
        \item Let $A \in \catobj ( \catset )$. We have that the identity
                morphism for $A$ is $(1_A \colon A \to A) \in \hom (A, A)$.
        \item The morphisms of $\catset$ are associative because functions are
                associative. Let $A, B, C, D \in \catobj ( \catset )$ be distinct. Also, let
                $f \in \hom (A, B)$, $g \in \hom (B, C)$, and $h \in
                \hom (C, D)$. By function associativity, we have that
                \[
                        ( f \circ g ) \circ h = f \circ ( g \circ h ).
                \]
\end{itemize}
Hence, all the category axioms are satisfied. Going forwards, we will avoid the
pedantry of stating facts that are trivially true.

\subsection{The category of partially ordered sets}
A nice path to advance down is the exploration of a category whose objects are
sets that satisfy certain properties. In other words, the objects are sets that have
additional structure imposed upon them. We will also assign functions that
preserve this structure to be the morphisms of our category. In this case,
let's think about partially ordered sets, or \emph{posets}.

We must first establish some definitions:
\begin{itemize}
        \item A poset is a set, $A$, with a relation, $\sim_A$, that is
                reflexive, transitive, and antisymmetric. We covered these properties in
                the Autumn Term.
        \item Let $A$ and $B$ be posets. A monotone map, $m \colon A \to B$, is an
                order-preserving function such that
                \[
                        \forall a, b \in A. \quad a \sim_A b
                        \implies m(a) \sim_B m(b).
                \]
\end{itemize}
We have a category $\catpos$ whose objects are posets and morphisms are monotone
maps between those posets. Verifying that the category axioms are met is, again, not
that difficult:
\begin{itemize}
        \item Let $A, B, C \in \catobj ( \catpos )$ be distinct. Also, let
                $f \in \hom (A, B)$ and $g \in \hom (B, C)$. Since monotone maps
                are functions, we can compose them to get  $g \circ f \colon A
                \to C$. Is this new function also a monotone map? Well, for
                all $a_1, a_2 \in A$, we have that
                \begin{align*}
                        \quad a_1 \sim_A a_2 &\implies f(a_1) \sim_B f(a_2) \\
                        &\implies g(f(a_1)) \sim_C g(f(a_2)).
                \end{align*}
                Thus, $g \circ f$ is indeed a monotone map, and the composition
                axiom is satisfied.
        \item The identity morphism for any object, $A$, is just the monotone
                map \\ $1_A \colon A \to A ; \, a \mapsto a$.
        \item The associativity axiom is satisfied since our morphisms are
                functions, and function composition is associative.
\end{itemize}

\subsection{The category of vector spaces}
We can look to Linear Algebra to provide another simple infinite category. The
category of finite-dimensional $\mathbb{R}$\emph{-vector spaces} is denoted
$\catfinvect$. Let's establish a couple more definitions:
\begin{itemize}
        \item A finite-dimensional $\mathbb{R}$-vector space is an element of
                $\{ \mathbb{R} ^n \, | \, n \in \mathbb{N} \}$.
        \item A linear map is a function, $f$, between two vector spaces,
                such that vector addition and scalar multiplication is
                preserved. That is,
                \begin{alignat*}{2}
                        f( \vec{u} + \vec{v} ) &= f( \vec{u} ) + f( \vec{v} );
                        \qquad && \text{[\textbf{Additivity}]} \\
                        f( \lambda \vec{u} ) &= \lambda f( \vec{u} ),
                        \qquad && \text{[\textbf{Homogeneity}]}
                \end{alignat*}
                with $\vec{u},\vec{v} \in \mathbb{R} ^n \; \exists n \in \mathbb{N}$
                and $\lambda \in \mathbb{R}$.
\end{itemize}
The objects of $\catfinvect$ are finite-dimensional $\mathbb{R}$-vector spaces
and its morphisms are linear maps between them. We will verify that
$\catfinvect$ satisfies the category axioms: 
\begin{itemize}
        \item Let $\mathbb{R} ^a, \mathbb{R} ^b, \mathbb{R} ^c \in
                \catobj ( \catfinvect )$ be distinct. Additionally, let \\
                $f \in \hom ( \mathbb{R} ^a, \mathbb{R} ^b)$ and
                $g \in \hom ( \mathbb{R} ^b, \mathbb{R} ^c)$.
                We have that $g \circ f \colon \mathbb{R} ^a \to \mathbb{R} ^c$
                is a function. Now, let $\vec{u} , \vec{v} \in \mathbb{R} ^a$
                and $\lambda \in \mathbb{R}$. We have that
                \begin{alignat*}{8}
                        (g \circ f)( \vec{u} + \vec{v} )
                        &= g(f( \vec{u} + \vec{v} )) \\
                        &= g(f( \vec{u} ) + f( \vec{v} )) \\
                        &= g(f( \vec{u} )) + g(f( \vec{v} )) \\
                        &= (g \circ f)( \vec{u} ) + (g \circ f)( \vec{v} );
                        \qquad && \text{[\textbf{Additivity}]} \\
                        (g \circ f)( \lambda \vec{u} )
                        &= g(f( \lambda \vec{u} )) \\
                        &= g( \lambda f( \vec{u} )) \\
                        &= \lambda g(f( \vec{u} )) \\
                        &= \lambda (g \circ f)( \vec{u} ),
                        \qquad && \text{[\textbf{Homogeneity}]}
                \end{alignat*}
                so the axiom of composition is met.
        \item The identity morphism for any object, $\mathbb{R} ^n$, is the linear map
                $1_{ \mathbb{R} ^n } \colon \mathbb{R} ^n \to \mathbb{R} ^n ; \,
                \vec{u} \mapsto \vec{u}$.
        \item As before, the associativity axiom is satisfied due to linear maps
                being functions.
\end{itemize}

\subsection{The category of deduction and deducability}
As mentioned at the start of this section, the final exemplary category that we
will cover does not have functions for morphisms and sets for objects. It is
akin to the categories we will see later in this report in that regard.
Remember, objects and morphisms can be anything such that the category axioms
are met -- and these are not very restrictive axioms.
In this case, we are going to explore a category whose objects are theorems and
morphisms are proofs.


\section{\for{toc}{\texorpdfstring{\textbf{[OD]~}}{}}Further Applications: %
        Functional Programming and \texorpdfstring{$\lambda$}{Lambda}-Calculus}
\begin{flushright}
        \textbf{[Written by Oliver Dixon]}
\end{flushright}

\noindent Temporarily moved to \texttt{owd\_report.tex}.
\end{document}


