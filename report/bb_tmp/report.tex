% OWD 2023
%
% BEN'S REPORT

\documentclass[10pt,a4paper,reqno]{amsart}
\usepackage{xcolor}
\usepackage[foot]{amsaddr} % Move authorship information to the front page
\usepackage{caption, subcaption, float, etoolbox} % Enhanced float environments
\usepackage{minted, realboxes, listings} % Source code (Haskell) formatting

\usepackage[%
	backend = biber,
	sorting = none,
	bibencoding = utf8,
	style = alphabetic
]{biblatex}

\usepackage[%
	colorlinks = true,
	allcolors = blue,
        linktoc = page
]{hyperref}

% OWD 2023

% DO NOT COMPILE THIS FILE IN ISOLATION! It is designed to be included by other
% TeX files, and will not produce anything on its own.
%
% Whenever you need a macro involving categories, or load a package for the
% typesetting thereof, add it here, since it's likely that most macros will be
% mutually required between the report and presentation.

\usepackage{amsmath}
\usepackage{tikz-cd} % Drawing commutative diagrams with TikZ.
\usepackage{adjustbox} % Scaling tikz-cd commutative diagrams.

\tikzcdset{outer sep=2pt}

% Style note: please typeset named categories (e.g. Hask) in boldface, as shown
% below. For arbitrary categories (e.g. C or C_0), use \arbcat{...}.
\DeclareMathOperator{\catobj}{ob}
\DeclareMathOperator{\catid}{id}
\DeclareMathOperator{\cathom}{hom}

\newcommand{\arbcat}[1]{\ensuremath\mathcal{#1}}
\DeclareMathOperator{\catendo}{\mathbf{Endo}}
\DeclareMathOperator{\cathask}{\mathbf{Hask}}
\DeclareMathOperator{\catset}{\mathbf{Set}}
\DeclareMathOperator{\catpos}{\mathbf{Pos}}
\DeclareMathOperator{\catvect}{\mathit{K}-\mathbf{Vect}}
\DeclareMathOperator{\catpre}{\mathbf{Ord}}
\DeclareMathOperator{\catpoint}{\mathbf{Based}}
\DeclareMathOperator{\catdeduce}{\mathbf{Deduce}}
\DeclareMathOperator{\catfun}{\mathbf{Fun}}

 % Common category-theoretic macros

% -- Begin listing options for Haskell

\newcommand{\inlinehask}[1]{\mintinline{haskell}{#1}}
\newcommand{\barehask}[1]{\Colorbox{codebg}{\lstinline[style = inlineHask]{#1}}}
\newcommand{\haskmath}[2]{%
        \ensuremath\text{\Colorbox{codebg}{%
                \vphantom{\lstinline[basicstyle = \haskMorphStyle]{fg}}%
                \lstinline[style = inlineHask, basicstyle = #2]{#1}%
        }}%
}
\AtBeginEnvironment{listing}{\setlength{\abovecaptionskip}{.8em}}
\newcommand{\listingautorefname}{Listing}

\renewcommand{\ttdefault}{pcr} % Use Courier instead of CM for fixed-width text.
\newcommand{\haskMorphStyle}{\normalsize\ttfamily\color{haskkeyword}}
\newcommand{\haskObjStyle}{\normalsize\ttfamily\color{black}}
\newcommand{\haskFuncStyle}{\normalsize\ttfamily\color{haskkeyword}\bfseries}

\definecolor{codebg}{RGB}{220,220,220}
\definecolor{haskkeyword}{RGB}{173,46,252}

% I am using the Listings package for printing inline Haskell, and Minted for
% printing verbatim blocks. In general, Minted is far superior, but Listings is
% more crudely customisable for pseudo-code inline blocks that require very
% literal pattern-matching for given tokens.

\setminted{%
        frame = lines,
        framesep = 2mm,
        baselinestretch = 1.2,
        bgcolor = codebg,
        fontsize = \footnotesize,
        linenos
}

\lstdefinestyle{inlineHask}{%
        language = Haskell,
        alsoletter = {->, ::},
        morekeywords = {->, ::, fmap},
        keywordstyle = \color{haskkeyword}\bfseries,
        basicstyle = \footnotesize\ttfamily
}

% -- End listing options for Haskell

% \yorkemail: takes the username and outputs a clickable mailto link.
\newcommand{\yorkemail}[1]{\href{mailto:#1@york.ac.uk}{#1@york.ac.uk}}

% Make bullet points slightly smaller (they look overwhelming in a 10pt font).
\renewcommand{\labelitemi}{$\vcenter{\hbox{\tiny$\bullet$}}$}

% Add appropriately padded frames around numbered figures and sub-figures.
\numberwithin{figure}{section}
\AtBeginEnvironment{subfigure}{\vspace{2em}}
\floatstyle{boxed}
\restylefloat*{figure}

\setcounter{tocdepth}{1} % Only show major sections in the Table of Contents
\addbibresource{report.bib} % Use a universal sources file for BibLaTeX

% -- Begin document metadata setup

\title[An Investigation of Elementary Category Theory]{An Investigation of %
        Elementary Category Theory, with applications in Pure Mathematics and %
        Theoretical Computer Science}

\author{Matthew Drury} \email{\yorkemail{md1499}}
\author{Ben Brook}     \email{\yorkemail{bb1170}}
\author{Oliver Dixon}  \email{\yorkemail{od641}}

\address{Department of Mathematics \\
        University of York \\
        United Kingdom}
\date{Spring--Summer Term, 2023}

% -- End document metadata setup
\begin{document}
\begin{abstract}
        TODO
\end{abstract}
\maketitle
\tableofcontents
\section{\for{toc}{\texorpdfstring{\textbf{[MD]~}}{}}Theoretical %
        Underpinnings: Axiomatic Constructions}
\begin{flushright}
        \textbf{[Written by Matthew Drury]}
\end{flushright}

\noindent Temporarily moved to \texttt{md\_tmp/report.tex}.

\section{\for{toc}{\texorpdfstring{\textbf{[BB]~}}{}}Category-Theoretic %
        Interpretations of Familiar Structures}
\begin{flushright}
        \textbf{[Written by Ben Brook]}
\end{flushright}

\subsection{Introductory examples}
Before proceeding to explore applications of category in functional programming,
it will be helpful to cover categories of structures that most mathematicians
are familiar with. In this section, we will have objects being everything from
sets to proofs.

\subsection{The category of sets}
An easy to digest category is the category of sets, whose objects are sets and
morphisms are functions between sets. These are the set-theoretic functions we
are familiar with. We will follow the common denotation [REF] for this category:
$\catset$.

It is not hard to verify that this concept indeed meets the category axioms seen
in \autoref{sec:axioms}. Let $A,B,C,D \in \mathrm{ob} ( \catset )$ be distinct,
$f \in \mathrm{hom} (A, B)$, $g \in \mathrm{hom} (B, C)$, and $h \in
\mathrm{hom} (C, D)$. Immediately, we have that
\begin{equation*}
        ( f \circ g ) \circ h = f \circ ( g \circ h ),
\end{equation*}
due to the normal associativity property of function composition. We also have
the obvious identity morphism for some object $A$, $1_A \in \mathrm{ob}
( \catset )$. Hence, with the rest being true by definition, the category axioms
are satisfied.

\subsection{The category of partially ordered sets}
Moving forwards, we will consider a category of sets with additional structure:
in this case, a partially ordered set, or \emph{poset}. We must first establish
some definitions:
\begin{itemize}
        \item A poset is a set, $A$, with a relation, $\sim_A$, that is
        reflexive, transitive, and antisymmetric. We covered these properties in
        the Autumn Term.
        \item Let $A$ and $B$ be sets. A monotone map, $m \colon A \to B$, is an
        order-preserving function such that
        \begin{equation*}
                ( \forall a, b \in A. \  a \sim_A b )
                \implies ( m(a) \sim_B m(b) ).
        \end{equation*}
\end{itemize}
We have a category $\catpos$ whose objects are posets and morphisms are monotone
maps betwen those posets. The identity morphism exists for all objects due to
the reflexivity of $\sim_A$, and composition follows similarly due to
transitivity.

\subsection{The category of vector spaces}
Another category that is easy to understand

\section{\for{toc}{\texorpdfstring{\textbf{[OD]~}}{}}Further Applications: %
        Functional Programming and \texorpdfstring{$\lambda$}{Lambda}-Calculus}
\begin{flushright}
        \textbf{[Written by Oliver Dixon]}
\end{flushright}

\noindent Temporarily moved to \texttt{owd\_report.tex}.
\end{document}

