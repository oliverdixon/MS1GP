% OWD 2023
%
% MATTHEW'S REPORT
% REMOVED MINTED AND LISTINGS

\documentclass[10pt,a4paper,reqno]{amsart}
\usepackage{xcolor}
\usepackage[foot]{amsaddr} % Move authorship information to the front page
\usepackage{caption, subcaption, float, etoolbox,amsmath} % Enhanced float environments

\usepackage[%
	backend = biber,
	sorting = none,
	bibencoding = utf8,
	style = alphabetic
]{biblatex}

\usepackage[%
	colorlinks = true,
	allcolors = blue,
        linktoc = page
]{hyperref}

% OWD 2023

% DO NOT COMPILE THIS FILE IN ISOLATION! It is designed to be included by other
% TeX files, and will not produce anything on its own (other than a swathe of
% meaningless temporary files and an incomprehensible error message).
%
% Whenever you need a macro involving categories, or load a package for the
% typesetting thereof, add it here, since it's likely that most macros will be
% mutually required between the report and presentation.

\usepackage{amsmath}
\usepackage{tikz-cd} % Drawing commutative diagrams with TikZ.
\usepackage{adjustbox} % Scaling tikz-cd commutative diagrams.

\tikzcdset{outer sep=2pt} % Apply some padding to arrows in comm. diagrams.

% Provide a thicker arrow style (1pt) for functor morphisms in pseudo-CDs.
\tikzset{FUNCTOR/.style = {line width=#1},%
         FUNCTOR/.default=1pt}

% Style note: please typeset named categories (e.g. Hask) in boldface, as shown
% below. For arbitrary categories (e.g. C or C_0), use \arbcat. For the homset,
% use the built-in \hom macro (\cathom is superfluous).

\newcommand{\arbcat}[1]{\ensuremath\mathcal{#1}}
\newcommand{\arbfunc}[1]{\ensuremath #1} % I might change this to \mathbf{...}

\DeclareMathOperator{\catobj}{ob}
\DeclareMathOperator{\catid}{id}
\DeclareMathOperator{\catcat}{\mathbf{Cat}}
\DeclareMathOperator{\catendo}{\mathbf{Endo}}
\DeclareMathOperator{\cathask}{\mathbf{Hask}}
\DeclareMathOperator{\catset}{\mathbf{Set}}
\DeclareMathOperator{\catpos}{\mathbf{Pos}}
\DeclareMathOperator{\catkvect}{\mathit{K}-\mathbf{Vect}}
\DeclareMathOperator{\catfinvect}{\mathbf{FinVect}_\mathbb{R}}
\DeclareMathOperator{\catpre}{\mathbf{Ord}}
\DeclareMathOperator{\catpoint}{\mathbf{Based}}
\DeclareMathOperator{\catdeduce}{\mathbf{Deduce}}
\DeclareMathOperator{\catfun}{\mathbf{Fun}}

 % Common category-theoretic macros

% \yorkemail: takes the username and outputs a clickable mailto link.
\newcommand{\yorkemail}[1]{\href{mailto:#1@york.ac.uk}{#1@york.ac.uk}}

% Make bullet points slightly smaller (they look overwhelming in a 10pt font).
\renewcommand{\labelitemi}{$\vcenter{\hbox{\tiny$\bullet$}}$}

% Add appropriately padded frames around numbered figures and sub-figures.
\numberwithin{figure}{section}
\AtBeginEnvironment{subfigure}{\vspace{2em}}
\floatstyle{boxed}
\restylefloat*{figure}

\setcounter{tocdepth}{1} % Only show major sections in the Table of Contents
\addbibresource{report.bib} % Use a universal sources file for BibLaTeX

% -- Begin document metadata setup

\title[An Investigation of Elementary Category Theory]{An Investigation of %
        Elementary Category Theory, with applications in Pure Mathematics and %
        Theoretical Computer Science}

\author{Matthew Drury} \email{\yorkemail{md1499}}
\author{Ben Brook}     \email{\yorkemail{bb1170}}
\author{Oliver Dixon}  \email{\yorkemail{od641}}

\address{Department of Mathematics \\
        University of York \\
        United Kingdom}
\date{Spring--Summer Term, 2023}

% -- End document metadata setup
\begin{document}
\begin{abstract}
        TODO
\end{abstract}
\maketitle
\tableofcontents
\section{\for{toc}{\texorpdfstring{\textbf{[MD]~}}{}}Theoretical %
        Underpinnings: Axiomatic Constructions}
\begin{flushright}
        \textbf{[Written by Matthew Drury]}
\end{flushright}


\subsection{The idea of a Category}
A category is a mathematical structure that links a collection of objects with
non-symmetric relationships called morphisms. Each morphism can be said to
traverse from one object to another. They are used to create abstract models of
mathematical theories based on the role each object plays. Categories have the
versatility to talk about different areas of mathematics in the same language,
which is why it is sometimes said to be the mathematics of mathematics. The
objects and morphisms have little restriction as to what they can represent
which can be seen in the commutative diagrams below:
\begin{enumerate}
\item \textbf{Family relationships}\begin{equation}
        \begin{tikzcd}[sep=large]
                A \ar[r,"\text{has father}"] & B \ar[d,"\text{has brother}"] \\
                D & C \ar[l,"\text{has husband}"]
        \end{tikzcd}
\end{equation}
\item \textbf{Factors}\begin{equation}
        \begin{tikzcd}
                2 \ar[r] \ar[dr] & 6 \ar[dr]\\
                3 \ar[ur] \ar[dr] & 10 \ar[r] & 30\\
                5 \ar[ur] \ar[r] & 15 \ar[ur]
        \end{tikzcd}
\end{equation}
\item \textbf{Train routes from York to Sheffield}\begin{equation}
        \begin{tikzcd}
                \text{York} \ar[d] \ar[dr] \\
                \text{Leeds} \ar[d] \ar[dr] \ar[r] & \text{Hull} \ar[d] \\
                \text{Manchester} \ar[r] & \text{Sheffield}
        \end{tikzcd}
\end{equation}
\end{enumerate}
These diagrams are usually used to give a sample of a category to demonstrate
it's properties. Mathematicians like to generalize with categories and consider
infinite objects and morphisms, meaning we want categories to link all of a type
of object we can define. The most valuable mathematical facts tend to be those
that apply to the most situations possible. For instance, uncovering the
quadratic formula being more important than solving a particularly difficult
quadratic. It makes all non-trivial quadratics easier to solve and uncovers a
method to determine whether the solutions are real or complex using the
``$b^2-4ac$'' term. By doing this we have found a relationship between all
quadratics and their solutions. From there we could generalize even further
and think about how this is significant for higher order polynomials.

Morphisms are combined together in a process called composition. It means to do
one morphism and then another, given that the target object of the first
morphism is the same as the source of the second morphism. Above we can see
that person $A$ has father person $B$, who has brother person $C$.  We can
compose these two together to have a morphism from $A$ to $C$ which we could
call ``uncle''. These composite morphisms are implied to exist by the diagram
and such are usually redundant to show. This can be done as many times as one
wants, so if there is a path to traverse from any object to another, there is a
composite morphism between the two. Hence there is a morphism from $A$ to $D$
which we could call ``uncle in-law''.

There are also identity morphisms. These go from an object to itself and are
equivalent to not doing a morphism at all, as when they are composed with
another morphism $f$, the resulting composite morphism is equal to $f$. The
morphism has done nothing, like how multiplying a number by 1(The multiplicative
identity) does not change it's value. It is similar to stating that something
is equal to itself which is a trivial fact but an important feature of the
structure of categories. It should be noted that it may not be the only
morphism from one object to itself. These are called endomorphisms and the
identity is just one of them.

The point of categories is to get more subtle notions of similarity or
``sameness''. By creating categories we abstract things to their roles and by
comparing categories or spotting patterns and structures within them, we can see
properties shared by things that would initially seem very different. Lines of
thought in one context can be more easily translated into another by seeing
equivalent structural features. Mathematicians can then try new methods to
solve problems and prove new facts or describe multiple problems as one that is
more general.
\subsection{Axioms and notation}
We will now formally define a category. The following axioms are necessary for a
structure to be considered a category.
\begin{enumerate}
        \item \textbf{Objects:} In a category $\arbcat{C}$ there is
        $\catobj(\arbcat{C})$, which is a collection of all the objects of $C$.

        \item \textbf{Morphisms:} For each pair ordered pair of objects $(A,B)$
        in a category $\arbcat{C}$ we have $\hom(A,B)$, also written as
        $\arbcat{C}(A,B)$.  This is the collection of all morphisms from A to B,
        short for ``homomorphism set''. Homomorphisms are those that preserve
        the structure of objects and for many useful categories they are
        necessary for the axioms to hold. However, this property does not need
        to hold to use this notation.  A morphism $f$ is notated to map $A$ to
        $B$ by $f: A\to B$.

        \item \textbf{Composition:} For objects $A,B,C\in \catobj(\arbcat{C})$
        if there exists morphisms $(f: A\to B)\in \hom(A,B)$ and $(g: B\to C)\in
        \hom(B,C)$, then there exists $(g\circ f: A\to C)\in \hom(A,C)$
        \begin{equation}
        \begin{tikzcd}
                A \ar[dr,"g\circ f",bend left=20, dashrightarrow] \ar[d,"f"] \\
                B \ar[r,"g"] & C
        \end{tikzcd}
        \end{equation}
        \item \textbf{Identities:} Every object $A$ in a category $\arbcat{C}$
        has an identity morphism which maps from $A$ to itself.  It is notated
        $1_A: A\to A$ for $A\in ob(\arbcat{C}), 1_A\in \hom(A,A)$.  For $f\in
        \hom(A,B), g\in \hom(B,A)$, the identity has the property that $f\circ
        1_A = f, 1_A\circ g = g$. Each identity is unique to it's object. If
        there were two identities $I_1,I_2\in A$ then $(f\circ I_1 = f = f\circ
        I_2) \implies I_1=I_2$

        \begin{equation}
        \begin{tikzcd}
                A \ar[loop right,"1_A"]
        \end{tikzcd}
        \end{equation}

        \item \textbf{Associativity:} A morphism $h\circ(g\circ f) = (h\circ
        g)\circ f$.  This means that two morphisms are equal if they are
        composed from the same morphisms in the same order regardless of the
        order in which we compute the individual composite pairs.

        \begin{equation}
        \begin{tikzcd}
                A \ar[r,"f"] \ar[rr,"g\circ f", bend right=40] \ar[rrr,"h\circ(g\circ f)",bend right=60] \ar[rrr, bend left=60,"(h\circ g)\circ f"]
                & B \ar[r,"g"] \ar[rr,"h\circ g",bend left=40]
                & C \ar[r,"h"] 
                & D
        \end{tikzcd}
        \end{equation}
\end{enumerate}

\subsection{Size of categories}
The collection $\catobj(\arbcat{C})$ and the collections $\hom(A,B)$ do not have
to be finite. The idea of category size means to place categories in a
hierarchy of containment. This is to accommodate how $\catobj(\arbcat{C})$ and
$\hom(A,B)$ do not have to be sets either.

When defining an infinite collection of objects by their properties, it can
create contradictions. Famously in set theory there is Russel's paradox. It
states that if you have a set S that contains every set which doesn't contain
itself, then suppose that S does not contain S, it would imply that S is in fact
in S as it does not contain itself and vice versa. Formally, $S = \{x\;\text{is
a set}\:|\:x\notin x\}, (S\in S) \iff (S\notin S)$.

There were many examples of these paradoxes that were uncovered in the early
20th century. Mathematicians are too rigorous to allow such a thing, so to solve
the issue they devised axiomatic systems to limit the properties that members of
sets can be said to follow. This makes some sets very convoluted to define and
means there are well defined collections of objects that cannot be put into a
set. Categories address this issue in a more elegant way. Firstly, categories
do not claim to contain elements; the core concept is instead relationships.
Secondly, categories have a hierarchy of containment such that they only contain
categories ``smaller'' than themselves. It follows then that categories are not
defined in a self referential manner like in Russel's paradox.

A small category is where $\catobj(\arbcat{C})$ and each $\hom(A,B)$ can be
described as a set. A locally small category is where $\catobj(\arbcat{C})$
does not form a set but each $\hom(A,B)$ does. A large category is where nether
$\catobj(\arbcat{C})$ nor $\hom(A,B)$ are a set. The hierarchy means that if you
wanted a category where small categories are objects, it would have to be a
large category. Then, if you wanted a category of large categories, you would
need a super-large category and so on. This can be used to formally think about
multiple layers of generalization and abstraction.

We can easily avoid Russel's paradox and create categories of categories. 
This is a sign that categories are a robust idea, as we can think about 
categories using the same language as with other mathematical constructs.
\subsection{Functors}

A functor maps between the objects of two categories, similarly to how a function maps between two sets.
Functors preserve the structure of the category, so we want equivalent morphisms between the equivalent objects that have been chosen.
They are homomorphisms in a category of categories.
For categories $\arbcat{C}$, $\arbcat{D}$, if we have a functor $F: \arbcat{C}\to\arbcat{D}$ between them:
\begin{enumerate}
        \item \textbf{Objects:} For every $A\in\catobj(\arbcat{C})$, we have $F_A\in\catobj(\arbcat{D})$
        So $A$ is mapped to $F_A$

        \item \textbf{Morphisms:} For every $f\in\hom_{\arbcat{C}}(A,B)$, we have $F_f\in\hom_{\arbcat{D}}(F_A,F_B)$
        So $f$ is mapped to $F_f$

        \item \textbf{Identities:} For every $A\in\catobj(\arbcat{C})$, we have $1_A$. For every $1_A$, there is $F_{1_A}\in\hom_{\arbcat{D}}(F_A,F_A)$.
        So the identity of $A$ is mapped to the identity of $F_A$

        \item \textbf{Composites:} For every $f\in\hom_{\arbcat{C}}(A,B)$, $g\in\hom_{\arbcat{C}}(B,C)$, we have that $F(g\circ f) = F_g\circ F_f$.
        This means that it is doesn't matter the order in which compositions and functors are applied.
        Hence, composing has the same effect on the morphisms in each category.

        \begin{equation}
        \begin{tikzcd}[column sep=huge]
                A \ar[r,"F",dashrightarrow] \ar[d,"f"] & F_A \ar[d,"F_f"]\\
                B \ar[r,"F",dashrightarrow] \ar[d,"g"] & F_B \ar[d,"F_g"]\\
                C \ar[r,"F",dashrightarrow] & F_C
        \end{tikzcd}
        \end{equation}

        The commutative diagram above shows how we can traverse $g\circ f$ and then apply $F$,
        or we can apply $F$, then traverse $F_g\circ F_f$.
\end{enumerate}

\section{\for{toc}{\texorpdfstring{\textbf{[BB]~}}{}}Category-Theoretic %
        Interpretations of Familiar Structures}
\begin{flushright}
        \textbf{[Written by Ben Brook]}
\end{flushright}

\noindent Temporarily moved to \texttt{bb\_tmp/report.tex}.

\section{\for{toc}{\texorpdfstring{\textbf{[OD]~}}{}}Further Applications: %
        Functional Programming and \texorpdfstring{$\lambda$}{Lambda}-Calculus}
\begin{flushright}
        \textbf{[Written by Oliver Dixon]}
\end{flushright}

\noindent Temporarily moved to \texttt{owd\_report.tex}.
\end{document}

