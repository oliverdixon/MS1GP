% OWD 2023

\documentclass{article}
\usepackage[a4paper]{geometry}
\usepackage{fancyhdr,lastpage,xcolor,lineno,titlesec,minted,xstring}
\usepackage[%
	colorlinks = true,
	allcolors = darkgray,
        linktoc = page
]{hyperref}

% Per-document variables.
%
\author{Oliver Dixon \textless\yorkemail{od641}\textgreater}
\newcommand{\originalauthor}{Matthew Drury \textless\yorkemail{md1499}\textgreater}
\newcommand{\commithashlong}{1fdca05a3331dedea459ce802fe15c81230e967b}
\newcommand{\originaltexfile}{matthew.tex}

% Formatting the original text and source thereof. Minted is used for verbatim
% environments with LaTeX syntax highlighting.
%
\renewcommand{\linenumberfont}{\normalfont\small}
\setlength\linenumbersep{2em}
\renewcommand{\theFancyVerbLine}{\linenumberfont\arabic{FancyVerbLine}}

\setminted[latex]{%
        linenos,
        breaklines,
        frame=single,
        xleftmargin=-.8em
}

\newcommand{\printoriginal}{%
        \clearpage
        \section*{Original Copy (Rendered)}
        \begin{linenumbers}
                \input{\originaltexfile}
        \end{linenumbers}
        \section*{Original Copy (Source)}
        \inputminted{latex}{\originaltexfile}
}


% Document metadata set-up.
%
\title{MS1 GP Comments Track}
\DeclareRobustCommand{\commithashshort}{%
        \StrLeft{\commithashlong}{7}[\strtmp]%
        \commitref{\strtmp}
}

% Header and footer configuration.
%
\fancypagestyle{title}{%
        \thispagestyle{fancy}
        \fancyhead[L]{}
}
\makeatletter
\fancypagestyle{fancy}{%
        \fancyhf{}
	\fancyhead[R]{\@author}
        \fancyhead[L]{\@title\ @ \commithashshort}
        \fancyfoot[L]{\today}
        \fancyfoot[R]{Page \thepage{} of~\pageref{LastPage}}
}
\makeatother
\renewcommand{\headrulewidth}{.5pt}
\renewcommand{\footrulewidth}{\headrulewidth}

% Miscellaneous formatting commands.
%
\renewcommand{\labelitemi}{$\vcenter{\hbox{\tiny$\bullet$}}$}
\newcommand{\yorkemail}[1]{\href{mailto:#1@york.ac.uk}{#1@york.ac.uk}}
\newcommand{\commitref}[1]{\href{https://github.com/oliverdixon/MS1GP/%
        commit/\commithashlong}{\texttt{#1}}}
\titleformat*{\section}{\centering \bfseries\LARGE}
\makeatletter
\newcommand{\subtitle}{%
        \vspace{-.3\baselineskip}
        \begin{center}
                Commit Hash: \commitref{\commithashlong}
                \vspace{.3\baselineskip}

                Comments Author: \@author

                Original Text Author: \originalauthor
        \end{center}
        \vspace{2\baselineskip}
        \par\vspace{-2em}\noindent\rule{\textwidth}{\headrulewidth}\vspace{1em}
}
\makeatother

% End of the preamble; start of the document.
%
\begin{document}
\thispagestyle{title}
\pagestyle{fancy}
\makeatletter
\section*{\@title}
\makeatother
\subtitle
\begin{itemize}
        \item Line 1: Typo: ``Category'', not ``cateogory''. Please ensure that
        your editor has access to a working spell-checker.

        \item Line 1: Typo: ``Non-symmetric'', not ``non-symetric''.

        \item Line 2: The verbal form of ``abstract'' sounds strange here. I
        would also suggest incorporating this sentence as a secondary clause
        elsewhere, to avoid an interruption in the reading flow.

        \item Line 3: Typo: ``conceive'', not ``concieve''.

        \item Line 3: Typo: ``fulfil'', not ``fulfill'' (the latter is the US
        spelling).

        \item Lines 3--4: ``Either or neither objects and morphisms...'': this
        is very difficult to read.

        \item Line 6: TeX comments are prefixed with a percent symbol (‘\%'),
        not a double-oblique.

        \item Line 8: Try to keep the terminology consistent. ``Objects'' and
        ``morphisms'' or ``items'' and ``arrows''?

        \item Line 9: When, exactly, can morphisms be composed? I know that you
        don't want to delve into a discussion of domains and codomains just yet,
        but I still think that there's some room for increased precision here.

        \item Line 12: Perhaps you should mention that these are called
        \textit{commutative diagrams}? Why do they have this name?

        \item Line 14: ``Structure'' is probably more precise than
        ``situation''.  It's also the canonical terminology that'll be used
        throughout the rest of the report, so it's best to introduce it now.

        \item Lines 15--17: Could you try and make this sentence more succinct?
        Something like this may sound better: ``Drawing parallels between
        categories, through the use of functors, often reveals unexpected
        insights to their respective structures, thus giving rise to novel
        techniques in proof and computation.''

        \item Line 16: You have used ``situation'' again. Regardless of the
        appropriateness of this term, try to avoid excessive repetition of
        vocabulary.

        \item Lines 17--18: This doesn't make sense. Are you talking about
        operator analogues in higher- dimensional arithmetic?

        \item Lines 20--21: Can you add a bit of punctuation here? It doesn't
        read well.
\end{itemize}
\printoriginal
\end{document}

