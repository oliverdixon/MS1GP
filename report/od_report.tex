% MAJOR REPORT SECTION
%
% AUTHOR: OLIVER DIXON
% TITLE:  FURTHER APPLICATIONS: FUNCTIONAL PROGRAMMING AND LAMBDA-CALCULUS
%
\subsection{Functional Programming and Haskell}

In purely functional languages, there is no allowance for context, or mutable
variables of any kind. Each function must accept some data, perform some strict
transformation upon the data---as defined by the algorithm---and return the
result. Whilst this robust paradigm does open a wide range of mathematical
avenues involving proof, safety, and reproducibility, the prohibition of
stateful computation renders many common tasks, such as I/O or socket
communication, largely impossible, as these imperatively defined operations
inherently contravene the purity principles of functional programming.

Haskell is a commonly used purely functional programming language, and suffers,
as do all languages in the same class, from this blaring issue.  Indeed, early
versions of Haskell did not support the chaining of stateful computation in any
sense, due to the obligatory absence of a fixed execution order in functional
paradigms; programmers were forced to resort to breaking the purity of the
language through aesthetically unpleasant techniques, ultimately obviating the
mathematical essence of the Haskell formal type system.

Due to the strength of the Haskell type system and function interface, we may
define a corresponding category, $\cathask$, within which the objects are
Haskell types, and the morphisms are functions\footnotemark.
%
\footnotetext{Due to the $\lambda$-Calculus concept of \emph{currying}, named
after Haskell Curry, functions taking multiple arguments may be decomposed into
a chain of function compositions, in which each function strictly accepts and
returns a single argument. This is made explicit in Haskell, where the type
signature of a function \inlinehask{f} may be defined as \inlinehask{f :: a -> b
-> c}, invoked as \inlinehask{f a b}, and expected to return a value of type
\barehask{c}. This function signature is trivially equivalent to the
\emph{uncurried} form of \inlinehask{f}, defined as \inlinehask{f :: (a -> b) ->
c}.} %
By analysing the structure of $\cathask$, its endofunctors, and the categories
formed by taking product, we can draw a swathe of parallels between generic
purely functional paradigms; these connexions will eventually reveal the
powerful concept of \emph{monads}, allowing stateful computation, control flow,
and error-handling.

\subsection{Monoidal Categories and Monoids}%
\label{sec:monoidal-categories}

Before fully exploiting the structure of $\cathask$, we must develop the theory
of \emph{monoidal categories} and their corresponding \emph{monoids}. Monoidal
categories can be regarded as a six-tuple $(\arbcat{C}_0, \otimes, I, \alpha,
\lambda, \rho)$, containing various components \autocite{Kelly:1982}:
\begin{itemize}
        \item A base category, $\arbcat{C}_0$;

        \item A bifunctor $\otimes \colon \arbcat{C}_0 \times \arbcat{C}_0 \to
        \arbcat{C}_0$;

        \item An identity object $I \in \catobj \arbcat{C}_0$;

        \item An associativity natural transformation $\alpha_{A,B,C} \colon (A
        \otimes B) \otimes C \to A \otimes (B \otimes C)$;

        \item A left-identity natural transformation $\lambda_A \colon I \otimes
        A \to A$;

        \item A right-identity natural transformation $\rho_A \colon A \otimes I
        \to A$.
\end{itemize}

To maintain brevity, the natural transformations are often omitted from the
tuple-descriptions of monoidal categories: $(\arbcat{C}_0, \otimes, I)$. In this
context, the natural transformations $\alpha$, $\lambda$, and $\rho$ are used to
induce certain properties on $\otimes$---\emph{up to isomorphism}---which it
may not possess natively; the effects of these transformations can be
illustrated as morphisms on a pair of abstracted commutative diagrams (c.f.
\autoref{fig:monoidal-cat-commute}).

Then, monoids can be considered as three-tuples, consisting of an object in a
monoidal category $(\arbcat{C}, \otimes, I)$, coupled with two transformations:
\begin{itemize}
        \item A base object $M \in \catobj \arbcat{C}_0$;
        \item A \emph{multiplication} transformation $\mu \in \hom_\arbcat{C}
        (M \otimes M, M)$;
        \item A \emph{unit} transformation $\eta \in \hom_\arbcat{C} (I, M)$.
\end{itemize}
Once more, the natural transformations from the parent monoidal category
$\arbcat{C}$ can be used to induce properties on $\mu$, as in
\autoref{fig:monoid-commute}.

\begin{figure}[ht]
        \vspace{\belowcaptionskip}
        \begin{subfigure}{\textwidth}
                \centering
                \begin{tikzcd}[ampersand replacement=\&]
                        \left[(A \otimes B) \otimes C \right] \otimes D
                                \arrow[r, "\alpha"]
                                \arrow[d, "\alpha\,\otimes\,\catid D" swap] \&
                        (A \otimes B) \otimes (C \otimes D)
                                \arrow[r, "\alpha"] \&
                        A \otimes \left[ B \otimes (C \otimes D) \right] \\
                        \left[ A \otimes (B \otimes C) \right] \otimes D
                                \arrow[rr, "\alpha" swap] \& \&
                        A \otimes \left[ (B \otimes C) \otimes D \right]
                                \arrow[u, "\catid A\,\otimes\,\alpha" swap]
                \end{tikzcd}%
                \caption{Associativity on $\otimes$ induced by $\alpha$}
        \end{subfigure}
        \vspace{-\belowcaptionskip}
        \begin{subfigure}{\textwidth}
                \centering
                \begin{tikzcd}[sep=large, ampersand replacement=\&]
                        (B \otimes I) \otimes C
                                \arrow[rr, "\alpha"]
                                \arrow[rd, "\rho\,\otimes\,\catid C" swap] \& \&
                        B \otimes (I \otimes C)
                                \arrow[ld, "\catid B\,\otimes\,\lambda"] \\
                        \& B \otimes C \&
                \end{tikzcd}%
                \caption{Left- and right-identities of objects $A,B,C,D$ induced
                        by $\lambda$ and $\rho$}
        \end{subfigure}%
        \vspace{-.5em}
        \caption{\textbf{(Monoidal Categories)} The natural transformations
                invoke commutativity on the bifunctor $\otimes$.}
        \label{fig:monoidal-cat-commute}
\end{figure}
\begin{figure}[ht]
        \vspace{\belowcaptionskip}
        \begin{subfigure}{\textwidth}
                \centering
                \begin{tikzcd}[sep=large, ampersand replacement=\&]
                        (M \otimes M) \otimes M
                                \arrow[r, "\alpha"]
                                \arrow[d, "\mu\,\otimes\,\catid M" swap] \&
                        M \otimes (M \otimes M)
                                \arrow[r, "\catid M\,\otimes\,\mu"] \&
                        M \otimes M
                                \arrow[d, "\mu"] \\
                        M \otimes M
                                \arrow[rr, "\mu" swap] \& \& M
                \end{tikzcd}%
                \caption{Associativity on $\mu$ induced by $\alpha$}
        \end{subfigure}
        \vspace{-1.5em}
        \begin{subfigure}{\textwidth}
                \centering
                \begin{tikzcd}[sep=large, ampersand replacement=\&]
                        I \otimes M
                                \arrow[r, "\eta\,\otimes\,\catid M"]
                                \arrow[rd, "\lambda" swap] \&
                        M \otimes M
                                \arrow[d, "\mu" swap] \&
                        M \otimes I
                                \arrow[l, "\catid M\,\otimes\,\eta" swap]
                                \arrow[ld, "\rho"] \\
                        \& M \&
                \end{tikzcd}%
                \vspace{-.5em}
                \caption{Left- and right-identities of $M$ induced by $\lambda$
                        and $\rho$}
        \end{subfigure}%
        \caption{\textbf{(Monoids)} The natural transformations from the parent
                monoidal category also apply within the monoid.}
        \label{fig:monoid-commute}
\end{figure}

It is important that the domain of the $\mu$ transformation is a \emph{product
combination} of the monoidal object; this will allow the imperative-like
threading of state in functional paradigms.

\subsubsection{Examples of Monoidal Categories: Set Theory}

The canonical example of a monoidal category whose bifunctor/tensor product is
not associative is $\catset$, with the cross product; this is not
\emph{naturally associative}, but can be made associative up to isomorphism with
a suitable choice of the natural transformation $\alpha$. The details are
established in \autocite{Fong:2018}, using an associativity transform as
described in \autoref{eqn:assoc-transform-example}, providing a monoidal
category of the form $(\catset, \times, I)$, where $I$ represents some fixed
singleton.
\begin{equation}
        \alpha_{A,B,C} \colon (A \times B) \times C \to A \times (B \times C)%
        \label{eqn:assoc-transform-example}
\end{equation}

\subsubsection{Examples of Monoidal Categories: Haskell}

In Haskell, the simplest practical implementation of a monoidal category is
outlined in \autoref{lst:hask-tuple}, where the base category is $\cathask$, the
identity is the empty tuple, and the bifunctor is the tuple-building native
function.
\begin{listing}[ht]
        \inputminted{haskell}{haskell/Cross.hs}%
        \caption{A binary Haskell function \protect\inlinehask{cross} that
                encodes its arguments into a tuple. In category-theoretic
                language, the corresponding monoidal category could be expressed
                as the three-tuple
                $(\cathask, \text{\;\protect\barehask{cross}},
                \text{\;\protect\barehask{()}})$.}
        \label{lst:hask-tuple}
\end{listing}

\subsection{The Category of Endofunctors}

For the purposes of Functional Programming, and the wider formal treatment of
functional type systems, a particularly useful monoidal category concerns the
\emph{category of endofunctors} over some fixed base category $\arbcat{C}_0$.
Denoted as $\catendo{(\arbcat{C}_0)}$, this forms a monoidal category with the
associated bifunctor being the standard operation of endofunctor composition;
the identity element is the obligatory identity endofunctor. Objects in
$\catendo{(\arbcat{C}_0)}$ are the endofunctors over $\arbcat{C}_0$, and
morphisms are the natural transformations between these objects.  Monoids in the
category of endofunctors are sometimes called monads. Thus, endofunctors over
$\arbcat{C}_0$ with appropriately selected morphisms $\mu$ and $\eta$ are
henceforth termed as monads \autocite{MacLane:1998}.

\subsubsection{Monads in Haskell} An abbreivated definition of a Haskell
\inlinehask{Monad} is given in \autoref{lst:haskell-monad-definition}; here we
see the semi-curried form of the \emph{bind operator}. The importance of
\inlinehask{>>=} is reflected by its inclusion in the Haskell logo!
\vspace{-.5em} % Can I get away with this? :)
\begin{listing}
        \inputminted{haskell}{haskell/Monad.hs}%
        \caption{The Haskell \protect\inlinehask{>>=} and
                \protect\inlinehask{return} functions allow programmers to
                interact with the \protect\inlinehask{Monad} class in the
                categorical sense.}
        \label{lst:haskell-monad-definition}
\end{listing}

\begin{figure}[ht]
        \vspace{\belowcaptionskip}
        \begin{tikzcd}[outer sep=5pt, ampersand replacement=\&]
                \haskmath{a}{\haskObjStyle}
                        \arrow[r, "\haskmath{f}{\haskMorphStyle}" swap]
                        \arrow[bend left=40, rr, "\haskmath{f.g}%
                                {\haskMorphStyle}"] \&
                \haskmath{b}{\haskObjStyle}
                        \arrow[r, "\haskmath{g}{\haskMorphStyle}" swap] \&
                \haskmath{c}{\haskObjStyle}
        \end{tikzcd}
        \begin{tikzcd}[ampersand replacement=\&]
                {} \arrow[rr, FUNCTOR, "\haskmath{[]}{\haskFuncStyle}"] \& \& {}
        \end{tikzcd}
        \begin{tikzcd}[sep=large, outer sep=5pt, ampersand replacement=\&]
                \haskmath{[a]}{\haskObjStyle}
                        \arrow[r, "\haskmath{map f}{\haskMorphStyle}" swap]
                        \arrow[bend left=40, rr,
                                "\haskmath{map \$ f.g}{\haskMorphStyle}"] \&
                \haskmath{[b]}{\haskObjStyle}
                        \arrow[r, "\haskmath{map g}{\haskMorphStyle}" swap] \&
                \haskmath{[c]}{\haskObjStyle}
        \end{tikzcd}%
        \caption{The action of the \inlinehask{[]} functor on $\cathask$. The
                list constructor functor \emph{lifts} the Haskell types
                \protect\barehask{a}, \protect\barehask{b}, and
                \protect\barehask{c} into the list context.}
        \label{fig:functor-list-map}
\end{figure}

\begin{figure}[ht]
        \vspace{\belowcaptionskip}
        \begin{tikzcd}[outer sep=5pt, ampersand replacement=\&]
                \haskmath{a}{\haskObjStyle}
                        \arrow[r, "\haskmath{f}{\haskMorphStyle}" swap]
                        \arrow[bend left=40, rr, "\haskmath{f.g}%
                                {\haskMorphStyle}"] \&
                \haskmath{b}{\haskObjStyle}
                        \arrow[r, "\haskmath{g}{\haskMorphStyle}" swap] \&
                \haskmath{c}{\haskObjStyle}
        \end{tikzcd}
        \begin{tikzcd}[ampersand replacement=\&]
                {} \arrow[rr, FUNCTOR, "\haskmath{F}{\haskFuncStyle}"] \& \& {}
        \end{tikzcd}
        \begin{tikzcd}[sep=large, outer sep=5pt, ampersand replacement=\&]
                \haskmath{F a}{\haskObjStyle}
                        \arrow[r, "\haskmath{fmap f}{\haskMorphStyle}" swap]
                        \arrow[bend left=40, rr,
                                "\haskmath{fmap \$ f.g}{\haskMorphStyle}"] \&
                \haskmath{F b}{\haskObjStyle}
                        \arrow[r, "\haskmath{fmap g}{\haskMorphStyle}" swap] \&
                \haskmath{F c}{\haskObjStyle}
        \end{tikzcd}%
        \caption{The action of the generic functor \inlinehask{F} on some
                $\cathask$. The mappings of the objects are implicit in the
                mappings of the morphisms, as domains and codomains must be
                preserved between isomorphic categories.}
        \label{fig:functor-general-fmap}
\end{figure}

\subsection{Functors in Haskell}

This interdisciplinary review of Category Theory and Functional Programming
becomes useful when considering the category of endofunctors over the $\cathask$
category, $\catendo{(\cathask)}$, thus forming a monoidal category with monads
as endofunctors over $\cathask$ \autocite{Milewski:2019}. In Haskell, these are
simply typed as \inlinehask{Functor}, defined as a typeclass providing
appropriate mappings from $\cathask$ to $\cathask$ for types and functions, as
shown in \autoref{lst:hask-functor}. The Haskell function \inlinehask{fmap} is
used to lift a function \barehask{a -> b}, embedded in $\cathask$ as the domain
category, to the functorial context \barehask{f a -> f b}, embedded in
$\cathask$ as the codomain category.

Haskell Functors can be solidified with the most trivial example: the \emph{list
constructor}, which takes types $A,B,C \in \catobj \cathask$ and lifts them into
the list structure with the \inlinehask{fmap} endofunctor\footnotemark. This
process is illustrated in case of lists in \autoref{fig:functor-list-map}, and
in the general case in \autoref{fig:functor-general-fmap}.
%
\footnotetext{In the Haskell \inlinehask{[]} instantiation of
\inlinehask{Functor}, the \inlinehask{fmap} field is set to the stricter
\inlinehask{map} function; this is an unimportant implementation detail in this
case.}

\subsection{Controlling State with Haskell Monads}

To address our original problem of stateless computation, how might a
\emph{specific} usage of \inlinehask{Monad} allow the threading of state through
pseudo-imperative function calls? By recalling the domain of the
\emph{multiplication} transformation on the monoid, as defined in
\autoref{sec:monoidal-categories}, we have an immensely useful functor
transformation $\catendo{(\cathask)}$, such that $\mu \colon M \times M \to M$,
where $M$ is an object in $\arbcat{C}_0 \coloneqq \cathask$. This can be
implemented as a Haskell function which allows the merging of two $\cathask$
instances into a single \emph{combined instance}.
%
By exploiting the lazy evaluation of Haskell, such that functions are only
executed when directly invoked, programmers can enforce an execution order by
chaining evaluations of some transformation $\mu$. Context is achieved by
\emph{applying} a $\cathask$, as the standard function parameter, to a given
context; the function must then return the transformed context according to the
prescribed algorithm or process.

\begin{listing}
        \inputminted{haskell}{haskell/Functor.hs}%
        \caption{The Haskell \protect\inlinehask{Functor} type signature, of
                which the list type constructor \inlinehask{[]} is an instance.}
        \label{lst:hask-functor}
\end{listing}
%
% OWD 2023 :)
%
