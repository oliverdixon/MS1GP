% OWD 2023

\documentclass[10pt,a4paper,reqno]{amsart}
\usepackage{xcolor}
\usepackage[foot]{amsaddr} % Move authorship information to the front page
\usepackage{caption, subcaption, float, etoolbox} % Enhanced float environments
\usepackage{minted, realboxes, listings} % Source code (Haskell) formatting

\usepackage[%
	backend = biber,
	sorting = none,
	bibencoding = utf8,
	style = alphabetic
]{biblatex}

\usepackage[%
	colorlinks = true,
	allcolors = blue,
        linktoc = page
]{hyperref}

% OWD 2023

% DO NOT COMPILE THIS FILE IN ISOLATION! It is designed to be included by other
% TeX files, and will not produce anything on its own (other than a swathe of
% meaningless temporary files and an incomprehensible error message).
%
% Whenever you need a macro involving categories, or load a package for the
% typesetting thereof, add it here, since it's likely that most macros will be
% mutually required between the report and presentation.

\usepackage{amsmath}
\usepackage{tikz-cd} % Drawing commutative diagrams with TikZ.
\usepackage{adjustbox} % Scaling tikz-cd commutative diagrams.

\tikzcdset{outer sep=2pt} % Apply some padding to arrows in comm. diagrams.

% Provide a thicker arrow style (1pt) for functor morphisms in pseudo-CDs.
\tikzset{FUNCTOR/.style = {line width=#1},%
         FUNCTOR/.default=1pt}

% Style note: please typeset named categories (e.g. Hask) in boldface, as shown
% below. For arbitrary categories (e.g. C or C_0), use \arbcat. For the homset,
% use the built-in \hom macro (\cathom is superfluous).

\newcommand{\arbcat}[1]{\ensuremath\mathcal{#1}}
\newcommand{\arbfunc}[1]{\ensuremath #1} % I might change this to \mathbf{...}

\DeclareMathOperator{\catobj}{ob}
\DeclareMathOperator{\catid}{id}
\DeclareMathOperator{\catcat}{\mathbf{Cat}}
\DeclareMathOperator{\catendo}{\mathbf{Endo}}
\DeclareMathOperator{\cathask}{\mathbf{Hask}}
\DeclareMathOperator{\catset}{\mathbf{Set}}
\DeclareMathOperator{\catpos}{\mathbf{Pos}}
\DeclareMathOperator{\catkvect}{\mathit{K}-\mathbf{Vect}}
\DeclareMathOperator{\catfinvect}{\mathbf{FinVect}_\mathbb{R}}
\DeclareMathOperator{\catpre}{\mathbf{Ord}}
\DeclareMathOperator{\catpoint}{\mathbf{Based}}
\DeclareMathOperator{\catdeduce}{\mathbf{Deduce}}
\DeclareMathOperator{\catfun}{\mathbf{Fun}}

 % Common category-theoretic macros

% -- Begin listing options for Haskell

\newcommand{\inlinehask}[1]{\mintinline{haskell}{#1}}
\newcommand{\barehask}[1]{\Colorbox{codebg}{\lstinline[style = inlineHask]{#1}}}
\newcommand{\haskmath}[2]{%
        \ensuremath\text{\Colorbox{codebg}{%
                \vphantom{\lstinline[basicstyle = \haskMorphStyle]{fg}}%
                \lstinline[style = inlineHask, basicstyle = #2]{#1}%
        }}%
}
\AtBeginEnvironment{listing}{\setlength{\abovecaptionskip}{.8em}}
\newcommand{\listingautorefname}{Listing}

\renewcommand{\ttdefault}{pcr} % Use Courier instead of CM for fixed-width text.
\newcommand{\haskMorphStyle}{\normalsize\ttfamily\color{haskkeyword}}
\newcommand{\haskObjStyle}{\normalsize\ttfamily\color{black}}
\newcommand{\haskFuncStyle}{\normalsize\ttfamily\color{haskkeyword}\bfseries}

\definecolor{codebg}{RGB}{220,220,220}
\definecolor{haskkeyword}{RGB}{173,46,252}

% I am using the Listings package for printing inline Haskell, and Minted for
% printing verbatim blocks. In general, Minted is far superior, but Listings is
% more crudely customisable for pseudo-code inline blocks that require very
% literal pattern-matching for given tokens.

\setminted{%
        frame = lines,
        framesep = 2mm,
        baselinestretch = 1.2,
        bgcolor = codebg,
        fontsize = \footnotesize,
        linenos
}

\lstdefinestyle{inlineHask}{%
        language = Haskell,
        alsoletter = {->, ::},
        morekeywords = {->, ::, fmap},
        keywordstyle = \color{haskkeyword}\bfseries,
        basicstyle = \footnotesize\ttfamily
}

% -- End listing options for Haskell

% \yorkemail: takes the username and outputs a clickable mailto link.
\newcommand{\yorkemail}[1]{\href{mailto:#1@york.ac.uk}{#1@york.ac.uk}}

% Make bullet points slightly smaller (they look overwhelming in a 10pt font).
\renewcommand{\labelitemi}{$\vcenter{\hbox{\tiny$\bullet$}}$}

% Add appropriately padded frames around numbered figures and sub-figures.
\numberwithin{figure}{section}
\AtBeginEnvironment{subfigure}{\vspace{2em}}
\floatstyle{boxed}
\restylefloat*{figure}

\setcounter{tocdepth}{1} % Only show major sections in the Table of Contents
\addbibresource{report.bib} % Use a universal sources file for BibLaTeX

% -- Begin document metadata setup

\title[An Investigation of Elementary Category Theory]{An Investigation of %
        Elementary Category Theory, with applications in Pure Mathematics and %
        Theoretical Computer Science}

\author{Matthew Drury} \email{\yorkemail{md1499}}
\author{Ben Brook}     \email{\yorkemail{bb1170}}
\author{Oliver Dixon}  \email{\yorkemail{od641}}

\address{Department of Mathematics \\
        University of York \\
        United Kingdom}
\date{Spring--Summer Term, 2023}

% -- End document metadata setup

\begin{document}
\begin{abstract}
        TODO
\end{abstract}
\maketitle
\tableofcontents
\section{\for{toc}{\texorpdfstring{\textbf{[MD]~}}{}}Theoretical %
        Underpinnings: Axiomatic Constructions}
\begin{flushright}
        \textbf{[Written by Matthew Drury]}
\end{flushright}
\noindent \textit{Section omitted.}

\section{\for{toc}{\texorpdfstring{\textbf{[BB]~}}{}}Category-Theoretic %
        Interpretations of Familiar Structures}
\begin{flushright}
        \textbf{[Written by Ben Brook]}
\end{flushright}
\noindent \textit{Section omitted.}

\section{\for{toc}{\texorpdfstring{\textbf{[OD]~}}{}}Further Applications: %
        Functional Programming and \texorpdfstring{$\lambda$}{Lambda}-Calculus}
\begin{flushright}
        \textbf{[Written by Oliver Dixon]}
\end{flushright}

\subsection{Functional Programming and Haskell}

In purely functional languages, there is no allowance for context, or mutable
variables of any kind. Each function must accept some data, perform some strict
transformation upon the data---as defined by the algorithm---and return the
result. Whilst this robust paradigm does open a wide range of mathematical
avenues involving proof, safety, and reproducibility, the prohibition of
stateful computation renders many common tasks, such as system I/O or
communication over a network socket, largely impossible, as these imperatively
defined operations inherently contravene the purity principles of functional
programming.

Haskell is a commonly used purely functional programming language, and
suffers, as do all languages in the same class, from this blaring issue.
Indeed, early versions of Haskell did not support the chaining of stateful
computation in any sense, due to the obligatory absence of a fixed execution
order in functional paradigms; programmers were forced to resort to breaking the
purity of the language through aesthetically unpleasant techniques, ultimately
obviating the mathematical essence of the Haskell formal type system.

Due to the strength of the Haskell type system and function interface, we may
define a corresponding category, $\cathask$, within which the objects are
Haskell types, and the morphisms are functions\footnotemark.
%
\footnotetext{Due to the $\lambda$-Calculus concept of \emph{currying}, named
after Haskell Curry, functions taking multiple arguments may be decomposed into
a chain of function compositions, in which each function strictly accepts and
returns a single argument. This is made explicit in Haskell, where the type
signature of a function \inlinehask{f} may be defined as \inlinehask{f :: a -> b
-> c}, invoked as \inlinehask{f a b}, and expected to return a value of type
\barehask{c}. This function signature is trivially equivalent to the
\emph{uncurried} form of \inlinehask{f}, defined as \inlinehask{f :: (a -> b) ->
c}.}%

\subsection{Monoidal Categories and Monoids}

Before fully exploiting the structure of $\cathask$, we must develop the theory
of \emph{monoidal categories} and their corresponding \emph{monoids}. Monoidal
categories can be regarded as a six-tuple $(\arbcat{C}_0, \otimes, I, \alpha,
\lambda, \rho)$, containing various components \autocite{Kelly:1982}:
\begin{itemize}
        \item A base category, $\arbcat{C}_0$;

        \item A bifunctor $\otimes \colon \arbcat{C}_0 \times \arbcat{C}_0 \to
        \arbcat{C}_0$;

        \item An identity object $I \in \catobj \arbcat{C}_0$;

        \item An associativity natural transformation $\alpha_{A,B,C} \colon (A
        \otimes B) \otimes C \to A \otimes (B \otimes C)$;

        \item A left-identity natural transformation $\lambda_A \colon I \otimes
        A \to A$;

        \item A right-identity natural transformation $\rho_A \colon A \otimes I
        \to A$.
\end{itemize}

To maintain brevity, the natural transformations are often omitted from the
tuple-descriptions of monoidal categories: $(\arbcat{C}_0, \otimes, I)$. In this
context, the natural transformations $\alpha$, $\lambda$, and $\rho$ are used to
induce certain properties on $\otimes$, which it may not possess natively; this
can be illustrated with a pair of commutative diagrams as in
\autoref{fig:monoidal-cat-commute}\footnotemark.
Then, monoids can be considered as three-tuples, consisting of an object in a
monoidal category $(\arbcat{C}, \otimes, I)$, coupled with two \emph{lifting}
natural transformations $\mu$ and $\eta$:
\begin{itemize}
        \item A base object $M \in \catobj \arbcat{C}$;
        \item A \emph{multiplication} transformation $\mu \in \hom_\arbcat{C}
        (M \otimes M, M)$;
        \item A \emph{unit} transformation $\eta \in \hom_\arbcat{C} (I, M)$.
\end{itemize}
Once more, the natural transformations from the parent monoidal category
$\arbcat{C}$ can be used to induce properties on $\mu$, as in
\autoref{fig:monoid-commute}.
%
\footnotetext{The canonical example of a monoidal category whose
bifunctor/tensor product is not associative is $\catset$, with the cross
product; this is not \emph{naturally associative}, but can be made associative
up to isomorphism with a suitable choice of the natural transformation $\alpha$.
The details are established in \autocite{Fong:2018}.}

\begin{figure}
        \begin{subfigure}{\textwidth}
                \centering
                \begin{tikzcd}[ampersand replacement=\&]
                        \left[(A \otimes B) \otimes C \right] \otimes D
                                \arrow[r, "\alpha"]
                                \arrow[d, "\alpha\,\otimes\,\catid D" swap] \&
                        (A \otimes B) \otimes (C \otimes D)
                                \arrow[r, "\alpha"] \&
                        A \otimes \left[ B \otimes (C \otimes D) \right] \\
                        \left[ A \otimes (B \otimes C) \right] \otimes D
                                \arrow[rr, "\alpha" swap] \& \&
                        A \otimes \left[ (B \otimes C) \otimes D \right]
                                \arrow[u, "\catid A\,\otimes\,\alpha" swap]
                \end{tikzcd}%
                \caption{Associativity induced by $\alpha$}
        \end{subfigure}

        \begin{subfigure}{\textwidth}
                \centering
                \begin{tikzcd}[sep=large, ampersand replacement=\&]
                        (B \otimes I) \otimes C
                                \arrow[rr, "\alpha"]
                                \arrow[rd, "\rho\,\otimes\,\catid C" swap] \& \&
                        B \otimes (I \otimes C)
                                \arrow[ld, "\catid B\,\otimes\,\lambda"] \\
                        \& B \otimes C \&
                \end{tikzcd}%
                \caption{Left- and right-identities induced by $\lambda$ and
                        $\rho$}
        \end{subfigure}%
        \caption{The natural transformations invoke commutativity on the
        bifunctor $\otimes$ applied to objects $A$, $B$, $C$, $D$, and $I$.}
        \vspace{1.5em}
        \label{fig:monoidal-cat-commute}
\end{figure}

\begin{figure}
        \begin{subfigure}{\textwidth}
                \centering
                \begin{tikzcd}[sep=large, ampersand replacement=\&]
                        (M \otimes M) \otimes M
                                \arrow[r, "\alpha"]
                                \arrow[d, "\mu\,\otimes\,\catid M" swap] \&
                        M \otimes (M \otimes M)
                                \arrow[r, "\catid M\,\otimes\,\mu"] \&
                        M \otimes M
                                \arrow[d, "\mu"] \\
                        M \otimes M
                                \arrow[rr, "\mu" swap] \& \& M
                \end{tikzcd}%
                \caption{Associativity on $\mu$ induced by $\alpha$}
        \end{subfigure}

        \begin{subfigure}{\textwidth}
                \centering
                \begin{tikzcd}[sep=large, ampersand replacement=\&]
                        I \otimes M
                                \arrow[r, "\eta\,\otimes\,\catid M"]
                                \arrow[rd, "\lambda" swap] \&
                        M \otimes M
                                \arrow[d, "\mu" swap] \&
                        M \otimes I
                                \arrow[l, "\catid M\,\otimes\,\eta" swap]
                                \arrow[ld, "\rho"] \\
                        \& M \&
                \end{tikzcd}%
                \caption{Left- and right-identities of $M$ induced by $\lambda$
                        and $\rho$}
        \end{subfigure}%
        \caption{The natural transformations also apply within the monoid.}
        \vspace{1.5em}
        \label{fig:monoid-commute}
\end{figure}

\subsection{The Category of Endofunctors}

For the purposes of Functional Programming, and the wider formal treatment of
functional type systems, a particularly useful monoidal category concerns the
\emph{category of endofunctors} over some fixed base category $\arbcat{C}_0$.
Denoted as $\catendo{(\arbcat{C}_0)}$, this forms a monoidal category with the
associated bifunctor being the standard operation of endofunctor composition;
the identity element is the obligatory identity endofunctor. Objects in
$\catendo{(\arbcat{C}_0)}$ are the endofunctors over $\arbcat{C}_0$, and
morphisms are the natural transformations between these objects.  Monoids in the
category of endofunctors are sometimes called \emph{monads}. Thus, endofunctors
over $\arbcat{C}_0$ are henceforth termed as monads \autocite{MacLane:1998}.

\subsection{Functors in Haskell}

This interdisciplinary review of Category Theory and Functional Programming
becomes useful when considering the category of endofunctors over the $\cathask$
category, $\catendo{(\cathask)}$, thus forming a monoidal category with monads
as endofunctors over $\cathask$ \autocite{Milewski:2019}. In Haskell, these are
simply typed as \inlinehask{Functor}, defined as a typeclass providing
appropriate mappings from $\cathask$ to $\cathask$ for types and functions, as
shown in \autoref{lst:hask-functor}. The Haskell function \inlinehask{fmap} is
used to lift a function \barehask{a -> b}, embedded in $\cathask$ as the domain
category, to the functorial context \barehask{f a -> f b}, embedded in
$\cathask$ as the codomain category.

\begin{listing}
        \inputminted{haskell}{haskell/Functor.hs}%
        \caption{The Haskell \texttt{Functor} type signature, of which the list
                type constructor \inlinehask{[]} is an instance.}
        \label{lst:hask-functor}
\end{listing}

Haskell Functors can be solidified with the most trivial example: the \emph{list
constructor}, which takes types $A,B,C \in \catobj \cathask$ and lifts them into
the list structure with the \inlinehask{fmap} endofunctor\footnotemark. This
process is illustrated in case of lists in \autoref{fig:functor-list-map}, and
in the general case in \autoref{fig:functor-general-fmap}.
%
\footnotetext{In the Haskell \inlinehask{[]} instantiation of
\inlinehask{Functor}, the \inlinehask{fmap} field is set to the stricter
\inlinehask{map} function; this is an unimportant implementation detail in this
case.}

\begin{figure}
        \centering
        \vspace{1.5em}
        \begin{tikzcd}[outer sep=5pt, ampersand replacement=\&]
                \haskmath{A}{\haskObjStyle}
                        \arrow[r, "\haskmath{f}{\haskMorphStyle}" swap]
                        \arrow[bend left=40, rr, "\haskmath{f.g}%
                                {\haskMorphStyle}"] \&
                \haskmath{B}{\haskObjStyle}
                        \arrow[r, "\haskmath{g}{\haskMorphStyle}" swap] \&
                \haskmath{C}{\haskObjStyle}
        \end{tikzcd}
        \begin{tikzcd}[ampersand replacement=\&]
                {} \arrow[rr, FUNCTOR, "\haskmath{[]}{\haskFuncStyle}"] \& \& {}
        \end{tikzcd}
        \begin{tikzcd}[sep=large, outer sep=5pt, ampersand replacement=\&]
                \haskmath{[A]}{\haskObjStyle}
                        \arrow[r, "\haskmath{map f}{\haskMorphStyle}" swap]
                        \arrow[bend left=40, rr,
                                "\haskmath{map \$ f.g}{\haskMorphStyle}"] \&
                \haskmath{[B]}{\haskObjStyle}
                        \arrow[r, "\haskmath{map g}{\haskMorphStyle}" swap] \&
                \haskmath{[C]}{\haskObjStyle}
        \end{tikzcd}%
        \caption{The action of the \inlinehask{[]} functor on $\cathask$}
        \vspace{1.5em}
        \label{fig:functor-list-map}
\end{figure}

\begin{figure}
        \centering
        \vspace{1.5em}
        \begin{tikzcd}[outer sep=5pt, ampersand replacement=\&]
                \haskmath{A}{\haskObjStyle}
                        \arrow[r, "\haskmath{f}{\haskMorphStyle}" swap]
                        \arrow[bend left=40, rr, "\haskmath{f.g}%
                                {\haskMorphStyle}"] \&
                \haskmath{B}{\haskObjStyle}
                        \arrow[r, "\haskmath{g}{\haskMorphStyle}" swap] \&
                \haskmath{C}{\haskObjStyle}
        \end{tikzcd}
        \begin{tikzcd}[ampersand replacement=\&]
                {} \arrow[rr, FUNCTOR, "\haskmath{F}{\haskFuncStyle}"] \& \& {}
        \end{tikzcd}
        \begin{tikzcd}[sep=large, outer sep=5pt, ampersand replacement=\&]
                \haskmath{F A}{\haskObjStyle}
                        \arrow[r, "\haskmath{fmap f}{\haskMorphStyle}" swap]
                        \arrow[bend left=40, rr,
                                "\haskmath{fmap \$ f.g}{\haskMorphStyle}"] \&
                \haskmath{F B}{\haskObjStyle}
                        \arrow[r, "\haskmath{fmap g}{\haskMorphStyle}" swap] \&
                \haskmath{F C}{\haskObjStyle}
        \end{tikzcd}%
        \caption{The action of the generic functor \inlinehask{F} on some
                $\cathask$}
        \vspace{1.5em}
        \label{fig:functor-general-fmap}
\end{figure}

\addtocontents{toc}{\protect\vspace{5pt}}
\printbibliography[title=Cited Works]
\end{document}

