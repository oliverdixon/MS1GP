% MAJOR REPORT SECTION
%
% AUTHOR: MATTHEW DRURY
% TITLE:  THEORETICAL UNDERPINNINGS: AXIOMATIC CONSTRUCTIONS
%
\subsection{The Idea of a Category}

A category is a mathematical structure that links a collection of objects with
non-symmetric relationships called morphisms. Each morphism can be said to
traverse from one object to another. They are used to create abstract models of
mathematical theories based on the role each object plays. Categories have the
versatility to talk about different areas of mathematics in the same language,
which is why it is sometimes said to be the mathematics of mathematics. The
objects and morphisms have little restriction as to what they can represent
which can be seen in the commutative diagrams illustrated in
\autoref{fig:initial-cd} (\autoref{fig:factor-cd} is adapted
from~\autocite{Cheng:2022}).
\begin{figure}[ht]
        \vspace{\belowcaptionskip}
        \begin{subfigure}{.45\textwidth}
                \centering
                \begin{tikzcd}
                        2 \ar[r] \ar[dr] & 6 \ar[dr]\\
                        3 \ar[ur] \ar[dr] & 10 \ar[r] & 30\\
                        5 \ar[ur] \ar[r] & 15 \ar[ur]
                \end{tikzcd}%
                \vspace{1em}%
                \caption{Factors}%
                \label{fig:factor-cd}
        \end{subfigure}
        \vspace{-\belowcaptionskip}%
        \begin{subfigure}{.45\textwidth}
                \centering
                \begin{tikzcd}
                        \text{York} \ar[d] \ar[dr] \\
                        \text{Leeds} \ar[d] \ar[dr] \ar[r] & \text{Hull} \ar[d] \\
                        \text{Manchester} \ar[r] & \text{Sheffield}
                \end{tikzcd}%
                \vspace{1em}%
                \caption{Train routes from York to Sheffield}
        \end{subfigure}%
        \caption{Real-World Examples of Categories}%
        \label{fig:initial-cd}
\end{figure}

These diagrams are usually used to give a sample of a category to demonstrate
these properties. Mathematicians like to generalise with categories and consider
infinite objects and morphisms, meaning we want categories to link all types of
objects that we can define. The most valuable mathematical facts tend to be
those that apply to the widest range of situations. For instance, uncovering the
quadratic formula being more important than solving a particularly difficult
quadratic. It makes all non-trivial quadratics easier to solve and uncovers a
method to determine whether the solutions are real or complex using the
discriminant term ``$b^2-4ac$''. By doing this we have found a relationship
between all quadratics and their solutions. From there we could generalise even
further and think about how this is significant for higher order polynomials. 

Morphisms are combined together in a process called composition. It means to do
one morphism and then another, given that the target object of the first
morphism is the same as the source of the second morphism. Above we can see that
we can go from York to Leeds to Manchester. This means we have a morphism that
goes from York to Manchester. There are four different ways to get to Sheffield,
meaning four morphisms. These may not be the same morphisms, as they only share
in source and target, not necessarily meaning. When two different composite
morphisms are equal, we say they commute, hence the name \emph{commutative
diagram}.

There are also identity morphisms. These go from an object to itself and are
equivalent to not doing a morphism at all, as when they are composed with
another morphism $f$, the resulting composite morphism is equal to $f$. The
morphism has done nothing, like how multiplying a number by 1(The multiplicative
identity) does not change its value. It is similar to stating that something is
equal to itself which is a trivial fact but an important feature of the
structure of categories. It should be noted that it may not be the only morphism
from one object to itself. These are called endomorphisms and the identity is
just one of them.

Categories illuminate subtle similarities. By creating categories we abstract
things to their roles and by comparing categories or spotting patterns and
structures within them, we can see properties shared by things that would
initially seem very different. Lines of thought in one context can be more
easily translated into another by seeing equivalent structural features.
Mathematicians can then try new methods to solve problems and prove new facts or
describe multiple problems as one that is more general.

\subsection{Axioms and Notation}

We will now formally define a category. The following axioms are necessary for a
structure to be considered a category.
\begin{enumerate}
        \item \textbf{Objects:} In a category $\arbcat{C}$ there is
        $\catobj(\arbcat{C})$, which is a collection of all the objects of
        $\arbcat{C}$.

        \item \textbf{Morphisms:} For each pair ordered pair of objects $(A,B)$
        in a category $\arbcat{C}$ we have $\hom_\arbcat{C}(A,B)$, also written
        as $\arbcat{C}(A,B)$.  This is the collection of all morphisms from A to
        B, short for ``homomorphism set''. Homomorphisms are those that preserve
        the structure of objects and for many useful categories they are
        necessary for the axioms to hold. However, this property does not need
        to hold to use this notation. A morphism $f$ is notated to map $A$ to
        $B$ by $f\colon A\to B$.

        \item \textbf{Composition:} For objects $A,B,C\in \catobj(\arbcat{C})$
        if there exists morphisms $(f\colon A\to B)\in \hom_\arbcat{C}(A,B)$ and
        $(g \colon B\to C)\in \hom_\arbcat{C}(B,C)$, then there exists $(g\circ
        f\colon A\to C)\in \hom_\arbcat{C}(A,C)$

        \item \textbf{Identities:} Every object $A$ in a category $\arbcat{C}$
        has an identity morphism which maps from $A$ to itself.  It is notated
        $1_A\colon A\to A$ for $A\in ob(\arbcat{C}), 1_A\in
        \hom_\arbcat{C}(A,A)$.  For $f\in \hom_\arbcat{C}(A,B), g\in
        \hom_\arbcat{C}(B,A)$, the identity has the property that $f\circ 1_A =
        f, 1_A\circ g = g$. Each identity is unique to its object. If there were
        two identities $I_1,I_2\in A$ then $(f\circ I_1 = f = f\circ I_2)
        \implies I_1=I_2$

        \item \textbf{Associativity:} A morphism $h\circ(g\circ f) = (h\circ
        g)\circ f$. This means that two morphisms are equal if they are composed
        from the same morphisms in the same order regardless of the order in
        which we compute the individual composite pairs.
\end{enumerate}
\begin{figure}
        \vspace{\belowcaptionskip}
        \begin{subfigure}{.3\textwidth}
                \centering
                \vspace{1em}
                \begin{tikzcd}
                        A
                                \ar[dr, "g\circ f", bend left=20,
                                        dashrightarrow]
                                \ar[d, "f"] \\
                        B
                                \ar[r, "g"]
                                \ar[dr, "h\circ g", bend right=20,
                                        dashrightarrow, swap] &
                        C
                                \ar[d, "h"] \\ &
                        D
                \end{tikzcd}%
                \vspace{1em}
                \caption{Composition}
        \end{subfigure}
        \hfill
        \begin{subfigure}{.15\textwidth}
                \centering
                \vspace{1em}
                \begin{tikzcd}
                        A
                                \ar[loop right,"1_A"]
                                \ar[d , "f" swap] \\
                        B
                                \ar[loop right,"1_B"]
                                \ar[d , "g" swap] \\
                        C
                                \ar[loop right,"1_C"]
                \end{tikzcd}%
                \vspace{1em}
                \caption{Identities}
        \end{subfigure}
        \hfill
        \begin{subfigure}{.4\textwidth}
                \centering
                \begin{tikzcd}
                        A
                                \ar[r,"f"]
                                \ar[rr, "g\circ f", bend right=40]
                                \ar[rrr, "h\circ(g\circ f)",%
                                        bend right=60]
                                \ar[rrr, "(h\circ g)\circ f",%
                                        bend left=60,] &
                        B
                                \ar[r, "g"]
                                \ar[rr, "h\circ g", bend left=40] &
                        C
                                \ar[r,"h"]  &
                        D
                \end{tikzcd}%
                \caption{Associativity}
        \end{subfigure}%
        \vspace{-\belowcaptionskip}
        \caption{Essential Properties of Categorical Structures}
\end{figure}

\subsection{Size of Categories}

The collection $\catobj(\arbcat{C})$ and the collections $\hom_\arbcat{C}(A,B)$
do not have to be finite. The idea of category size means to place categories in
a hierarchy of containment. This is to accommodate how $\catobj(\arbcat{C})$ and
$\hom_\arbcat{C}(A,B)$ do not have to be sets either.

When defining an infinite collection of objects by their properties, it can
create contradictions. Famously in set theory there is Russell's
paradox~\autocite{Russell:1903}. It states that if you have a set S that
contains every set which doesn't contain itself, then suppose that S does not
contain S, it would imply that S is in fact in S as it does not contain itself
and vice versa. Formally, 
\begin{equation}
S = \{x\;\text{is a set}\:|\:x\notin x\}, (S\in S) \iff (S\notin S)
\end{equation}
There were many examples of these paradoxes that were uncovered in the early
20th century. Mathematicians are too rigorous to allow such a thing, so to solve
the issue they devised axiomatic systems to limit the properties that members of
sets can be said to follow. This makes some sets very convoluted to define and
means there are well defined collections of objects that cannot be put into a
set. Categories address this issue in a more elegant way. Firstly, categories do
not claim to contain elements; the core concept is instead relationships.
Secondly, categories have a hierarchy of containment such that they only contain
categories ``smaller'' than themselves. It follows then that categories are not
defined in a self referential manner like in Russell's paradox.

A small category is where $\catobj(\arbcat{C})$ and each $\hom_\arbcat{C}(A,B)$
can be described as a set. A locally small category is where
$\catobj(\arbcat{C})$ does not form a set but each $\hom_\arbcat{C}(A,B)$ does.
A large category is where nether $\catobj(\arbcat{C})$ nor
$\hom_\arbcat{C}(A,B)$ are a set. The hierarchy means that if one wanted a
category where small categories are objects, it would have to be a large
category. Then, if you wanted a category of large categories, you would need a
super-large category and so on. This can be used to formally think about
multiple layers of generalization and abstraction.

We can easily avoid Russell's paradox and create categories of categories. This
is a sign that categories are a robust idea, as we can think about a collection
of categories using the same language as within individual categories.

\subsection{Functors}

A functor maps between the objects of two categories, similarly to how a
function maps between two sets.  Functors preserve the structure of the
category, so we want equivalent morphisms between the equivalent objects that
have been chosen. This relationship is illustrated in
\autoref{fig:functor-intro}.
\begin{figure}[ht]
        \vspace{\belowcaptionskip}
        $\underbrace{\begin{tikzcd}[sep=large, ampersand replacement=\&]
                A \arrow[r, "f"] \arrow[rd, "g \circ f" swap]
                \& B \arrow[d, "g"] \\
                \& C
        \end{tikzcd}}_{\substack{\\[.2em]\text{\small Domain Category
                $\arbcat{C}$}}}$%
        \begin{tikzcd}[sep=large, ampersand replacement=\&]
                {} \arrow[rr, FUNCTOR, %
                        "\substack{\text{\large\textit{F}}\\[.2em]}", %
                        yshift=-.8em]
                \& \& {}
        \end{tikzcd}
        $\underbrace{\begin{tikzcd}[sep=large, ampersand replacement=\&]
                \arbfunc{F}(B) \arrow[d, "\arbfunc{F}(g)" swap] \&
                \arbfunc{F}(A) \arrow[l, "\arbfunc{F}(f)" swap]
                        \arrow[dl, "\arbfunc{F}(g \circ f)"]
                \\ \arbfunc{F}(C) \&
        \end{tikzcd}}_{\substack{\\[.2em]\text{\small Codomain Category
                $\arbcat{D}$}}}$%
        \caption{A functor $\arbfunc{F}$ between simple categories
                $\arbcat{C}$ and $\arbcat{D}$}
        \label{fig:functor-intro}
\end{figure}

For categories $\arbcat{C}$, $\arbcat{D}$, if we have a functor $\arbfunc{F}
\colon \arbcat{C}\to\arbcat{D}$ between them:
\begin{enumerate}
        \item \textbf{Objects:} For every $A\in\catobj(\arbcat{C})$, 
        $F_A\in\catobj(\arbcat{D})$. Thus, $A$ is mapped to $F(A)$.

        \item \textbf{Morphisms:} For every $f\in\hom_{\arbcat{C}}(A,B)$, we
        have $F(f)\in\hom_{\arbcat{D}}(F_A,F_B)$. Thus, $f$ is mapped to $F(f)$.

        \item \textbf{Identities:} For every $A\in\catobj{\arbcat{C}}$, we have
        $1_A$. For every $1_A$, there is corresponding morphism
        $F(1_A)\in\hom_{\arbcat{D}}(F(A),F(A))$. Thus, the identity of $A$ is
        mapped to the identity of $F(A)$.

        \item \textbf{Composites:} For every $f\in\hom_{\arbcat{C}}(A,B)$,
        $g\in\hom_{\arbcat{C}}(B,C)$, we have that $F{(g\circ f)} = F_g\circ
        F_f$. This means that the order in which compositions and functors are
        applied is unimportant. This implies that composition in each category
        has an equivalent effect on its morphisms.
\end{enumerate}
The commutative diagram above shows how from $A$ we can traverse $g\circ f
\colon A\to C$ and then apply $F$, or we can apply $F$, then traverse $F(g \circ
f) \colon F(A)\to F(C)$.

In general, a functor is a way of finding the structure of one category in
another.  For instance, a category with two objects and one morphism between
them $f \colon A\to B$ can have a functor to any category by selecting any
morphism in it, which could even be the identity of an object.  This principle
extends to categories of infinite objects.

\subsection{Non-Small Categories}

The large category $\catcat$ has objects which are all small categories and
homomorphisms which are all functors between them.  Each functor $F \colon
\arbcat{C}\to\arbcat{D}$ can be seen as a function between sets
$\catobj(\arbcat{C})$ and $\catobj({\arbcat{D}})$.  It could be injective,
meaning that each object of $\arbcat{C}$ maps to a different object of
$\arbcat{D}$. Otherwise, the functor is showing that one object in $\arbcat{D}$
has the same categorical properties as a structure of objects in $\arbcat{C}$.

If it is bijective, then the functor is an isomorphism, a morphism that has an
inverse which composes to form the identity. This implies that $\arbcat{C}$ and
$\arbcat{D}$ can reach the same categories with functors and be reached by the
same categories with functors. In general, isomorphisms are a way of saying that
objects indistinguishable from the perspective of the category they are in.
However, it is not a suitable notion of equivalence between categories.

The identities in $\catcat$ are identity functors which map a category to a copy
of itself with the exact same objects, morphisms, compositions. More generally
it is one of the endofunctors which map categories to themselves.

Composition is simply doing one functor after another. For functors
$F\colon\arbcat{C}\to\arbcat{D}$, $G \colon \arbcat{D}\to\arbcat{E}$, we can
describe a single functor $G\circ F\colon\arbcat{C}\to\arbcat{E}$.  When a lot
of functors have been composed together we can get an increasingly subtle
structural similarity between categories that we would not be able to notice by
laying out diagrams next to each other and seeing a pattern.
%
