% MAJOR REPORT SECTION
%
% AUTHOR: BEN BROOK
% TITLE:  CATEGORY-THEORETIC INTERPRETATIONS OF FAMILIAR STRUCTURES
%
\setlength{\jot}{5pt}
\subsection{Introductory Examples}

Now that we have established an axiomatic definition of the category, the
natural next step involves the exploration of some simple applications of
categories in a familiar context.

For the first three of these examples, our objects will be
nothing but sets and morphisms nothing but functions (both with some additional
restrictions). Categories of this type are sometimes called
\emph{concrete} \autocite{Awodey:2010}.
However, it's important to remember that categories need not
contain functions nor sets. In fact, we will see this in our final introductory
example, where we do away with both entirely!

All the exemplary categories that we are going to touch on will be infinite.
This means that we won't be using any diagrams for this section of the report.

\subsection{Category of Sets}

To begin, an easily digestible example is the category of sets, whose objects
are sets, and morphisms are total functions between sets. These are the
set-theoretic functions with which we are familiar and have been using
throughout the Autumn Term.  We will follow a common denotation for this
category: $\catset$~\autocite{Leinster:2014}. We can easily verify that this
category satisfies the category axioms seen in \autoref{sec:cat-axioms}.
\begin{itemize}
        \item There are objects, which are sets.
        \item There are morphisms between those objects, which are functions.
                Remember, however: morphisms are not necessarily functions! In
                fact, there are an abundance of examples where this is not the
                case.
        \item The composition of two functions is also a function, with its
                domain and codomain coming from the functions involved in the
                composition operation. Hence, the axiom of composition is
                satisfied.
        \item Let $A \in \catobj{(\catset)}$. We have that the identity
                morphism for $A$ is $(1_A \colon A \to A) \in \hom (A, A)$,
                which maps each element of $A$ to itself.
        \item The morphisms of $\catset$ are associative due to the
                native associativity of function composition. Let $A, B, C, D
                \in \catobj{(\catset)}$ be distinct. Also, let $f \in \hom (A,
                B)$, $g \in \hom (B, C)$, and $h \in \hom (C, D)$. Thus,
                \begin{equation}
                        ( f \circ g ) \circ h = f \circ ( g \circ h ).
                \end{equation}
\end{itemize}

Hence, all the category axioms are satisfied. Going forwards, we will avoid the
pedantry of listing that our category contains morphisms and objects.

\subsection{Category of Partially Ordered Sets}

A nice path to advance down is the exploration of a category whose objects are
sets that satisfy certain properties. In other words, the objects are sets that
have additional structure imposed upon them. We will also assign functions that
preserve this structure to be the morphisms of our category. In this case, let's
think about partially ordered sets, or \emph{posets}. We must first establish
some definitions:
\begin{itemize}
        \item A poset is a set, $A$, with a relation, $\sim_A$, that is
                reflexive, transitive, and antisymmetric. We covered these
                properties in the Autumn Term.
        \item Let $A$ and $B$ be posets. A monotone map $m \colon A \to B$ is
                an order-preserving function such that
                \begin{equation}
                        \forall\, a, b \in A \qquad a \sim_A b
                        \implies m(a) \sim_B m(b).
                \end{equation}
\end{itemize}
We have a category $\catpos$ whose objects are posets and morphisms are monotone
maps between those posets. Once more, we can verify the satisfaction of the
relevant axioms.
\begin{itemize}
        \item Let $A, B, C \in \catobj{(\catpos)}$ be distinct. Also, let
                $f \in \hom (A, B)$ and $g \in \hom (B, C)$. Since monotone maps
                are functions, we can compose them to get  $g \circ f \colon A
                \to C$. Is this new function also a monotone map? Well, for
                all $a_1, a_2 \in A$, we have that
                \begin{align}
                        \quad a_1 \sim_A a_2 &\implies f(a_1) \sim_B f(a_2) \\
                        &\implies g(f(a_1)) \sim_C g(f(a_2)).
                \end{align}
                Thus, $g \circ f$ is a monotone map, and the composition axiom
                is satisfied.
        \item The identity morphism for any fixed object, $A$, is the
                monotone map $1_A \colon A \to A$, such that $a \mapsto a$.
        \item The associativity axiom is satisfied since our morphisms are
                functions, and function composition is associative.
\end{itemize}

\subsection{Category of Finite-Dimensional Vector Spaces}

We can look to Linear Algebra to provide another simple infinite category. The
category of finite-dimensional $\mathbb{R}$\emph{-vector spaces} is often denoted
$\catrvect$ \autocite{Hasegawa:2008}.
\begin{itemize}
        \item A finite-dimensional $\mathbb{R}$-vector space is an element of
                $\{ \mathbb{R} ^n \, | \, n \in \mathbb{N} \}$.
        \item A linear map is a function, $f$, between two vector spaces,
                such that vector addition and scalar multiplication is
                preserved. That is,
                \begin{alignat}{2}
                        f( \vec{u} + \vec{v} ) &= f( \vec{u} ) + f( \vec{v} );
                        \qquad && \text{[\textbf{Additivity}]}%
                                \label{eqn:vs-additivity}\\
                        f( \lambda \vec{u} ) &= \lambda f( \vec{u} ),
                        \qquad && \text{[\textbf{Homogeneity}]}%
                                \label{eqn:vs-homo}
                \end{alignat}
                with $\lambda \in \mathbb{R}$ and
                $\vec{u},\vec{v} \in \mathbb{R} ^n$ for some $n \in \mathbb{N}$.
\end{itemize}
The objects of $\catrvect$ are finite-dimensional real-valued vector spaces, and
its morphisms are linear maps between them. We will verify that $\catrvect$
satisfies the category axioms:
\begin{itemize}
        \item Let $\mathbb{R} ^a, \mathbb{R} ^b, \mathbb{R} ^c \in
                \catobj{(\catrvect)}$ be distinct. Additionally, suppose that
                $f \in \hom ( \mathbb{R} ^a, \mathbb{R} ^b)$ and
                $g \in \hom ( \mathbb{R} ^b, \mathbb{R} ^c)$.
                Trivially, $g \circ f \colon \mathbb{R} ^a \to \mathbb{R} ^c$
                is a map. Now, let $\vec{u} , \vec{v} \in \mathbb{R} ^a$
                and $\lambda \in \mathbb{R}$. We have that
                \begingroup
                \renewcommand{\equationautorefname}{Eqn.}
                \begin{alignat}{2}
                        (g \circ f)( \vec{u} + \vec{v} )
                        &= g(f( \vec{u} )) + g(f( \vec{v} ))\nonumber \\
                        &= (g \circ f)( \vec{u} ) + (g \circ f)( \vec{v} )
                        \qquad&&\text{[By \autoref{eqn:vs-additivity}]} \\[.5em]
                        %
                        (g \circ f)( \lambda \vec{u} )
                        &= \lambda g(f( \vec{u} ))\nonumber \\
                        &= \lambda (g \circ f)( \vec{u} ),
                        \qquad&&\text{[By \autoref{eqn:vs-homo}]}
                \end{alignat}
                \endgroup
                so the axiom of composition is met.
        \item The identity morphism for any object is the linear map such that
                $\vec{u} \mapsto \vec{u}$.
        \item The associativity axiom is satisfied due to functional nature of
                the maps.
\end{itemize}

\subsection{Category of Propositions}
As mentioned at the start of this section, the final exemplary category that we
cover is not concrete. In this regard, it is similar to the categories we will
see later in this report. Here we leverage how objects and morphisms can be
anything such that the category axioms are met.

In this case, we are going to explore a category whose objects are propositions
and morphisms are proofs, which we will denote $\catprop$. We will consider this
category informally, by employing our well-established intuition of proofs.
Suppose $P, Q \in \catobj{(\catprop)}$. If there is a proof which, under $P$,
gives $Q$ -- or a proof leading from $P$ to $Q$ -- then this gives a morphism
$(f \colon P \to Q) \in \hom (P, Q)$ (which is not a function!) We will use $P
\vdash Q$ to denote this. We can confirm the categorical nature of this
structure:
\begin{itemize}
        \item Let $P, Q, R \in \catobj{(\catprop)}$ be distinct and let
                $f \in \hom (P, Q), g \in \hom (Q, R)$. Then $P \vdash Q \land
                Q \vdash R$ so, intuitively, $P \vdash R$ and $g \circ f$ belongs
                to the category.
        \item We can also intuitively reason that there is a proof leading from
                any proposition to itself, so each object has an identity
                morphism.
        \item Finally, we know that morphism composition is associative, again
                through our intuition. Let $P, Q, R, S \in \catobj{(\catprop)}$
                be distinct. Let $f \in \hom (P, Q)$, $g \in \hom (Q, R)$, and
                $h \in \hom (R, S)$. That $(h \circ g) \circ f$ is a morphism
                asserts both that there is a proof leading from $Q$ to $S$, and
                that $P$ leads to $Q$. Likewise, that $h \circ (g \circ f)$ is a
                morphism asserts both the existence of a proof leading from $P$
                to $R$, and that $R$ leads to $S$. These two assertions are
                identical, thus morphism composition is associative.
\end{itemize}
We covered this category informally in that some facts stated above do not hold
for \emph{every} conceivable system of logic, but just some particularly
well-behaved constructions\footnote{In the scope of this report, the
distinctions between varying systems of logic is unimportant.}. Our intuition
was hopefully enough to take value from the example. A further, and more
rigorous, exploration of this concept is detailed by \autocite{Baez:2009}.
%
