% OWD 2023
%
% MASTER REPORT FILE
% FINAL COPY, READY FOR SUBMISSION
%
% DIRECT DEPENDENCIES:
%  - ../catmacros.tex   [CT macro library ]
%  - ./report.bib       [Bibliography keys]
%  - ./md_report.tex    [Major section    ]
%  - ./bb_report.tex    [Major section    ]
%  - ./od_report.tex    [Major section    ]

\documentclass[10pt,a4paper,reqno]{amsart}
\usepackage{xcolor, fancyhdr, lastpage, mathtools, enumitem}
\usepackage[en-GB]{datetime2} % Regional formatting of \today
\usepackage[foot]{amsaddr} % Move authorship information to the front page
\usepackage{caption, subcaption, float, etoolbox} % Enhanced float environments
\usepackage{minted, realboxes, listings} % Source code (Haskell) formatting

\usepackage[%
	backend = biber,
	sorting = none,
	bibencoding = utf8,
	style = alphabetic
]{biblatex}

\usepackage[%
	colorlinks = true,
	allcolors = blue,
        %
        % Add interactive hyperlinks to page numbers in the ToC.
        linktoc = page,
        %
        % Build the PDF page index across the entire section hierarchy.
        bookmarksdepth = subsubsection
]{hyperref}

% OWD 2023

% DO NOT COMPILE THIS FILE IN ISOLATION! It is designed to be included by other
% TeX files, and will not produce anything on its own (other than a swathe of
% meaningless temporary files and an incomprehensible error message).
%
% Whenever you need a macro involving categories, or load a package for the
% typesetting thereof, add it here, since it's likely that most macros will be
% mutually required between the report and presentation.

\usepackage{amsmath}
\usepackage{tikz-cd} % Drawing commutative diagrams with TikZ.
\usepackage{adjustbox} % Scaling tikz-cd commutative diagrams.

\tikzcdset{outer sep=2pt} % Apply some padding to arrows in comm. diagrams.

% Provide a thicker arrow style (1pt) for functor morphisms in pseudo-CDs.
\tikzset{FUNCTOR/.style = {line width=#1},%
         FUNCTOR/.default=1pt}

% Style note: please typeset named categories (e.g. Hask) in boldface, as shown
% below. For arbitrary categories (e.g. C or C_0), use \arbcat. For the homset,
% use the built-in \hom macro (\cathom is superfluous).

\newcommand{\arbcat}[1]{\ensuremath\mathcal{#1}}
\newcommand{\arbfunc}[1]{\ensuremath #1} % I might change this to \mathbf{...}

\DeclareMathOperator{\catobj}{ob}
\DeclareMathOperator{\catid}{id}
\DeclareMathOperator{\catcat}{\mathbf{Cat}}
\DeclareMathOperator{\catendo}{\mathbf{Endo}}
\DeclareMathOperator{\cathask}{\mathbf{Hask}}
\DeclareMathOperator{\catset}{\mathbf{Set}}
\DeclareMathOperator{\catpos}{\mathbf{Pos}}
\DeclareMathOperator{\catkvect}{\mathit{K}-\mathbf{Vect}}
\DeclareMathOperator{\catfinvect}{\mathbf{FinVect}_\mathbb{R}}
\DeclareMathOperator{\catpre}{\mathbf{Ord}}
\DeclareMathOperator{\catpoint}{\mathbf{Based}}
\DeclareMathOperator{\catdeduce}{\mathbf{Deduce}}
\DeclareMathOperator{\catfun}{\mathbf{Fun}}

 % Common category-theoretic macros
\setcounter{tocdepth}{1} % Only show major sections in the ToC
\addbibresource{report.bib}
\setlength{\bibitemsep}{3pt}

% Add some additional padding around items in list environments.
\setlist{itemsep = 3pt, topsep = 5pt}

% Make bullet points slightly smaller (they look overwhelming in a 10pt font).
\renewcommand{\labelitemi}{$\vcenter{\hbox{\tiny$\bullet$}}$}
\renewcommand{\mathbb}[1]{\ensuremath\mathbf{#1}}

\renewcommand{\sectionautorefname}{Section}
\renewcommand{\subsectionautorefname}{\sectionautorefname}

\numberwithin{equation}{section} % Number equations with respect to the section
\numberwithin{listing}{section}  % ...\ listing environments ...
\numberwithin{figure}{section}   % ...\ figure environments  ...

% Add appropriately padded frames around numbered figures and sub-figures.
\setlength{\belowcaptionskip}{1.5em}
\AtBeginEnvironment{subfigure}{\setlength{\abovecaptionskip}{0pt}}
\AtBeginEnvironment{figure}{\setlength{\abovecaptionskip}{2em}}
\floatstyle{boxed}
\restylefloat*{figure}

% -- Begin header/footer options

\setlength{\headheight}{13pt}
\setlength{\footskip}{30pt}
\setlength{\topmargin}{0pt}

\renewcommand{\footrulewidth}{.3pt}
\fancypagestyle{fancy}{%
	\fancyhf{}
        \renewcommand{\headrulewidth}{\footrulewidth}
        \fancyhead[C]{\footnotesize\textsc{\plainshorttitle}\vspace{.3em}}
        \fancyfoot[R]{\footnotesize Page \thepage{} of~\pageref{LastPage}}
        \fancyfoot[L]{\footnotesize Compilation Date: \today}
}
\fancypagestyle{firstpage}{%
        \fancyhead{}
        \renewcommand{\headrulewidth}{0pt}
}

\newcommand\anonfootnote[1]{%
        \begingroup
        \renewcommand\thefootnote{}\footnote{#1}%
        \addtocounter{footnote}{-1}%
        \endgroup
}

% -- End header/footer options

% -- Begin listing options for Haskell

\newcommand{\inlinehask}[1]{\mintinline{haskell}{#1}}
\newcommand{\barehask}[1]{\Colorbox{codebg}{\lstinline[style = inlineHask]{#1}}}
\newcommand{\haskmath}[2]{%
        \ensuremath\text{\Colorbox{codebg}{%
                \vphantom{\lstinline[basicstyle = \haskMorphStyle]{fg}}%
                \lstinline[style = inlineHask, basicstyle = #2]{#1}%
        }}%
}
\AtBeginEnvironment{listing}{\setlength{\abovecaptionskip}{.8em}}
\newcommand{\listingautorefname}{Listing}

% Use Courier instead of CM for fixed-width text. This is an aesthetic choice
% that also affects hyperref, but we need Courier for supporting the boldface
% fonts in Haskell formatting environments (and related commutative diagrams).
\renewcommand{\ttdefault}{pcr}

\newcommand{\haskMorphStyle}{\normalsize\ttfamily\color{haskkeyword}}
\newcommand{\haskObjStyle}{\normalsize\ttfamily\color{black}}
\newcommand{\haskFuncStyle}{\normalsize\ttfamily\color{haskkeyword}\bfseries}

\definecolor{codebg}{RGB}{230,230,230}
\definecolor{haskkeyword}{RGB}{173,46,252}

% I am using the Listings package for printing inline Haskell, and Minted for
% printing verbatim blocks. In general, Minted is far superior, but Listings is
% more crudely customisable for pseudo-code inline blocks that require very
% literal pattern-matching for given tokens.

\setminted{%
        frame = lines,
        framesep = 2mm,
        baselinestretch = 1.2,
        bgcolor = codebg,
        fontsize = \footnotesize,
        linenos
}

\lstdefinestyle{inlineHask}{%
        language = Haskell,
        alsoletter = {->, ::},
        morekeywords = {->, ::, fmap},
        keywordstyle = \color{haskkeyword}\bfseries,
        basicstyle = \footnotesize\ttfamily
}

% -- End listing options for Haskell

% \yorkemail: takes the username and outputs a clickable mailto link.
\newcommand{\yorkemail}[1]{\href{mailto:#1@york.ac.uk}{#1@york.ac.uk}}

% \startcontribution: open a new contribution section:
%
%       #1: The author's initials;
%       #2: The author's name;
%       #3: Section title;
%       #4: Prose page count;
%       #5: (Prose) word count.
%
\newcommand{\startcontribution}[5]{%
        \section{\for{toc}{\texorpdfstring{\textbf{[#1]~}}{}}#3}
        \begin{flushright}
                \vspace{5pt}
                \textbf{[Written by #2]}

                \vspace{5pt}
                \textit{Prose, Excluding Floats:} #4 pages

                \textit{Word Count of Prose:} Approx. #5 words
        \end{flushright}
}

% -- Begin document metadata setup

\newcommand{\plainshorttitle}{An Investigation of Elementary Category Theory}
\title[\plainshorttitle]{An Investigation of Elementary Category Theory, with %
        applications in Pure Mathematics and Theoretical Computer Science}

\author{Matthew Drury} \email{\yorkemail{md1499}}
\author{Ben Brook}     \email{\yorkemail{bb1170}}
\author{Oliver Dixon}  \email{\yorkemail{od641}}

\address{Department of Mathematics \\
        University of York \\
        United Kingdom}
\date{Spring--Summer Term, 2023}

% -- End document metadata setup

\begin{document}
\thispagestyle{firstpage}
\pagestyle{fancy}
\begin{abstract}
        In this preliminary investigation of elementary Category Theory, we
        discuss the foundational concepts of analytical abstraction, coupled
        with an exploration of the methods by which the study of categories
        facilitate the discovery of fundamental insights into complex networks
        of mathematical structure. We exploit this understanding to establish
        basic parallels between categorical instantiations of structure and
        modern concepts in Abstract Algebra, Set Theory, and Logic. A commentary
        of Category Theory in Computer Science and Functional Programming is
        also included, throughout which we interlace theoretical discourse with
        concrete examples in the purely functional Haskell programming language.
\end{abstract}
\maketitle
\tableofcontents
\vspace{-1.5\belowcaptionskip}
\par\noindent\rule{\textwidth}{\headrulewidth}
\vspace{.15em}
\startcontribution{MD}{Matthew Drury}%
        {Theoretical Underpinnings: Axiomatic Constructions}%
        {3--1/2}{1839} % 3.5 pages, ~1839 words
% MAJOR REPORT SECTION
%
% AUTHOR: MATTHEW DRURY
% TITLE:  THEORETICAL UNDERPINNINGS: AXIOMATIC CONSTRUCTIONS
%
\subsection{The Essence of Categorical Thinking}

A category is a mathematical structure that links a collection of objects with
non-symmetric relationships called morphisms. Each morphism can be said to
traverse from one object to another. They are used to create abstract models of
mathematical theories based on the role each object plays. Categories have the
versatility to talk about different areas of mathematics in the same language,
which is why it is sometimes said to be the mathematics of mathematics. The
objects and morphisms have little restriction as to what they can represent
which can be seen in the commutative diagrams illustrated in
\autoref{fig:initial-cd} (\autoref{fig:factor-cd} is adapted
from~\autocite{Cheng:2022}).
\begin{figure}[ht]
        \vspace{\belowcaptionskip}
        \begin{subfigure}{.45\textwidth}
                \centering
                \begin{tikzcd}
                        2 \ar[r] \ar[dr] & 6 \ar[dr]\\
                        3 \ar[ur] \ar[dr] & 10 \ar[r] & 30\\
                        5 \ar[ur] \ar[r] & 15 \ar[ur]
                \end{tikzcd}%
                \vspace{1em}%
                \caption{Factors}%
                \label{fig:factor-cd}
        \end{subfigure}
        \vspace{-\belowcaptionskip}%
        \begin{subfigure}{.45\textwidth}
                \centering
                \begin{tikzcd}
                        \text{York} \ar[d] \ar[dr] \\
                        \text{Leeds} \ar[d] \ar[dr] \ar[r] & \text{Hull} \ar[d] \\
                        \text{Manchester} \ar[r] & \text{Sheffield}
                \end{tikzcd}%
                \vspace{1em}%
                \caption{Train routes from York to Sheffield}
        \end{subfigure}%
        \caption{Intuitive Examples of Categories}%
        \label{fig:initial-cd}
\end{figure}

These diagrams are usually used to give a sample of a category to demonstrate
these properties. Mathematicians like to generalise with categories and consider
infinite objects and morphisms, meaning we want categories to link all types of
objects that we can define. The most valuable mathematical facts tend to be
those that apply to the widest range of situations. For instance, uncovering the
quadratic formula being more important than solving a particularly difficult
quadratic. It makes all non-trivial quadratics easier to solve and uncovers a
method to determine whether the solutions are real or complex using the
discriminant term ``$b^2-4ac$''. By doing this we have found a relationship
between all quadratics and their solutions. From there we could generalise even
further and think about how this is significant for higher order polynomials. 

Morphisms are combined together in a process called composition. It means to do
one morphism and then another, given that the target object of the first
morphism is the same as the source of the second morphism. Above we can see that
we can go from York to Leeds to Manchester. This means we have a morphism that
goes from York to Manchester. There are four different ways to get to Sheffield,
meaning four morphisms. These may not be the same morphisms, as they only share
in source and target, not necessarily meaning. When two different composite
morphisms are equal, we say they commute, hence the name \emph{commutative
diagram}.

There are also identity morphisms. These go from an object to itself and are
equivalent to not doing a morphism at all, as when they are composed with
another morphism $f$, the resulting composite morphism is equal to $f$. The
morphism has done nothing, like how multiplying a number by 1(The multiplicative
identity) does not change its value. It is similar to stating that something is
equal to itself which is a trivial fact but an important feature of the
structure of categories. It should be noted that it may not be the only morphism
from one object to itself. These are called endomorphisms and the identity is
just one of them.

Categories illuminate subtle similarities. By creating categories we abstract
things to their roles and by comparing categories or spotting patterns and
structures within them, we can see properties shared by things that would
initially seem very different. Lines of thought in one context can be more
easily translated into another by seeing equivalent structural features.
Mathematicians can then try new methods to solve problems and prove new facts or
describe multiple problems as one that is more general.

\subsection{Axioms and Notation}%
\label{sec:cat-axioms}

We will now formally define a category. The following axioms are necessary for a
structure to be considered a category.
\begin{enumerate}
        \item \textbf{Objects:} In a category $\arbcat{C}$ there is
        $\catobj(\arbcat{C})$, which is a collection of all the objects of
        $\arbcat{C}$.

        \item \textbf{Morphisms:} For each pair ordered pair of objects $(A,B)$
        in a category $\arbcat{C}$ we have $\hom_\arbcat{C}(A,B)$, also written
        as $\arbcat{C}(A,B)$.  This is the collection of all morphisms from A to
        B, short for ``homomorphism set''. Homomorphisms are those that preserve
        the structure of objects and for many useful categories they are
        necessary for the axioms to hold. However, this property does not need
        to hold to use this notation. A morphism $f$ is notated to map $A$ to
        $B$ by $f\colon A\to B$.

        \item \textbf{Composition:} For objects $A,B,C\in \catobj(\arbcat{C})$
        if there exists morphisms $(f\colon A\to B)\in \hom_\arbcat{C}(A,B)$ and
        $(g \colon B\to C)\in \hom_\arbcat{C}(B,C)$, then there exists $(g\circ
        f\colon A\to C)\in \hom_\arbcat{C}(A,C)$

        \item \textbf{Identities:} Every object $A$ in a category $\arbcat{C}$
        has an identity morphism which maps from $A$ to itself.  It is notated
        $1_A\colon A\to A$ for $A\in ob(\arbcat{C}), 1_A\in
        \hom_\arbcat{C}(A,A)$.  For $f\in \hom_\arbcat{C}(A,B), g\in
        \hom_\arbcat{C}(B,A)$, the identity has the property that $f\circ 1_A =
        f, 1_A\circ g = g$. Each identity is unique to its object. If there were
        two identities $I_1,I_2\in A$ then $(f\circ I_1 = f = f\circ I_2)
        \implies I_1=I_2$

        \item \textbf{Associativity:} A morphism $h\circ(g\circ f) = (h\circ
        g)\circ f$. This means that two morphisms are equal if they are composed
        from the same morphisms in the same order regardless of the order in
        which we compute the individual composite pairs.
\end{enumerate}
\begin{figure}
        \vspace{\belowcaptionskip}
        \begin{subfigure}{.3\textwidth}
                \centering
                \vspace{1em}
                \begin{tikzcd}
                        A
                                \ar[dr, "g\circ f", bend left=20,
                                        dashrightarrow]
                                \ar[d, "f"] \\
                        B
                                \ar[r, "g"]
                                \ar[dr, "h\circ g", bend right=20,
                                        dashrightarrow, swap] &
                        C
                                \ar[d, "h"] \\ &
                        D
                \end{tikzcd}%
                \vspace{1em}
                \caption{Composition}
        \end{subfigure}
        \hfill
        \begin{subfigure}{.15\textwidth}
                \centering
                \vspace{1em}
                \begin{tikzcd}
                        A
                                \ar[loop right,"1_A"]
                                \ar[d , "f" swap] \\
                        B
                                \ar[loop right,"1_B"]
                                \ar[d , "g" swap] \\
                        C
                                \ar[loop right,"1_C"]
                \end{tikzcd}%
                \vspace{1em}
                \caption{Identities}
        \end{subfigure}
        \hfill
        \begin{subfigure}{.4\textwidth}
                \centering
                \begin{tikzcd}
                        A
                                \ar[r,"f"]
                                \ar[rr, "g\circ f", bend right=40]
                                \ar[rrr, "h\circ(g\circ f)",%
                                        bend right=60]
                                \ar[rrr, "(h\circ g)\circ f",%
                                        bend left=60,] &
                        B
                                \ar[r, "g"]
                                \ar[rr, "h\circ g", bend left=40] &
                        C
                                \ar[r,"h"]  &
                        D
                \end{tikzcd}%
                \caption{Associativity}
        \end{subfigure}%
        \vspace{-\belowcaptionskip}
        \caption{Essential Properties of Categorical Structures}
\end{figure}

\subsection{Size of Categories}

The collection $\catobj(\arbcat{C})$ and the collections $\hom_\arbcat{C}(A,B)$
do not have to be finite. The idea of category size means to place categories in
a hierarchy of containment. This is to accommodate how $\catobj(\arbcat{C})$ and
$\hom_\arbcat{C}(A,B)$ do not have to be sets either.

When defining an infinite collection of objects by their properties, it can
create contradictions. Famously in set theory there is Russell's
paradox~\autocite{Russell:1903}. It states that if you have a set S that
contains every set which doesn't contain itself, then suppose that S does not
contain S, it would imply that S is in fact in S as it does not contain itself
and vice versa. Formally, 
\begin{equation}
S = \{x\;\text{is a set}\:|\:x\notin x\}, (S\in S) \iff (S\notin S)
\end{equation}
There were many examples of these paradoxes that were uncovered in the early
20th century. Mathematicians are too rigorous to allow such a thing, so to solve
the issue they devised axiomatic systems to limit the properties that members of
sets can be said to follow. This makes some sets very convoluted to define and
means there are well defined collections of objects that cannot be put into a
set. Categories address this issue in a more elegant way. Firstly, categories do
not claim to contain elements; the core concept is instead relationships.
Secondly, categories have a hierarchy of containment such that they only contain
categories ``smaller'' than themselves. It follows then that categories are not
defined in a self referential manner like in Russell's paradox.

A small category is where $\catobj(\arbcat{C})$ and each $\hom_\arbcat{C}(A,B)$
can be described as a set. A locally small category is where
$\catobj(\arbcat{C})$ does not form a set but each $\hom_\arbcat{C}(A,B)$ does.
A large category is where nether $\catobj(\arbcat{C})$ nor
$\hom_\arbcat{C}(A,B)$ are a set. The hierarchy means that if one wanted a
category where small categories are objects, it would have to be a large
category. Then, if you wanted a category of large categories, you would need a
super-large category and so on. This can be used to formally think about
multiple layers of generalization and abstraction.

We can easily avoid Russell's paradox and create categories of categories. This
is a sign that categories are a robust idea, as we can think about a collection
of categories using the same language as within individual categories.

\subsection{Functors}

A functor maps between the objects of two categories, similarly to how a
function maps between two sets.  Functors preserve the structure of the
category, so we want equivalent morphisms between the equivalent objects that
have been chosen. This relationship is illustrated in
\autoref{fig:functor-intro}.
\begin{figure}[ht]
        \vspace{\belowcaptionskip}
        $\underbrace{\begin{tikzcd}[sep=large, ampersand replacement=\&]
                A \arrow[r, "f"] \arrow[rd, "g \circ f" swap]
                \& B \arrow[d, "g"] \\
                \& C
        \end{tikzcd}}_{\substack{\\[.2em]\text{\small Domain Category
                $\arbcat{C}$}}}$%
        \begin{tikzcd}[sep=large, ampersand replacement=\&]
                {} \arrow[rr, FUNCTOR, %
                        "\substack{\text{\large\textit{F}}\\[.2em]}", %
                        yshift=-.8em]
                \& \& {}
        \end{tikzcd}
        $\underbrace{\begin{tikzcd}[sep=large, ampersand replacement=\&]
                \arbfunc{F}(B) \arrow[d, "\arbfunc{F}(g)" swap] \&
                \arbfunc{F}(A) \arrow[l, "\arbfunc{F}(f)" swap]
                        \arrow[dl, "\arbfunc{F}(g \circ f)"]
                \\ \arbfunc{F}(C) \&
        \end{tikzcd}}_{\substack{\\[.2em]\text{\small Codomain Category
                $\arbcat{D}$}}}$%
        \caption{A functor $\arbfunc{F}$ between simple categories
                $\arbcat{C}$ and $\arbcat{D}$}
        \label{fig:functor-intro}
\end{figure}

For categories $\arbcat{C}$, $\arbcat{D}$, if we have a functor $\arbfunc{F}
\colon \arbcat{C}\to\arbcat{D}$ between them:
\begin{enumerate}
        \item \textbf{Objects:} For every $A\in\catobj(\arbcat{C})$, 
        $F_A\in\catobj(\arbcat{D})$. Thus, $A$ is mapped to $F(A)$.

        \item \textbf{Morphisms:} For every $f\in\hom_{\arbcat{C}}(A,B)$, we
        have $F(f)\in\hom_{\arbcat{D}}(F_A,F_B)$. Thus, $f$ is mapped to $F(f)$.

        \item \textbf{Identities:} For every $A\in\catobj{\arbcat{C}}$, we have
        $1_A$. For every $1_A$, there is corresponding morphism
        $F(1_A)\in\hom_{\arbcat{D}}(F(A),F(A))$. Thus, the identity of $A$ is
        mapped to the identity of $F(A)$.

        \item \textbf{Composites:} For every $f\in\hom_{\arbcat{C}}(A,B)$,
        $g\in\hom_{\arbcat{C}}(B,C)$, we have that $F{(g\circ f)} = F_g\circ
        F_f$. This means that the order in which compositions and functors are
        applied is unimportant. This implies that composition in each category
        has an equivalent effect on its morphisms.
\end{enumerate}
The commutative diagram above shows how from $A$ we can traverse $g\circ f
\colon A\to C$ and then apply $F$, or we can apply $F$, then traverse $F(g \circ
f) \colon F(A)\to F(C)$.

In general, a functor is a way of finding the structure of one category in
another.  For instance, a category with two objects and one morphism between
them $f \colon A\to B$ can have a functor to any category by selecting any
morphism in it, which could even be the identity of an object.  This principle
extends to categories of infinite objects.

\subsection{Non-Small Categories}

The large category $\catcat$ has objects which are all small categories and
homomorphisms which are all functors between them.  Each functor $F \colon
\arbcat{C}\to\arbcat{D}$ can be seen as a function between sets
$\catobj(\arbcat{C})$ and $\catobj({\arbcat{D}})$.  It could be injective,
meaning that each object of $\arbcat{C}$ maps to a different object of
$\arbcat{D}$. Otherwise, the functor is showing that one object in $\arbcat{D}$
has the same categorical properties as a structure of objects in $\arbcat{C}$.

If it is bijective, then the functor is an isomorphism, a morphism that has an
inverse which composes to form the identity. This implies that $\arbcat{C}$ and
$\arbcat{D}$ can reach the same categories with functors and be reached by the
same categories with functors. In general, isomorphisms are a way of saying that
objects indistinguishable from the perspective of the category they are in.
However, it is not a suitable notion of equivalence between categories.

The identities in $\catcat$ are identity functors which map a category to a copy
of itself with the exact same objects, morphisms, compositions. More generally
it is one of the endofunctors which map categories to themselves.

Composition is simply doing one functor after another. For functors
$F\colon\arbcat{C}\to\arbcat{D}$, $G \colon \arbcat{D}\to\arbcat{E}$, we can
describe a single functor $G\circ F\colon\arbcat{C}\to\arbcat{E}$.  When a lot
of functors have been composed together we can get an increasingly subtle
structural similarity between categories that we would not be able to notice by
laying out diagrams next to each other and seeing a pattern.
%
  % MAJOR SECTION INCLUDE: MATTHEW DRURY (1/3)

\vspace{\belowcaptionskip}
\par\noindent\rule{\textwidth}{\headrulewidth}
\vspace{-.15em}

\startcontribution{BB}{Ben Brook}%
        {Category-Theoretic Interpretations of Familiar Structures}%
        {2--1/2}{1294} % 2.5 pages, ~1294 words
% MAJOR REPORT SECTION
%
% AUTHOR: BEN BROOK
% TITLE:  CATEGORY-THEORETIC INTERPRETATIONS OF FAMILIAR STRUCTURES
%
\setlength{\jot}{5pt}
\subsection{Introductory Examples}

Now that we have established an axiomatic definition of the category, the
natural next step involves the exploration of some simple applications of
categories in a familiar context.

For the first three of these examples, our objects will be
nothing but sets and morphisms nothing but functions (both with some additional
restrictions). Categories of this type are sometimes called
\emph{concrete} \autocite{Awodey:2010}.
However, it's important to remember that categories need not
contain functions nor sets. In fact, we will see this in our final introductory
example, where we do away with both entirely!

All the exemplary categories that we are going to touch on will be infinite.
This means that we won't be using any diagrams for this section of the report.

\subsection{Category of Sets}

To begin, an easily digestible example is the category of sets, whose objects
are sets, and morphisms are total functions between sets. These are the
set-theoretic functions with which we are familiar and have been using
throughout the Autumn Term.  We will follow a common denotation for this
category: $\catset$~\autocite{Leinster:2014}. We can easily verify that this
category satisfies the category axioms seen in \autoref{sec:cat-axioms}.
\begin{itemize}
        \item There are objects, which are sets.
        \item There are morphisms between those objects, which are functions.
                Remember, however: morphisms are not necessarily functions! In
                fact, there are an abundance of examples where this is not the
                case.
        \item The composition of two functions is also a function, with its
                domain and codomain coming from the functions involved in the
                composition operation. Hence, the axiom of composition is
                satisfied.
        \item Let $A \in \catobj{(\catset)}$. We have that the identity
                morphism for $A$ is $(1_A \colon A \to A) \in \hom (A, A)$,
                which maps each element of $A$ to itself.
        \item The morphisms of $\catset$ are associative due to the
                native associativity of function composition. Let $A, B, C, D
                \in \catobj{(\catset)}$ be distinct. Also, let $f \in \hom (A,
                B)$, $g \in \hom (B, C)$, and $h \in \hom (C, D)$. Thus,
                \begin{equation}
                        ( f \circ g ) \circ h = f \circ ( g \circ h ).
                \end{equation}
\end{itemize}

Hence, all the category axioms are satisfied. Going forwards, we will avoid the
pedantry of listing that our category contains morphisms and objects.

\subsection{Category of Partially Ordered Sets}

A nice path to advance down is the exploration of a category whose objects are
sets that satisfy certain properties. In other words, the objects are sets that
have additional structure imposed upon them. We will also assign functions that
preserve this structure to be the morphisms of our category. In this case, let's
think about partially ordered sets, or \emph{posets}. We must first establish
some definitions:
\begin{itemize}
        \item A poset is a set, $A$, with a relation, $\sim_A$, that is
                reflexive, transitive, and antisymmetric. We covered these
                properties in the Autumn Term.
        \item Let $A$ and $B$ be posets. A monotone map $m \colon A \to B$ is
                an order-preserving function such that
                \begin{equation}
                        \forall\, a, b \in A \qquad a \sim_A b
                        \implies m(a) \sim_B m(b).
                \end{equation}
\end{itemize}
We have a category $\catpos$ whose objects are posets and morphisms are monotone
maps between those posets. Once more, we can verify the satisfaction of the
relevant axioms.
\begin{itemize}
        \item Let $A, B, C \in \catobj{(\catpos)}$ be distinct. Also, let
                $f \in \hom (A, B)$ and $g \in \hom (B, C)$. Since monotone maps
                are functions, we can compose them to get  $g \circ f \colon A
                \to C$. Is this new function also a monotone map? Well, for
                all $a_1, a_2 \in A$, we have that
                \begin{align}
                        \quad a_1 \sim_A a_2 &\implies f(a_1) \sim_B f(a_2) \\
                        &\implies g(f(a_1)) \sim_C g(f(a_2)).
                \end{align}
                Thus, $g \circ f$ is a monotone map, and the composition axiom
                is satisfied.
        \item The identity morphism for any fixed object, $A$, is the
                monotone map $1_A \colon A \to A$, such that $a \mapsto a$.
        \item The associativity axiom is satisfied since our morphisms are
                functions, and function composition is associative.
\end{itemize}

\subsection{Category of Finite-Dimensional Vector Spaces}

We can look to Linear Algebra to provide another simple infinite category. The
category of finite-dimensional $\mathbb{R}$\emph{-vector spaces} is often denoted
$\catrvect$ \autocite{Hasegawa:2008}.
\begin{itemize}
        \item A finite-dimensional $\mathbb{R}$-vector space is an element of
                $\{ \mathbb{R} ^n \, | \, n \in \mathbb{N} \}$.
        \item A linear map is a function, $f$, between two vector spaces,
                such that vector addition and scalar multiplication is
                preserved. That is,
                \begin{alignat}{2}
                        f( \vec{u} + \vec{v} ) &= f( \vec{u} ) + f( \vec{v} );
                        \qquad && \text{[\textbf{Additivity}]}%
                                \label{eqn:vs-additivity}\\
                        f( \lambda \vec{u} ) &= \lambda f( \vec{u} ),
                        \qquad && \text{[\textbf{Homogeneity}]}%
                                \label{eqn:vs-homo}
                \end{alignat}
                with $\lambda \in \mathbb{R}$ and
                $\vec{u},\vec{v} \in \mathbb{R} ^n$ for some $n \in \mathbb{N}$.
\end{itemize}
The objects of $\catrvect$ are finite-dimensional real-valued vector spaces, and
its morphisms are linear maps between them. We will verify that $\catrvect$
satisfies the category axioms:
\begin{itemize}
        \item Let $\mathbb{R} ^a, \mathbb{R} ^b, \mathbb{R} ^c \in
                \catobj{(\catrvect)}$ be distinct. Additionally, suppose that
                $f \in \hom ( \mathbb{R} ^a, \mathbb{R} ^b)$ and
                $g \in \hom ( \mathbb{R} ^b, \mathbb{R} ^c)$.
                Trivially, $g \circ f \colon \mathbb{R} ^a \to \mathbb{R} ^c$
                is a map. Now, let $\vec{u} , \vec{v} \in \mathbb{R} ^a$
                and $\lambda \in \mathbb{R}$. We have that
                \begingroup
                \renewcommand{\equationautorefname}{Eqn.}
                \begin{alignat}{2}
                        (g \circ f)( \vec{u} + \vec{v} )
                        &= g(f( \vec{u} )) + g(f( \vec{v} ))\nonumber \\
                        &= (g \circ f)( \vec{u} ) + (g \circ f)( \vec{v} )
                        \qquad&&\text{[By \autoref{eqn:vs-additivity}]} \\[.5em]
                        %
                        (g \circ f)( \lambda \vec{u} )
                        &= \lambda g(f( \vec{u} ))\nonumber \\
                        &= \lambda (g \circ f)( \vec{u} ),
                        \qquad&&\text{[By \autoref{eqn:vs-homo}]}
                \end{alignat}
                \endgroup
                so the axiom of composition is met.
        \item The identity morphism for any object is the linear map such that
                $\vec{u} \mapsto \vec{u}$.
        \item The associativity axiom is satisfied due to functional nature of
                the maps.
\end{itemize}

\subsection{Category of Propositions}
As mentioned at the start of this section, the final exemplary category that we
cover is not concrete. In this regard, it is similar to the categories we will
see later in this report. Here we leverage how objects and morphisms can be
anything such that the category axioms are met.

In this case, we are going to explore a category whose objects are propositions
and morphisms are proofs, which we will denote $\catprop$. We will consider this
category informally, by employing our well-established intuition of proofs.
Suppose $P, Q \in \catobj{(\catprop)}$. If there is a proof which, under $P$,
gives $Q$ -- or a proof leading from $P$ to $Q$ -- then this gives a morphism
$(f \colon P \to Q) \in \hom (P, Q)$ (which is not a function!) We will use $P
\vdash Q$ to denote this. We can confirm the categorical nature of this
structure:
\begin{itemize}
        \item Let $P, Q, R \in \catobj{(\catprop)}$ be distinct and let
                $f \in \hom (P, Q), g \in \hom (Q, R)$. Then $P \vdash Q \land
                Q \vdash R$ so, intuitively, $P \vdash R$ and $g \circ f$ belongs
                to the category.
        \item We can also intuitively reason that there is a proof leading from
                any proposition to itself, so each object has an identity
                morphism.
        \item Finally, we know that morphism composition is associative, again
                through our intuition. Let $P, Q, R, S \in \catobj{(\catprop)}$
                be distinct. Let $f \in \hom (P, Q)$, $g \in \hom (Q, R)$, and
                $h \in \hom (R, S)$. That $(h \circ g) \circ f$ is a morphism
                asserts both that there is a proof leading from $Q$ to $S$, and
                that $P$ leads to $Q$. Likewise, that $h \circ (g \circ f)$ is a
                morphism asserts both the existence of a proof leading from $P$
                to $R$, and that $R$ leads to $S$. These two assertions are
                identical, thus morphism composition is associative.
\end{itemize}
We covered this category informally in that some facts stated above do not hold
for \emph{every} conceivable system of logic, but just some particularly
well-behaved constructions\footnote{In the scope of this report, the
distinctions between varying systems of logic is unimportant.}. Our intuition
was hopefully enough to take value from the example. A further, and more
rigorous, exploration of this concept is detailed by \autocite{Baez:2009}.
%
  % MAJOR SECTION INCLUDE: BEN BROOK (2/3)

\clearpage
\startcontribution{OD}{Oliver Dixon}%
        {Further Applications: Functional Programming and %
                \texorpdfstring{$\lambda$}{Lambda}-Calculus}%
        {3--1/2}{1072} % 3.5 pages, ~1072 words
\label{sec:funcprog}
% MAJOR REPORT SECTION
%
% AUTHOR: OLIVER DIXON
% TITLE:  FURTHER APPLICATIONS: FUNCTIONAL PROGRAMMING AND LAMBDA-CALCULUS
%
\subsection{Functional Programming and Haskell}

In purely functional languages, there is no allowance for context, or mutable
variables of any kind. Each function must accept some data, perform some strict
transformation upon the data---as defined by the algorithm---and return the
result. Whilst this robust paradigm does open a wide range of mathematical
avenues involving proof, safety, and reproducibility, the prohibition of
stateful computation renders many common tasks, such as I/O or socket
communication, largely impossible, as these imperatively defined operations
inherently contravene the purity principles of functional programming.

Haskell is a commonly used purely functional programming language, and suffers,
as do all languages in the same class, from this blaring issue.  Indeed, early
versions of Haskell did not support the chaining of stateful computation in any
sense, due to the obligatory absence of a fixed execution order in functional
paradigms; programmers were forced to resort to breaking the purity of the
language through aesthetically unpleasant techniques, ultimately obviating the
mathematical essence of the Haskell formal type system.

Due to the strength of the Haskell type system and function interface, we may
define a corresponding category, $\cathask$, within which the objects are
Haskell types, and the morphisms are functions\footnotemark.
%
\footnotetext{Due to the $\lambda$-Calculus concept of \emph{currying}, named
after Haskell Curry, functions taking multiple arguments may be decomposed into
a chain of function compositions, in which each function strictly accepts and
returns a single argument. This is made explicit in Haskell, where the type
signature of a function \inlinehask{f} may be defined as \inlinehask{f :: a -> b
-> c}, invoked as \inlinehask{f a b}, and expected to return a value of type
\barehask{c}. This function signature is trivially equivalent to the
\emph{uncurried} form of \inlinehask{f}, defined as \inlinehask{f :: (a -> b) ->
c}.} %
By analysing the structure of $\cathask$, its endofunctors, and the categories
formed by taking the product, we can draw a swathe of parallels between generic
purely functional paradigms; these connexions will eventually reveal the
powerful concept of \emph{monads}, allowing stateful computation, control flow,
and error-handling.

\subsection{Monoidal Categories and Monoids}%
\label{sec:monoidal-categories}

Before fully exploiting the structure of $\cathask$, we must develop the theory
of \emph{monoidal categories} and their corresponding \emph{monoids}. Monoidal
categories can be regarded as a six-tuple $(\arbcat{C}_0, \otimes, I, \alpha,
\lambda, \rho)$, containing various components \autocite{Kelly:1982}:
\begin{itemize}
        \item A base category, $\arbcat{C}_0$;

        \item A bifunctor $\otimes \colon \arbcat{C}_0 \times \arbcat{C}_0 \to
        \arbcat{C}_0$;

        \item An identity object $I \in \catobj \arbcat{C}_0$;

        \item An associativity natural transformation $\alpha_{A,B,C} \colon (A
        \otimes B) \otimes C \to A \otimes (B \otimes C)$;

        \item A left-identity natural transformation $\lambda_A \colon I \otimes
        A \to A$;

        \item A right-identity natural transformation $\rho_A \colon A \otimes I
        \to A$.
\end{itemize}

To maintain brevity, the natural transformations are often omitted from the
tuple-descriptions of monoidal categories: $(\arbcat{C}_0, \otimes, I)$. In this
context, the natural transformations $\alpha$, $\lambda$, and $\rho$ are used to
induce certain properties on $\otimes$---\emph{up to isomorphism}---which it
may not possess natively; the effects of these transformations can be
illustrated as morphisms on a pair of abstracted commutative diagrams (c.f.
\autoref{fig:monoidal-cat-commute}).

Then, monoids can be considered as three-tuples, consisting of an object in a
monoidal category $(\arbcat{C}, \otimes, I)$, coupled with two transformations:
\begin{itemize}
        \item A base object $M \in \catobj \arbcat{C}_0$;
        \item A \emph{multiplication} transformation $\mu \in \hom_\arbcat{C}
        (M \otimes M, M)$;
        \item A \emph{unit} transformation $\eta \in \hom_\arbcat{C} (I, M)$.
\end{itemize}
Once more, the natural transformations from the parent monoidal category
$\arbcat{C}$ can be used to induce properties on $\mu$, as in
\autoref{fig:monoid-commute}.

\begin{figure}[ht]
        \vspace{\belowcaptionskip}
        \begin{subfigure}{\textwidth}
                \centering
                \begin{tikzcd}[ampersand replacement=\&]
                        \left[(A \otimes B) \otimes C \right] \otimes D
                                \arrow[r, "\alpha"]
                                \arrow[d, "\alpha\,\otimes\,\catid D" swap] \&
                        (A \otimes B) \otimes (C \otimes D)
                                \arrow[r, "\alpha"] \&
                        A \otimes \left[ B \otimes (C \otimes D) \right] \\
                        \left[ A \otimes (B \otimes C) \right] \otimes D
                                \arrow[rr, "\alpha" swap] \& \&
                        A \otimes \left[ (B \otimes C) \otimes D \right]
                                \arrow[u, "\catid A\,\otimes\,\alpha" swap]
                \end{tikzcd}%
                \caption{Associativity on $\otimes$ induced by $\alpha$}
        \end{subfigure}
        \vspace{-\belowcaptionskip}
        \begin{subfigure}{\textwidth}
                \centering
                \begin{tikzcd}[sep=large, ampersand replacement=\&]
                        (B \otimes I) \otimes C
                                \arrow[rr, "\alpha"]
                                \arrow[rd, "\rho\,\otimes\,\catid C" swap] \& \&
                        B \otimes (I \otimes C)
                                \arrow[ld, "\catid B\,\otimes\,\lambda"] \\
                        \& B \otimes C \&
                \end{tikzcd}%
                \caption{Left- and right-identities of objects $A,B,C,D$ induced
                        by $\lambda$ and $\rho$}
        \end{subfigure}%
        \vspace{-.5em}
        \caption{\textbf{(Monoidal Categories)} The natural transformations
                invoke commutativity on the bifunctor $\otimes$.}
        \label{fig:monoidal-cat-commute}
\end{figure}
\begin{figure}[ht]
        \vspace{\belowcaptionskip}
        \begin{subfigure}{\textwidth}
                \centering
                \begin{tikzcd}[sep=large, ampersand replacement=\&]
                        (M \otimes M) \otimes M
                                \arrow[r, "\alpha"]
                                \arrow[d, "\mu\,\otimes\,\catid M" swap] \&
                        M \otimes (M \otimes M)
                                \arrow[r, "\catid M\,\otimes\,\mu"] \&
                        M \otimes M
                                \arrow[d, "\mu"] \\
                        M \otimes M
                                \arrow[rr, "\mu" swap] \& \& M
                \end{tikzcd}%
                \caption{Associativity on $\mu$ induced by $\alpha$}
        \end{subfigure}
        \vspace{-1.5em}
        \begin{subfigure}{\textwidth}
                \centering
                \begin{tikzcd}[sep=large, ampersand replacement=\&]
                        I \otimes M
                                \arrow[r, "\eta\,\otimes\,\catid M"]
                                \arrow[rd, "\lambda" swap] \&
                        M \otimes M
                                \arrow[d, "\mu" swap] \&
                        M \otimes I
                                \arrow[l, "\catid M\,\otimes\,\eta" swap]
                                \arrow[ld, "\rho"] \\
                        \& M \&
                \end{tikzcd}%
                \vspace{-.5em}
                \caption{Left- and right-identities of $M$ induced by $\lambda$
                        and $\rho$}
        \end{subfigure}%
        \caption{\textbf{(Monoids)} The natural transformations from the parent
                monoidal category also apply within the monoid.}
        \label{fig:monoid-commute}
\end{figure}

It is important that the domain of the $\mu$ transformation is a \emph{product
combination} of the monoidal object; this will allow the imperative-like
threading of state in functional paradigms.

\subsubsection{Examples of Monoidal Categories: Set Theory}

The canonical example of a monoidal category whose bifunctor/tensor product is
not associative is $\catset$, with the cross product; this is not
\emph{naturally associative}, but can be made associative up to isomorphism with
a suitable choice of the natural transformation $\alpha$. The details are
established in \autocite{Fong:2018}, using an associativity transform as
described in \autoref{eqn:assoc-transform-example}, providing a monoidal
category of the form $(\catset, \times, I)$, where $I$ represents some fixed
singleton.
\begin{equation}
        \alpha_{A,B,C} \colon (A \times B) \times C \to A \times (B \times C)%
        \label{eqn:assoc-transform-example}
\end{equation}

\subsubsection{Examples of Monoidal Categories: Haskell}

In Haskell, the simplest practical implementation of a monoidal category is
outlined in \autoref{lst:hask-tuple}, where the base category is $\cathask$, the
identity is the empty tuple, and the bifunctor is the tuple-building native
function.
\begin{listing}[ht]
        \inputminted{haskell}{haskell/Cross.hs}%
        \caption{A binary Haskell function \protect\inlinehask{cross} that
                encodes its arguments into a tuple. In category-theoretic
                language, the corresponding monoidal category could be expressed
                as the three-tuple
                $(\cathask, \text{\;\protect\barehask{cross}},
                \text{\;\protect\barehask{()}})$.}
        \label{lst:hask-tuple}
\end{listing}

\subsection{The Category of Endofunctors}

For the purposes of Functional Programming, and the wider formal treatment of
functional type systems, a particularly useful monoidal category concerns the
\emph{category of endofunctors} over some fixed base category $\arbcat{C}_0$.
Denoted as $\catendo{(\arbcat{C}_0)}$, this forms a monoidal category with the
associated bifunctor being the standard operation of endofunctor composition;
the identity element is the obligatory identity endofunctor. Objects in
$\catendo{(\arbcat{C}_0)}$ are the endofunctors over $\arbcat{C}_0$, and
morphisms are the natural transformations between these objects.  Monoids in the
category of endofunctors are sometimes called monads. Thus, endofunctors over
$\arbcat{C}_0$ with appropriately selected morphisms $\mu$ and $\eta$ are
henceforth termed as monads \autocite{MacLane:1998}.

\subsubsection{Monads in Haskell} An abbreviated definition of a Haskell
\inlinehask{Monad} is given in \autoref{lst:haskell-monad-definition}; here we
see the semi-curried form of the \emph{bind operator}. The importance of
\inlinehask{>>=} is reflected by its inclusion in the Haskell logo!
\vspace{-.5em} % Can I get away with this? :)
\begin{listing}
        \inputminted{haskell}{haskell/Monad.hs}%
        \caption{The Haskell \protect\inlinehask{>>=} and
                \protect\inlinehask{return} functions allow programmers to
                interact with the \protect\inlinehask{Monad} class in the
                categorical sense.}
        \label{lst:haskell-monad-definition}
\end{listing}

\begin{figure}[ht]
        \vspace{\belowcaptionskip}
        \begin{tikzcd}[outer sep=5pt, ampersand replacement=\&]
                \haskmath{a}{\haskObjStyle}
                        \arrow[r, "\haskmath{f}{\haskMorphStyle}" swap]
                        \arrow[bend left=40, rr, "\haskmath{f.g}%
                                {\haskMorphStyle}"] \&
                \haskmath{b}{\haskObjStyle}
                        \arrow[r, "\haskmath{g}{\haskMorphStyle}" swap] \&
                \haskmath{c}{\haskObjStyle}
        \end{tikzcd}
        \begin{tikzcd}[ampersand replacement=\&]
                {} \arrow[rr, FUNCTOR, "\haskmath{[]}{\haskFuncStyle}"] \& \& {}
        \end{tikzcd}
        \begin{tikzcd}[sep=large, outer sep=5pt, ampersand replacement=\&]
                \haskmath{[a]}{\haskObjStyle}
                        \arrow[r, "\haskmath{map f}{\haskMorphStyle}" swap]
                        \arrow[bend left=40, rr,
                                "\haskmath{map \$ f.g}{\haskMorphStyle}"] \&
                \haskmath{[b]}{\haskObjStyle}
                        \arrow[r, "\haskmath{map g}{\haskMorphStyle}" swap] \&
                \haskmath{[c]}{\haskObjStyle}
        \end{tikzcd}%
        \caption{The action of the \inlinehask{[]} functor on $\cathask$. The
                list constructor functor \emph{lifts} the Haskell types
                \protect\barehask{a}, \protect\barehask{b}, and
                \protect\barehask{c} into the list context.}
        \label{fig:functor-list-map}
\end{figure}

\begin{figure}[ht]
        \vspace{\belowcaptionskip}
        \begin{tikzcd}[outer sep=5pt, ampersand replacement=\&]
                \haskmath{a}{\haskObjStyle}
                        \arrow[r, "\haskmath{f}{\haskMorphStyle}" swap]
                        \arrow[bend left=40, rr, "\haskmath{f.g}%
                                {\haskMorphStyle}"] \&
                \haskmath{b}{\haskObjStyle}
                        \arrow[r, "\haskmath{g}{\haskMorphStyle}" swap] \&
                \haskmath{c}{\haskObjStyle}
        \end{tikzcd}
        \begin{tikzcd}[ampersand replacement=\&]
                {} \arrow[rr, FUNCTOR, "\haskmath{F}{\haskFuncStyle}"] \& \& {}
        \end{tikzcd}
        \begin{tikzcd}[sep=large, outer sep=5pt, ampersand replacement=\&]
                \haskmath{F a}{\haskObjStyle}
                        \arrow[r, "\haskmath{fmap f}{\haskMorphStyle}" swap]
                        \arrow[bend left=40, rr,
                                "\haskmath{fmap \$ f.g}{\haskMorphStyle}"] \&
                \haskmath{F b}{\haskObjStyle}
                        \arrow[r, "\haskmath{fmap g}{\haskMorphStyle}" swap] \&
                \haskmath{F c}{\haskObjStyle}
        \end{tikzcd}%
        \caption{The action of the generic functor \inlinehask{F} on some
                $\cathask$. The mappings of the objects are implicit in the
                mappings of the morphisms, as domains and codomains must be
                preserved between isomorphic categories.}
        \label{fig:functor-general-fmap}
\end{figure}

\subsection{Functors in Haskell}

This interdisciplinary review of Category Theory and Functional Programming
becomes useful when considering the category of endofunctors over the $\cathask$
category, $\catendo{(\cathask)}$, thus forming a monoidal category with monads
as endofunctors over $\cathask$ \autocite{Milewski:2019}. In Haskell, these are
simply typed as \inlinehask{Functor}, defined as a typeclass providing
appropriate mappings from $\cathask$ to $\cathask$ for types and functions, as
shown in \autoref{lst:hask-functor}. The Haskell function \inlinehask{fmap} is
used to lift a function \barehask{a -> b}, embedded in $\cathask$ as the domain
category, to the functorial context \barehask{f a -> f b}, embedded in
$\cathask$ as the codomain category.

Haskell Functors can be solidified with the most trivial example: the \emph{list
constructor}, which takes types $A,B,C \in \catobj \cathask$ and lifts them into
the list structure with the \inlinehask{fmap} endofunctor\footnotemark. This
process is illustrated in case of lists in \autoref{fig:functor-list-map}, and
in the general case in \autoref{fig:functor-general-fmap}.
%
\footnotetext{In the Haskell \inlinehask{[]} instantiation of
\inlinehask{Functor}, the \inlinehask{fmap} field is set to the stricter
\inlinehask{map} function; this is an unimportant implementation detail in this
case.}

\subsection{Controlling State with Haskell Monads}

To address our original problem of stateless computation, how might a
\emph{specific} usage of \inlinehask{Monad} allow the threading of state through
pseudo-imperative function calls? By recalling the domain of the
\emph{multiplication} transformation on the monoid, as defined in
\autoref{sec:monoidal-categories}, we have an immensely useful functor
transformation $\catendo{(\cathask)}$, such that $\mu \colon M \times M \to M$,
where $M$ is an object in $\arbcat{C}_0 \coloneqq \cathask$. This can be
implemented as a Haskell function which allows the merging of two $\cathask$
instances into a single \emph{combined instance}.
%
By exploiting the lazy evaluation of Haskell, such that functions are only
executed when directly invoked, programmers can enforce an execution order by
chaining evaluations of some transformation $\mu$. Context is achieved by
\emph{applying} a $\cathask$, as the standard function parameter, to a given
context; the function must then return the transformed context according to the
prescribed algorithm or process.

\begin{listing}
        \inputminted{haskell}{haskell/Functor.hs}%
        \caption{The Haskell \protect\inlinehask{Functor} type signature, of
                which the list type constructor \inlinehask{[]} is an instance.}
        \label{lst:hask-functor}
\end{listing}
%
% OWD 2023 :)
%
  % MAJOR SECTION INCLUDE: OLIVER DIXON (3/3)

\anonfootnote{\vspace{-.5em}}
\anonfootnote{%
        \raggedleft%
        \color{darkgray}{%
                [\hypersetup{linkcolor=darkgray}%
                \autoref{sec:funcprog}~is wholly dedicated to \textit{MQ}.]%
        }%
        \vspace{-\baselineskip} % Push past the standard footer baseline
}

\par\noindent\rule{\textwidth}{\headrulewidth}
\addtocontents{toc}{\protect\vspace{5pt}}
\vspace{-.5\baselineskip}
\printbibliography[title=Cited Works]
\renewcommand{\footnoterule}{}
\end{document}

