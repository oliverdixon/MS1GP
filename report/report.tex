% OWD 2023

\documentclass[10pt,a4paper]{amsart}
\usepackage[foot]{amsaddr}
\usepackage{xcolor,lmodern}

\usepackage[%
	backend = biber,
	sorting = none,
	bibencoding = utf8,
	style = alphabetic
]{biblatex}

% OWD 2023

% DO NOT COMPILE THIS FILE IN ISOLATION! It is designed to be included by other
% TeX files, and will not produce anything on its own (other than a swathe of
% meaningless temporary files and an incomprehensible error message).
%
% Whenever you need a macro involving categories, or load a package for the
% typesetting thereof, add it here, since it's likely that most macros will be
% mutually required between the report and presentation.

\usepackage{amsmath}
\usepackage{tikz-cd} % Drawing commutative diagrams with TikZ.
\usepackage{adjustbox} % Scaling tikz-cd commutative diagrams.

\tikzcdset{outer sep=2pt} % Apply some padding to arrows in comm. diagrams.

% Provide a thicker arrow style (1pt) for functor morphisms in pseudo-CDs.
\tikzset{FUNCTOR/.style = {line width=#1},%
         FUNCTOR/.default=1pt}

% Style note: please typeset named categories (e.g. Hask) in boldface, as shown
% below. For arbitrary categories (e.g. C or C_0), use \arbcat. For the homset,
% use the built-in \hom macro (\cathom is superfluous).

\newcommand{\arbcat}[1]{\ensuremath\mathcal{#1}}
\newcommand{\arbfunc}[1]{\ensuremath #1} % I might change this to \mathbf{...}

\DeclareMathOperator{\catobj}{ob}
\DeclareMathOperator{\catid}{id}
\DeclareMathOperator{\catcat}{\mathbf{Cat}}
\DeclareMathOperator{\catendo}{\mathbf{Endo}}
\DeclareMathOperator{\cathask}{\mathbf{Hask}}
\DeclareMathOperator{\catset}{\mathbf{Set}}
\DeclareMathOperator{\catpos}{\mathbf{Pos}}
\DeclareMathOperator{\catkvect}{\mathit{K}-\mathbf{Vect}}
\DeclareMathOperator{\catfinvect}{\mathbf{FinVect}_\mathbb{R}}
\DeclareMathOperator{\catpre}{\mathbf{Ord}}
\DeclareMathOperator{\catpoint}{\mathbf{Based}}
\DeclareMathOperator{\catdeduce}{\mathbf{Deduce}}
\DeclareMathOperator{\catfun}{\mathbf{Fun}}



\usepackage[%
	colorlinks = true,
	allcolors = blue,
        linktoc = page
]{hyperref}

% \yorkemail: takes the username and outputs a clickable mailto link.
\newcommand{\yorkemail}[1]{\href{mailto:#1@york.ac.uk}{#1@york.ac.uk}}

\setcounter{tocdepth}{1} % Only show major sections in the Table of Contents.
\addbibresource{report.bib}

\title[An Investigation of Elementary Category Theory]{An Investigation of %
        Elementary Category Theory, with applications in Pure Mathematics and %
        Theoretical Computer Science}

\author{Oliver Dixon}  \email{\yorkemail{od641}}
\author{Matthew Drury} \email{\yorkemail{md1499}}
\author{Ben Brook}     \email{\yorkemail{bb1170}}

\address{Department of Mathematics \\
        University of York \\
        United Kingdom}
\date{Spring--Summer Term, 2023}

\begin{document}
\begin{abstract}
        TODO
\end{abstract}
\maketitle
\tableofcontents
\section{\for{toc}{\texorpdfstring{\textbf{[MD]~}}{}}Theoretical %
        Underpinnings: Axiomatic Constructions}
MD: TODO

\section{\for{toc}{\texorpdfstring{\textbf{[BB]~}}{}}Category-Theoretic %
        Interpretations of Familiar Structures}
BB: TODO

\section{\for{toc}{\texorpdfstring{\textbf{[OD]~}}{}}Further Applications: %
        Functional Programming and \texorpdfstring{$\lambda$}{Lambda}-Calculus}
\subsection{Functional Programming: The Problem} In purely functional languages,
there is no allowance for context, or mutable variables of any kind. Each
function must accept some data, perform some strict transformation upon the
data---as defined by the algorithm---and return the result. Whilst this robust
paradigm does open a wide range of mathematical avenues involving proof, safety,
and reproducibility, the prohibition of stateful computation renders many common
tasks, such as system I/O or communication over a network socket, largely
impossible, as these imperatively defined operations inherently contravene the
purity principles of functional programming.

Haskell is a commonly used purely functional programming language, and
suffers, as do all languages in the same class, from this blaring issue.
Indeed, early versions of Haskell did not support the chaining of stateful
computation in any sense, due to the obligatory absence of a fixed execution
order in functional paradigms; programmers were forced to resort to breaking the
purity of the language through aesthetically unpleasant techniques, ultimately
obviating the mathematical essence of the Haskell formal type system.

Category Theory can assist in the solution to this problem, in the form of
monads, which are \emph{monoids in the category of endofunctors}. To unpack this
definition, some more theory needs developing.

\subsection{Monoidal Categories}
\subsection{Monoids}
\subsection{The Category of Endofunctors and Monads}
\subsection{The $\cathask$ Category}

\addtocontents{toc}{\protect\vspace{5pt}}
\printbibliography[title=Cited Works]
\end{document}

