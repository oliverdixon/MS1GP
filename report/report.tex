% OWD 2023

\documentclass[10pt,a4paper]{amsart}
\usepackage[foot]{amsaddr}
\usepackage{xcolor,lmodern}
\usepackage{mathtools}

\usepackage[%
	backend = biber,
	sorting = none,
	bibencoding = utf8,
	style = alphabetic
]{biblatex}

\usepackage[%
	colorlinks = true,
	allcolors = blue,
        linktoc = page
]{hyperref}

% -- START: COMMANDS FOR TESTING THE SKELETON REPORT -- %

\usepackage{lipsum}
\usepackage[%
        color = gray,
        angle = 90,
        anchor = lm,
        hpos = 4em,
        fontsize = .06\paperwidth
]{draftwatermark}

\DraftwatermarkOptions{text=\textbf{Skeleton Report}}
\nocite{*}

% -- END: COMMANDS FOR TESTING THE SKELETON REPORT -- %

% \yorkemail: takes the username and outputs a clickable mailto link.
\newcommand{\yorkemail}[1]{\href{mailto:#1@york.ac.uk}{#1@york.ac.uk}}

\addbibresource{report.bib}

\title[An Investigation of Elementary Category Theory]{An Investigation of %
        Elementary Category Theory, with applications in Pure Mathematics and %
        Theoretical Computer Science}

\author{Oliver Dixon}  \email{\yorkemail{od641}}
\author{Matthew Drury} \email{\yorkemail{md1499}}
\author{Ben Brook}     \email{\yorkemail{bb1170}}

\address{Department of Mathematics \\
        University of York \\
        United Kingdom}
\date{Spring--Summer Term, 2023}

\begin{document}
%
\begin{abstract}
        \lipsum[1-1]
\end{abstract}
\maketitle
\tableofcontents
\section{Introduction: ``A Bird's-Eye View of Mathematics?''}
\addtocontents{toc}{\protect\vspace{5pt}}
\lipsum[2-8]
\section{\for{toc}{\texorpdfstring{\textbf{[MD]~}}{}}Theoretical %
        Underpinnings: Axiomatic Constructions}
\lipsum[9-28]
\section{\for{toc}{\texorpdfstring{\textbf{[BB]~}}{}}Category-Theoretic %
        Interpretations of Familiar Structures}
        Sets are a familiar structure that, with functions, fit nicely into category theory. This is easy to demonstrate.
        
        Let $A$ be a set and $f \colon A \to A$ be a function. We will denote $1_A \coloneqq f$.
        Now let $f \colon A \to B$, $y \colon B \to C$, and $h \colon C \to D$ with $A$, $B$, $C$, and $D$ being sets. We have that
        \[
        \left( f \circ g \right) \circ h = f \circ \left( g \circ h \right) \textrm{.}
        \]
        Hence the category axioms are satisfied, and there is a category whose objects are all the sets and morphisms are all the functions between those sets. We'll denote this category Sets moving forwards.

\section{\for{toc}{\texorpdfstring{\textbf{[OD]~}}{}}Further Applications: %
        Functional Programming and \texorpdfstring{$\lambda$}{Lambda}-Calculus}
\lipsum[49-68]
\addtocontents{toc}{\protect\vspace{5pt}}
\printbibliography[title=Cited Works and Further Reading]
%
\end{document}

