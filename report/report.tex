% OWD 2023

\documentclass[10pt,a4paper]{amsart}
\usepackage[foot]{amsaddr}
\usepackage{xcolor,lmodern}

\usepackage[%
	backend = biber,
	sorting = none,
	bibencoding = utf8,
	style = alphabetic
]{biblatex}

% OWD 2023

% DO NOT COMPILE THIS FILE IN ISOLATION! It is designed to be included by other
% TeX files, and will not produce anything on its own (other than a swathe of
% meaningless temporary files and an incomprehensible error message).
%
% Whenever you need a macro involving categories, or load a package for the
% typesetting thereof, add it here, since it's likely that most macros will be
% mutually required between the report and presentation.

\usepackage{amsmath}
\usepackage{tikz-cd} % Drawing commutative diagrams with TikZ.
\usepackage{adjustbox} % Scaling tikz-cd commutative diagrams.

\tikzcdset{outer sep=2pt} % Apply some padding to arrows in comm. diagrams.

% Provide a thicker arrow style (1pt) for functor morphisms in pseudo-CDs.
\tikzset{FUNCTOR/.style = {line width=#1},%
         FUNCTOR/.default=1pt}

% Style note: please typeset named categories (e.g. Hask) in boldface, as shown
% below. For arbitrary categories (e.g. C or C_0), use \arbcat. For the homset,
% use the built-in \hom macro (\cathom is superfluous).

\newcommand{\arbcat}[1]{\ensuremath\mathcal{#1}}
\newcommand{\arbfunc}[1]{\ensuremath #1} % I might change this to \mathbf{...}

\DeclareMathOperator{\catobj}{ob}
\DeclareMathOperator{\catid}{id}
\DeclareMathOperator{\catcat}{\mathbf{Cat}}
\DeclareMathOperator{\catendo}{\mathbf{Endo}}
\DeclareMathOperator{\cathask}{\mathbf{Hask}}
\DeclareMathOperator{\catset}{\mathbf{Set}}
\DeclareMathOperator{\catpos}{\mathbf{Pos}}
\DeclareMathOperator{\catkvect}{\mathit{K}-\mathbf{Vect}}
\DeclareMathOperator{\catfinvect}{\mathbf{FinVect}_\mathbb{R}}
\DeclareMathOperator{\catpre}{\mathbf{Ord}}
\DeclareMathOperator{\catpoint}{\mathbf{Based}}
\DeclareMathOperator{\catdeduce}{\mathbf{Deduce}}
\DeclareMathOperator{\catfun}{\mathbf{Fun}}



\usepackage[%
	colorlinks = true,
	allcolors = blue,
        linktoc = page
]{hyperref}

% \yorkemail: takes the username and outputs a clickable mailto link.
\newcommand{\yorkemail}[1]{\href{mailto:#1@york.ac.uk}{#1@york.ac.uk}}

\setcounter{tocdepth}{1} % Only show major sections in the Table of Contents.
\addbibresource{report.bib}

\title[An Investigation of Elementary Category Theory]{An Investigation of %
        Elementary Category Theory, with applications in Pure Mathematics and %
        Theoretical Computer Science}

\author{Oliver Dixon}  \email{\yorkemail{od641}}
\author{Matthew Drury} \email{\yorkemail{md1499}}
\author{Ben Brook}     \email{\yorkemail{bb1170}}

\address{Department of Mathematics \\
        University of York \\
        United Kingdom}
\date{Spring--Summer Term, 2023}

\begin{document}
\begin{abstract}
        TODO
\end{abstract}
\maketitle
\tableofcontents
\section{\for{toc}{\texorpdfstring{\textbf{[MD]~}}{}}Theoretical %
        Underpinnings: Axiomatic Constructions}
\subsection{The idea of a Category}
A category is a mathematical structure that links a collection of objects with non-symmetric relationships called morphisms.
Each morphism can be said to traverse from one object to another.
They are used to create abstract models of mathematical concepts and structures based on the role each object plays.
The objects and morphisms are extremely versatile in what they can represent as shown in the commutative diagrams below:

\begin{equation}
\begin{tikzcd} 
        A \ar[r,"Father"] & B \ar[d,"Brother"]\\
        D & C \ar[l,"Husband"]
\end{tikzcd}
%
\begin{tikzcd}
A
\end{tikzcd}
%
\begin{tikzcd}
A
\end{tikzcd}
\end{equation}
These diagrams are usually used to give a sample of a category to demonstrate it's properties.
Mathematicians like to generalize with categories and consider cases with infinite objects and morphisms.
The most valuable mathematical facts tend to be those that apply to the most situations possible.
For instance, uncovering the quadratic formula being more important than solving a particularly difficult quadratic.
It makes all non-trivial quadratics easier to solve and uncovers a method to determine whether the solutions are real or complex using the "$b^2-4ac$" term.

Morphisms are combined together in a process called composition.
It means to do one morphism and then another, given that the target object of the first morphism is the same as the source of the second morphism.
Above we can see that "person A" has father "person B", who has brother "person C".
We can compose these two together to have a morphism from A to C which we can call "uncle".
These composite morphisms are implied to exist by the diagram and such are usually redundant to show.
This can be done as many times as one wants, so if there is a path to traverse from any object to another, there is a composite morphism between the two.

There are also identity morphisms.
These go from an object to itself and are equivalent to not doing a morphism at all,
as when they are composed with another morphism $f$, the resulting composite morphism is equal to $f$.
The morphism has done nothing, like how multiplying by 1(The multiplicative identity) does not change it's value.
It is similar to stating that something is equal to itself which is a trivial fact but an important feature of the structure of categories.
It should be noted that it may not be the only morphism from one object to itself.

The point of categories is to get more subtle notions of similarity or "sameness".
By creating categories we abstract things to their roles and by comparing categories or spotting patterns and structures within them,
we can see properties shared by things that would initially seem very different.
Lines of thought in one context can be more easily translated into another by seeing equivalent structural features.
Mathematicians can then try new methods to solve problems and prove new facts or describe multiple problems as one that is more general.
\subsection{Axioms and notation}
We will now formally define a category. The following axioms are necessary for a structure to be considered a category.
\paragraph{Axiom 1: Objects}
In a category $C$ there is $ob(C)$, which is a collection of all objects associated with $C$.
\paragraph{Axiom 2: Morphisms}
For each pair ordered pair of objects $(A,B)$ in a category $C$ we have $hom(A,B)$, also written as $C(A,B)$.
This is the collection of all morphisms from A to B, short for "homomorphism set".
Homomorphisms are those that preserve the structure of objects and for many useful categories they are necessary for the axioms to hold.
However, this property does not need to hold to use this notation.
A morphism $f$ is notated to map $A$ to $B$ by $f: A\to B$.
\paragraph{Axiom 3: Composition}
For objects $A,B,C\in ob(C)$ if there exists morphisms $(f: A\to B)\in hom(A,B)$ and $(g: B\to C)\in hom(B,C)$, then there exists $(g\circ f: A\to C)\in hom(A,C)$
\paragraph{Axiom 4: Identities}
Every object $A$ in a category $C$ has an identity morphism which maps from $A$ to itself.
It is notated $1_A: A\to A$ for $A\in ob(C), 1_A\in hom(A,A)$
For $f\in hom(A,_), g\in hom(_,A)$ the identity has the property that $f\circ 1_A = f, 1_A\circ g = g$

Each identity is unique to it's object. If there was 2 identities $I_1,I_2\in A$ then $f\circ I_1 = f = f\circ I_2 \implies I_1=I_2$
\paragraph{Axiom 5: Asscociativity}
A morphism $h\circ(g\circ f) = (h\circ g)\circ f$. 
This means that two morphisms are equal if they are composed from the same morphisms in the same order
regardless of the order in which we compute the individual composite pairs.
\subsection{Size of categories}
$ob(C)$ and the collections $hom(A,B)$ do not have to be finite.
The idea of category size means to place categories in a hierarchy of containment.
This is to accommodate how $ob()$ and $hom()$ do not have to be sets either.

When defining an infinite collection of objects by their properties, it can create contradictions.
Famously in set theory there is Russel's paradox.
It states that if you have a set S that contains every set which doesn't contain itself,
then suppose that S does not contain S, 
it would imply that S is in fact in S as it does not contain itself and vice versa.
Formally, $S = {x \text{is a set} | x\notin x}, (S\in S) \iff (S\notin S)$.

There were many examples of these paradoxes
that were uncovered in the early 20th century.
Mathematicians are too rigorous to allow such a thing,
so to solve the issue they devised axiomatic systems to limit the comprehension of sets to those that would not cause paradoxes.

Categories address this issue in a more elegant way.
Firstly, categories do not claim to contain elements; the core concept is instead relationships.
Secondly, categories have a hierarchy of containment such that they only contain categories "smaller" than themselves.
It follows then that categories are not defined in a self referential manner like in Russel's paradox.
\paragraph{Definitions}
A small category is where $ob(c)$ and each $hom()$ can be described as a set.

A locally small category is where neither $ob(c)$ or $hom()$ form a set. 

A large category is a totality of a small category, meaning it has a set of small categories as it's objects.
The hierarchy means that if you wanted a category where small categories are objects,
it would have to be a large category.
Then, if you wanted a category of large categories, you would need a super-large category and so on.
This can be used to formally think about multiple layers of generalization and abstraction.
%Commutative diagram thingy showing a hierarchy from individual numbers up to types of numbers up to further layers.
\section{\for{toc}{\texorpdfstring{\textbf{[BB]~}}{}}Category-Theoretic %
        Interpretations of Familiar Structures}
BB: TODO

\section{\for{toc}{\texorpdfstring{\textbf{[OD]~}}{}}Further Applications: %
        Functional Programming and \texorpdfstring{$\lambda$}{Lambda}-Calculus}
\subsection{Functional Programming: The Problem} In purely functional languages,
there is no allowance for context, or mutable variables of any kind. Each
function must accept some data, perform some strict transformation upon the
data---as defined by the algorithm---and return the result. Whilst this robust
paradigm does open a wide range of mathematical avenues involving proof, safety,
and reproducibility, the prohibition of stateful computation renders many common
tasks, such as system I/O or communication over a network socket, largely
impossible, as these imperatively defined operations inherently contravene the
purity principles of functional programming.

Haskell is a commonly used purely functional programming language, and
suffers, as do all languages in the same class, from this blaring issue.
Indeed, early versions of Haskell did not support the chaining of stateful
computation in any sense, due to the obligatory absence of a fixed execution
order in functional paradigms; programmers were forced to resort to breaking the
purity of the language through aesthetically unpleasant techniques, ultimately
obviating the mathematical essence of the Haskell formal type system.

Category Theory can assist in the solution to this problem, in the form of
monads, which are \emph{monoids in the category of endofunctors}. To unpack this
definition, some more theory needs developing.

\subsection{Monoidal Categories}
\subsection{Monoids}
\subsection{The Category of Endofunctors and Monads}
\subsection{The $\cathask$ Category}

\addtocontents{toc}{\protect\vspace{5pt}}
\printbibliography[title=Cited Works]
\end{document}

