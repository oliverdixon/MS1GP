% OWD 2023

\documentclass[10pt,a4paper]{amsart}
\usepackage[foot]{amsaddr}
\usepackage{xcolor,lmodern}

\usepackage[%
	backend = biber,
	sorting = none,
	bibencoding = utf8,
	style = alphabetic
]{biblatex}

\usepackage[%
	colorlinks = true,
	allcolors = blue,
        linktoc = page
]{hyperref}

% \yorkemail: takes the username and outputs a clickable mailto link.
\newcommand{\yorkemail}[1]{\href{mailto:#1@york.ac.uk}{#1@york.ac.uk}}

\addbibresource{report.bib}

\title[An Investigation of Elementary Category Theory]{An Investigation of %
        Elementary Category Theory, with applications in Pure Mathematics and %
        Theoretical Computer Science}

\author{Oliver Dixon}  \email{\yorkemail{od641}}
\author{Matthew Drury} \email{\yorkemail{md1499}}
\author{Ben Brook}     \email{\yorkemail{bb1170}}

\address{Department of Mathematics \\
        University of York \\
        United Kingdom}
\date{Spring--Summer Term, 2023}

\begin{document}
\begin{abstract}
        TODO
\end{abstract}
\maketitle
\tableofcontents
\section{\for{toc}{\texorpdfstring{\textbf{[MD]~}}{}}Theoretical %
        Underpinnings: Axiomatic Constructions}
MD: TODO

\section{\for{toc}{\texorpdfstring{\textbf{[BB]~}}{}}Category-Theoretic %
        Interpretations of Familiar Structures}
BB: TODO

\section{\for{toc}{\texorpdfstring{\textbf{[OD]~}}{}}Further Applications: %
        Functional Programming and \texorpdfstring{$\lambda$}{Lambda}-Calculus}
\subsection{Functional Programming: The Problem} In purely functional languages,
there is no allowance for context, or mutable variables of any kind. Each
function must accept some data, perform some strict transformation upon the
data---as defined by the algorithm---and return the result. Whilst this robust
paradigm does open a wide range of mathematical avenues involving proof, safety,
and reproducibility, the prohibition of stateful computation renders many common
tasks, such as system I/O or communication over a network socket, largely
impossible, as these imperatively defined operations inherently contravene the
purity principles of functional programming.

Haskell is a commonly used purely functional programming language, and
suffers, as do all languages in the same class, from this blaring issue.
Indeed, early versions of Haskell did not support the chaining of stateful
computation in any sense, due to the obligatory absence of a fixed execution
order in functional paradigms; programmers were forced to resort to breaking the
purity of the language through aesthetically unpleasant techniques, ultimately
obviating the mathematical essence of the Haskell formal type system.

\addtocontents{toc}{\protect\vspace{5pt}}
\printbibliography[title=Cited Works]
\end{document}

