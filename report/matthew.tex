\documentclass{amsart}
\begin{document}
\subsection{What is a category?}
A cateogory is a mathematical structure that links
a collection of objects with non-symetric connections called morphisms.
Categories are an extremely versatile tool to abstract situations.
The objects and morphisms can represent any data one can concieve of and fulfill any role.
Either or neither objects and morphisms could form a notion of structure.
They could be numbers, operations, relations, people, places.

//some commutative diagrams as examples of elementary/real world categories like families

It is useful to see categories as just items with arrows between them like above but it should be noted that for each object, there is an identity morphism that points to itself.
Every time morphisms can be combined, such as how a parent's brother is an uncle, it is said a composite morphism exists, a direct connection from you to your uncle.
Compositions and identities are often not shown as they are redundant information given that it is known the diagram is meant to be a category.

Abstracting with categories and comparing them can allow us to spot similarities between situations that intuitively seem very different as each situation has been abstracted into the same format.
This comparison may help us understand one of the situations better or try an approach to proofs or computations around them based on the similarities to the other category.
For different contexts equivalent concepts could be more easily translated over, such as how we can think of adding numbers and then similarly add sets and vectors. 
Putting an idea into a category in itself can also lead to new lines of thought as it is taking away all of the context and presenting it in a new way.
Abstracting to motivate reasoning is an essential part of mathematics for instance how algebra has helped us think about numbers and arithmetic.

\subsection{Axioms and notation}
\subsection{Sizes of categories}
\subsection{Functors}
\subsection{The category of categories}
\subsection{Yoneda's Lemma}
\end{document}