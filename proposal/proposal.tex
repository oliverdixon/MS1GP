% OWD, 2023.

\documentclass{article}
\usepackage[a4paper]{geometry}
\usepackage{fancyhdr,lastpage,tikz-cd,adjustbox,listings,subcaption}
\usepackage[
	backend=biber,
	sorting=none,
	bibencoding=utf8,
	style=alphabetic
]{biblatex}

\title{Mathematical Skills I: Group Project Proposal}
\author{Oliver Dixon, Matthew Drury, and Ben Brook}
\newcommand{\semleader}{Dr. Chris Wood}

\renewcommand{\headrulewidth}{.5pt}
\renewcommand{\footrulewidth}{.5pt}

\fancypagestyle{titlepage}{%
	\renewcommand{\headrulewidth}{0pt}
	\fancyhf{}
        \fancyfoot[R]{Page \thepage{} of~\pageref{LastPage}}
}

\makeatletter
\fancypagestyle{genpage}{%
	\fancyhf{}
        \fancyhead[L]{\@title}
        \fancyfoot[L]{\@date}
        \fancyfoot[R]{Page \thepage{} of~\pageref{LastPage}}
}
\makeatother

% TODO: Clean up this pseudo-TikZ language definition.

\lstdefinelanguage{tikz-tmp}[LaTeX]{TeX}{%
        morekeywords = {swap, arrow, below},%
        morestring = [b]{"}
}

\lstset{language = tikz-tmp,
        showstringspaces = false,%
        columns = flexible,%
        basicstyle = {\small\ttfamily},%
        keywordstyle = \color{blue},%
        stringstyle = \color{darkgray},%
        emphstyle = {\color{darkgray}\bfseries},%
        breaklines = true,%
        breakatwhitespace = true,%
        tabsize = 4,%
        gobble = 24,%
        otherkeywords = {\&, \\\\},
        emph = {&, \\\\}
}

\lstset{language = tikz-tmp}
\addbibresource{proposal.bib}
\nocite{*}

\begin{document}
%
\thispagestyle{titlepage}
\pagestyle{genpage}
\makeatletter
\noindent\large\textbf{\@title}\\[0.8\baselineskip]
\normalsize Students: \@author\\
Seminar Leader: \semleader\\
Compilation Date: \@date%
\makeatother
%
\vspace{1.5\baselineskip}\hrule\vspace{\baselineskip}
\textit{Working Title:} ``An Undergraduate Investigation into Elementary
Category Theory, with applications in Pure Mathematics and Theoretical Computer
Science.'' A brief index of likely reference works, and an example Ti\textit{k}Z
commutative diagram, can be found overleaf.
\vspace{\baselineskip}\hrule\vspace{\baselineskip}
%
\begin{itemize}
        \item \textbf{Matthew Drury.} ``Axiomatic constructions of the general
        category; an investigation of functors, duality, pullbacks, and
        implications of the Yoneda lemma.''

        \textit{MD} will investigate and explore the foundational concepts of
        Category Theory, including a survey of its most basic axiomatic
        constructions, a discovery of concepts exclusively describable in
        category-theoretic language, such as the general isomorphism, and
        provide a framework for the contributions of \textit{BB} and
        \textit{OD}. Also comprising \textit{MD}'s contribution will be various
        \TeX-typeset commutative diagrams to succinctly visualise the
        relationships of objects and morphisms in simple categories. Although
        the respective contributions of each group member is expected to be
        reasonably distinct, \textit{MD}'s section will likely intersect, to
        some degree, with the sections of \textit{BB} and \textit{OD}, due to
        its fundamentalist nature. An investigation of Sheaf Theory was also
        considered, in which a discussion of the role of sheaves, pre-sheaves,
        and profunctors in a CT-context would take place, however this was
        deemed overly ambitious.
        %
        \item \textbf{Ben Brook.} ``Some category-theoretic constructions of
        familiar structures in Abstract Algebra, Topology, and Set Theory.''

        As done in many introductory texts on Category Theory, \textit{BB} will
        aim to take prominent structures from Abstract Algebra, Topology, and
        Set Theory, and proceed to describe the core structures of their
        constructions through category-theoretic language. Although there are a
        plethora of examples in this direction---posets; based sets with
        point-preserving maps; preorders; monoids---some more exotic cases will
        also be covered, potentially on a type-theoretic stage. \textit{BB}'s
        contribution will also, inevitably, involve extensive typesetting of
        commutative diagrams and general \LaTeX\ documents, in both article
        and Beamer contexts.
        %
        \item \textbf{Oliver Dixon.} ``Practical applications and examples of
        isomorphic relations to generalised descriptions of functional
        programming paradigms, with respect to $\lambda$-calculus.''

        \textit{OD} will review the most direct and modern application of
        Category Theory, with respect to formal systems of logic and functional
        programming whilst drawing upon tangential results from Theoretical
        Computer Science and, to a substantially lesser but still notable
        degree, applied Analytic Philosophy. There exist deep connexions between
        these fields and that of $\lambda$-calculus, and hence, the topics
        discussed will defer to this universal model where appropriate. There
        will be a slight intersection between the contributions of \textit{OD}
        and \textit{BB}, given the inherent theme of applicability; in
        particular, \textit{OD} will introduce various categories pertaining to
        structures found in logical systems, such as the category of Deduction
        under some fixed formal logic.
\end{itemize}
%
\pagebreak % TODO: Fix the bibliography heading.
\printbibliography[title={Supporting Works of Reference\\[-.5em]}]
\section*{An Example Commutative Diagram}
\begin{figure}[h!]
        \centering
        \begin{subfigure}{.31\linewidth}
        \adjustbox{scale=1.25,center}{%
                \begin{tikzcd}
                        A \arrow[r, "1_A"] \arrow[rd, "f\circ 1_A" swap] &
                                A \arrow[d, "f"] \arrow[rd, "1_B\circ f"] \\
                        & B \arrow[r, "1_B" below] & B
                \end{tikzcd}
        }
        \end{subfigure}\hfill%
        \begin{subfigure}{.61\linewidth}
                \begin{lstlisting}
                        \begin{tikzcd}
                            A \arrow[r, "1_A"] \arrow[rd, "f\circ 1_A" swap] &
                                A \arrow[d, "f"] \arrow[rd, "1_B\circ f"] \\
                            & B \arrow[r, "1_B" below] & B
                        \end{tikzcd}
                \end{lstlisting}
        \end{subfigure}
        \caption{A commutative diagram, showing objects $A,B$, and the morphism
        $f:A\longrightarrow B$, with explicit identity morphisms. The
        corresponding \LaTeX-Ti\textit{k}Z source is shown to the right. Such
        diagrams will be abundant throughout the final dissertation and
        presentation.}
\end{figure}
%
\end{document}

